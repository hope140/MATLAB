\documentclass{article}
\usepackage{ctex}                               % 显示中文,更改字体
\usepackage{indentfirst}                        % 首行缩进
\usepackage{setspace}                           % 调整页行距
\usepackage{geometry}                           % 设置页边距
\usepackage{makecell}                           % 处理表格
\usepackage{amsmath}                            % 公式处理
\usepackage{amssymb}                            % 公式处理
\usepackage{enumerate}                          % 编号排版
\usepackage{algorithmic}


\geometry{left=3.18cm,right=3.18cm,top=2.54cm,bottom=2.54cm}
%设置页边距,此处参考word默认间距

\begin{document}
\setcounter{page}{87}                           % 文章页码从87开始重新编排
{\heiti 证明.}我们直接对矩阵做初等变换:
%   \begin{pmatrix}
%       A + B    & O \\
%       A^2 + AB & O
%   \end{pmatrix}
\begin{equation*}                               % 带*的公式环境没有编号
    \begin{split}                               % 公式环境内的对齐方式(split)
        &\left(                                 % 在需要对齐的元素前&,换行使用\\实现
        \begin{array}{cc}                       % 尝试matrix,此为latex宏包实现
            A + B & O \\                        % Markdown环境下无法使用
            O     & O % 建议多使用array加括号实现
        \end{array}
        \right)
        \longrightarrow                         % long+right+arrow 长+右+箭头(大写l即双箭头)
        \left(
        \begin{array}{cc}
            A + B    & O \\
            A^2 + AB & O
        \end{array}
        \right)
        =
        \left(
        \begin{array}{cc}
            A + B & O \\
            A^2   & O
        \end{array}
        \right) \\
        \longrightarrow
        &\left(
        \begin{array}{cc}
            A + B & A^2 + BA \\
            A^2   & A^3
        \end{array}
        \right)
        =
        \left(
        \begin{array}{cc}
            A + B & A^2 \\
            A^2   & A^3
        \end{array}
        \right)
        \longrightarrow
        \left(
        \begin{array}{cc}
            B & A^2 \\
            O & A^3
        \end{array}
        \right)
    \end{split}
\end{equation*}

\noindent{注:第一个箭头表示第1行左乘A加到第2行;第二个箭头表示第1行右乘A加到第二列;第三个箭头表示第2行右乘$-P$加到第1列.}   % 去除此段段首缩进

所以$r \left(A + B\right)
    =
    r \left(\begin{array}{cc}
            A + B & O \\
            O     & O
        \end{array}\right)
    =
    r \left(\begin{array}{cc}
            B & A^2 \\
            O & A^3
        \end{array}\right)
    \ge
    r \left(A^3\right) + r \left(B\right)
    =
    r \left(A\right) + r \left(B\right)$
;而显然又有$r \left(A + B\right) \le \left(A\right) + r \left(B\right)$,所以$r \left(A + B\right) = r \left(A\right) + r \left(B\right)$.

\vspace{1ex}
同样的方法,请思考下面的例题:

\vspace{1ex}
\heiti{例题 4.24.}\kaishu{已知 $A$ , $B$都是$n$级方阵,且$AB = BA =O$,证明:存在正整数$m$使得$r \left(A^m + B^m\right) = r \left(A^m\right) + r\left(B^m\right)$.$\left(\text{试问:}m = n\text{可以吗?}\right)$}

当然,矩阵变换的手段多变,还有少数题目的技巧根本想不到,为此我们可以另辟蹊径,用方程组的手段去解决.扬哥列举一个:

\vspace{1ex}
\heiti{例题 4.25.}\kaishu{设$A$,$B$是数域$P$上的$n$级矩阵,且$AB = BA$,证明r $\left(A\right) + r \left(B\right) \ge r \left(AB\right) + r \left(A + B\right)$.}

\heiti{证明. 方法一. }用分块矩阵的方法,我们知道
\begin{equation*}
    \left(
    \begin{array}{cc}
        A & O \\
        O & B
    \end{array}
    \right)
    \longrightarrow
    \left(
    \begin{array}{cc}
        A & O \\
        A & B
    \end{array}
    \right)
    \longrightarrow
    \left(
    \begin{array}{cc}
        A & A     \\
        A & A + B
    \end{array}
    \right).
\end{equation*}
结合$AB = BA$,我们知道
\begin{equation*}
    \left(
    \begin{array}{cc}
            A & A     \\
            A & A + B
        \end{array}
    \right)
    \underbrace{\left(              % under + brace 下+括号
        \begin{array}{cc}
            A + B & O \\
            -A    & E
        \end{array}
        \right)}_{\text{非广义初等变换,难以想到}}
    =
    \left(
    \begin{array}{cc}
            AB & A     \\
            O  & A + B
        \end{array}
    \right).
\end{equation*}
\noindent{于是}
\begin{equation*}
    r \left(A\right) + r \left(B\right)
    =
    \left(
    \begin{array}{cc}
        A & O \\
        O & B
    \end{array}
    \right)
    =
    \left(
    \begin{array}{cc}
        A & A     \\
        A & A + B
    \end{array}
    \right)
    \ge
    r \left(
    \begin{array}{cc}
        AB & A     \\
        O  & A + B
    \end{array}
    \right)
    \ge
    r \left(AB\right) + r \left(A + B\right).
\end{equation*}
\heiti{方法二.}设方程组$AX = 0$与$BX = 0$的解空间分别是$V_1$,$V_2$,方程组$ABX = BAX = 0$与$\left(A + B\right)X = 0$的解空间分别为$W_1$,$W_2$,则$V_1\subseteq W_1$,$V_2\subseteq W_1$,从而$V_1 + V_2\subseteq W_1$,同时$V_1\cap V_2 \subseteq W_1$,利用维数公式就有
\begin{equation*}
    \mathrm{dim}V_1 + \mathrm{dim}V_2 = \mathrm{dim}\left(V_1 + V_2\right) + \mathrm{dim}\left(V_1 \cap V_2\right) \le \mathrm{dim}W_1 + \mathrm{dim}W_2.
\end{equation*}

即

\begin{equation*}
    \left(n - r \left(A\right)\right) + \left(n - r \left(B\right)\right) \le \left(n - r \left(AB\right)\right) + \left(n - r \left(A + B\right)\right).
\end{equation*}

即$r \left(A\right) + r \left(B\right) \ge r \left(AB\right) + r \left(A + B\right)$.

\vspace{1ex}
\noindent{\heiti{4.7.3 打洞原理证明秩不等式}}

\vspace{1ex}
\heiti{命题4.5.}\kaishu{已知$A$,$D$分别为$n$级与$m$级矩阵,且$A$可逆,$B$,$C$分别是$n\times m$与$m\times n$矩阵,利用打洞原理有}
\begin{equation*}
    \left(
    \begin{array}{cc}
        A & B \\
        C & D
    \end{array}
    \right)
    \longrightarrow
    \left(
    \begin{array}{cc}
        A & O          \\
        O & D-CA^{-1}B
    \end{array}
    \right).
\end{equation*}

所以

\begin{equation*}
    r
    \left(
    \begin{array}{cc}
            A & B \\
            C & D
        \end{array}
    \right)
    =
    r \left(A\right) + r \left(D - CA^{-1}B\right)
    =
    n + r \left(D - CA^{-1}B\right).
\end{equation*}

\heiti{例题4.26.}\kaishu{已知$A$是一个$s\times n$矩阵,证明$r \left(E_n - A'A\right)-r \left(E_s - AA'\right) = n - s$.}

\vspace{1ex}
\heiti{证明.}对$\left(\begin{array}{cc}
            E_n & A'  \\
            A   & E_s
        \end{array}\right)$利用打洞原理有
\begin{equation*}
    \left(
    \begin{array}{cc}
        E_n - A'A & O   \\
        O         & E_s
    \end{array}
    \right)
    \longleftarrow
    \left(
    \begin{array}{cc}
        E_n & A'  \\
        A   & E_s
    \end{array}
    \right)
    \longrightarrow
    \left(
    \begin{array}{cc}
        E_n & O         \\
        O   & E_s - AA'
    \end{array}
    \right).
\end{equation*}

所以$r \left(
    \begin{array}{cc}
            E_n - A'A & O   \\
            O         & E_s
        \end{array}
    \right)
    =
    r \left(
    \begin{array}{cc}
            E_n & O         \\
            O   & E_s - AA'
        \end{array}
    \right)$,即$s + r \left(E_n - A'A\right) = n + r \left(E_s - AA'\right)$,即
\begin{equation*}
    r \left(E_n - A'A\right) = n + r \left(E_s - AA'\right) = n - s
\end{equation*}

\heiti{例题4.27.}\kaishu{已知$A$是$n$级可逆矩阵,$\alpha$ ,$\beta$ 是任意两个$n$维列向量,则$r \left(A + \alpha \beta '\right) \ge n - 1$.}

\vspace{2ex}
\noindent{\heiti{\zihao{4} 4.8 等价标准形}}

\vspace{1ex}
在讨论矩阵的等价标准形之前,我们先给出了一个重要的引理,它在矩阵分解中有非常重要的应用,可以达到出奇制胜的效果.

\vspace{1ex}
\heiti{引理4.1.} $s \times n$的矩阵$\left(\begin{array}{cc}
            E_r & O \\
            O   & O
        \end{array}\right)$有一种极其重要的分解:
\begin{equation*}
    \left(
    \begin{array}{cc}
        E_r & O \\
        O   & O
    \end{array}
    \right)
    =
    \left(
    \begin{array}{c}
        E_r \\
        O
    \end{array}
    \right)
    \left(
    \begin{array}{cc}
        E_r & O
    \end{array}
    \right).
\end{equation*}
当$\left(
    \begin{array}{cc}
            E_r & O \\
            O   & O
        \end{array}
    \right)$是方阵时,还有另外一种分解:
\begin{equation*}
    \left(
    \begin{array}{cc}
        E_r & O \\
        O   & O
    \end{array}
    \right)
    =
    \left(
    \begin{array}{cc}
        E_r & O \\
        O   & O
    \end{array}
    \right)
    \left(
    \begin{array}{cc}
        E_r & O \\
        O   & O
    \end{array}
    \right).
\end{equation*}

\heiti{Theorem 4.9} $\left(\text{等价标准形}\right)$. \kaishu{设$A$是一个秩为$r$的$s \times n$矩阵,则存在$s$级可逆矩阵$P$与$n$级可逆矩阵$Q$使得}
\begin{equation*}
    A = P \left(
    \begin{array}{cc}
        E_r & O \\
        O   & O
    \end{array}
    \right)Q.
\end{equation*}

\heiti{命题4.6.} \kaishu{任意一个非零矩阵都可以分解成一个列满秩矩阵与一个行满秩矩阵的乘积.}

\heiti{证明.} 设矩阵$A$是$s \times n$矩阵,且$r \left(A\right) = r > 0$,则存在$s$级可逆矩阵$P$与$n$级可逆矩阵$Q$使得











\end{document}
