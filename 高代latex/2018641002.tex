\documentclass{article}
\usepackage{ctex}                               % 显示中文,更改字体
\usepackage{indentfirst}                        % 首行缩进
\usepackage{setspace}                           % 调整页行距
\usepackage{geometry}                           % 设置页边距
\usepackage{makecell}                           % 处理表格
\usepackage{amsmath}                            % 公式处理
\usepackage{amssymb}                            % 公式处理
\usepackage{enumerate}                          % 编号排版
\usepackage{algorithmic}


\geometry{left=3.18cm,right=3.18cm,top=2.54cm,bottom=2.54cm}
%设置页边距,此处参考word默认间距

\begin{document}
\setcounter{page}{87}                           % 文章页码从87开始重新编排
{\heiti 证明.}我们直接对矩阵做初等变换:
%   \begin{pmatrix}
%       A + B    & 0 \\
%       A^2 + AB & 0
%   \end{pmatrix}
\begin{equation*}                               % 带*的公式环境没有编号
    \begin{split}                               % 公式环境内的对齐方式(split)
        &\left(                                 % 在需要对齐的元素前&,换行使用\\实现
        \begin{array}{cc}                       % 尝试matrix,此为latex宏包实现
            A + B & 0 \\                        % Markdown环境下无法使用
            0     & 0                           % 建议多使用array加括号实现
        \end{array}
        \right)
        \longrightarrow                         % long+right+arrow 长+右+箭头(大写l即双箭头)
        \left(
        \begin{array}{cc}
            A + B    & 0 \\
            A^2 + AB & 0
        \end{array}
        \right)
        =
        \left(
        \begin{array}{cc}
            A + B & 0 \\
            A^2   & 0
        \end{array}
        \right) \\
        \longrightarrow
        &\left(
        \begin{array}{cc}
            A + B & A^2 + BA \\
            A^2   & A^3
        \end{array}
        \right)
        =
        \left(
            \begin{array}{cc}
                A + B & A^2 \\
                A^2 & A^3
            \end{array}
        \right)
        \longrightarrow
        \left(
            \begin{array}{cc}
                B & A^2 \\
                0 & A^3
            \end{array}
        \right)
    \end{split}
\end{equation*}







\end{document}
