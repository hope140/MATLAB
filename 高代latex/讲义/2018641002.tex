\documentclass{article}
\usepackage{ctex}                               % 显示中文,更改字体
\usepackage{indentfirst}                        % 首行缩进
\usepackage{setspace}                           % 调整页行距
\usepackage{geometry}                           % 设置页边距
\usepackage{makecell}                           % 处理表格
\usepackage{amsmath}                            % 公式处理
\usepackage{amssymb}                            % 公式处理
\usepackage{amsthm}                             % 数学字符
\usepackage{mathrsfs}                           % 数学字符
\usepackage{enumerate}                          % 编号排版
\usepackage{algorithmic}

\geometry{left=1.5cm,right=1.5cm,top=2cm,bottom=2cm}
%设置页边距,此处参考word默认间距

\begin{document}
\setcounter{page}{87}                           % 文章页码从87开始重新编排
{\heiti 证明.}我们直接对矩阵做初等变换:
%   \begin{pmatrix}
%       A + B    & O \\
%       A^2 + AB & O
%   \end{pmatrix}
\begin{equation*}                               % 带*的公式环境没有编号
    \begin{split}                               % 公式环境内的对齐方式(split)
        &\left(                                 % 在需要对齐的元素前&,换行使用\\实现
        \begin{array}{cc}                       % 尝试matrix,此为latex宏包实现
            A + B & O \\                        % Markdown环境下无法使用
            O     & O % 建议多使用array加括号实现
        \end{array}
        \right)
        \longrightarrow                         % long+right+arrow 长+右+箭头(大写l即双箭头)
        \left(
        \begin{array}{cc}
            A + B    & O \\
            A^2 + AB & O
        \end{array}
        \right)
        =
        \left(
        \begin{array}{cc}
            A + B & O \\
            A^2   & O
        \end{array}
        \right) \\
        \longrightarrow
        &\left(
        \begin{array}{cc}
            A + B & A^2 + BA \\
            A^2   & A^3
        \end{array}
        \right)
        =
        \left(
        \begin{array}{cc}
            A + B & A^2 \\
            A^2   & A^3
        \end{array}
        \right)
        \longrightarrow
        \left(
        \begin{array}{cc}
            B & A^2 \\
            O & A^3
        \end{array}
        \right)
    \end{split}
\end{equation*}

\noindent{注:第一个箭头表示第1行左乘A加到第2行;第二个箭头表示第1行右乘A加到第二列;第三个箭头表示第2行右乘$-P$加到第1列.}   % 去除此段段首缩进

所以$r \left(A + B\right)
    =
    r \left(\begin{array}{cc}
            A + B & O \\
            O     & O
        \end{array}\right)
    =
    r \left(\begin{array}{cc}
            B & A^2 \\
            O & A^3
        \end{array}\right)
    \ge
    r \left(A^3\right) + r \left(B\right)
    =
    r \left(A\right) + r \left(B\right)$
;而显然又有$r \left(A + B\right) \le \left(A\right) + r \left(B\right)$,所以$r \left(A + B\right) = r \left(A\right) + r \left(B\right)$.

\vspace{1ex}
同样的方法,请思考下面的例题:

\vspace{1ex}
{\heiti 例题 4.24.}{\kaishu 已知 $A$ , $B$都是$n$级方阵,且$AB = BA =O$,证明:存在正整数$m$使得$r \left(A^m + B^m\right) = r \left(A^m\right) + r\left(B^m\right)$.$\left(\text{试问:}m = n\text{可以吗?}\right)$}

当然,矩阵变换的手段多变,还有少数题目的技巧根本想不到,为此我们可以另辟蹊径,用方程组的手段去解决.扬哥列举一个:

\vspace{1ex}
{\heiti 例题 4.25.}{\kaishu 设$A$,$B$是数域$P$上的$n$级矩阵,且$AB = BA$,证明r $\left(A\right) + r \left(B\right) \ge r \left(AB\right) + r \left(A + B\right)$.}

{\heiti 证明. 方法一. }用分块矩阵的方法,我们知道
\begin{equation*}
    \left(
    \begin{array}{cc}
        A & O \\
        O & B
    \end{array}
    \right)
    \longrightarrow
    \left(
    \begin{array}{cc}
        A & O \\
        A & B
    \end{array}
    \right)
    \longrightarrow
    \left(
    \begin{array}{cc}
        A & A     \\
        A & A + B
    \end{array}
    \right).
\end{equation*}
结合$AB = BA$,我们知道
\begin{equation*}
    \left(
    \begin{array}{cc}
            A & A     \\
            A & A + B
        \end{array}
    \right)
    \underbrace{\left(              % under + brace 下+括号
        \begin{array}{cc}
            A + B & O \\
            -A    & E
        \end{array}
        \right)}_{\text{非广义初等变换,难以想到}}
    =
    \left(
    \begin{array}{cc}
            AB & A     \\
            O  & A + B
        \end{array}
    \right).
\end{equation*}
\noindent{于是}
\begin{equation*}
    r \left(A\right) + r \left(B\right)
    =
    \left(
    \begin{array}{cc}
        A & O \\
        O & B
    \end{array}
    \right)
    =
    \left(
    \begin{array}{cc}
        A & A     \\
        A & A + B
    \end{array}
    \right)
    \ge
    r \left(
    \begin{array}{cc}
        AB & A     \\
        O  & A + B
    \end{array}
    \right)
    \ge
    r \left(AB\right) + r \left(A + B\right).
\end{equation*}
{\heiti 方法二.}设方程组$AX = 0$与$BX = 0$的解空间分别是$V_1$,$V_2$,方程组$ABX = BAX = 0$与$\left(A + B\right)X = 0$的解空间分别为$W_1$,$W_2$,则$V_1\subseteq W_1$,$V_2\subseteq W_1$,从而$V_1 + V_2\subseteq W_1$,同时$V_1\cap V_2 \subseteq W_1$,利用维数公式就有
\begin{equation*}
    \mathrm{dim}V_1 + \mathrm{dim}V_2 = \mathrm{dim}\left(V_1 + V_2\right) + \mathrm{dim}\left(V_1 \cap V_2\right) \le \mathrm{dim}W_1 + \mathrm{dim}W_2.
\end{equation*}

即

\begin{equation*}
    \left(n - r \left(A\right)\right) + \left(n - r \left(B\right)\right) \le \left(n - r \left(AB\right)\right) + \left(n - r \left(A + B\right)\right).
\end{equation*}

即$r \left(A\right) + r \left(B\right) \ge r \left(AB\right) + r \left(A + B\right)$.

\vspace{1ex}
\noindent{{\heiti 4.7.3 \quad 打洞原理证明秩不等式}}

\vspace{1ex}
{\heiti 命题4.5.}{\kaishu 已知$A$,$D$分别为$n$级与$m$级矩阵,且$A$可逆,$B$,$C$分别是$n\times m$与$m\times n$矩阵,利用打洞原理有}
\begin{equation*}
    \left(
    \begin{array}{cc}
        A & B \\
        C & D
    \end{array}
    \right)
    \longrightarrow
    \left(
    \begin{array}{cc}
        A & O          \\
        O & D-CA^{-1}B
    \end{array}
    \right).
\end{equation*}

所以

\begin{equation*}
    r
    \left(
    \begin{array}{cc}
            A & B \\
            C & D
        \end{array}
    \right)
    =
    r \left(A\right) + r \left(D - CA^{-1}B\right)
    =
    n + r \left(D - CA^{-1}B\right).
\end{equation*}

{\heiti 例题4.26.}{\kaishu 已知$A$是一个$s\times n$矩阵,证明$r \left(E_n - A'A\right)-r \left(E_s - AA'\right) = n - s$.}

\vspace{1ex}
{\heiti 证明.}对$\left(\begin{array}{cc}
            E_n & A'  \\
            A   & E_s
        \end{array}\right)$利用打洞原理有
\begin{equation*}
    \left(
    \begin{array}{cc}
        E_n - A'A & O   \\
        O         & E_s
    \end{array}
    \right)
    \longleftarrow
    \left(
    \begin{array}{cc}
        E_n & A'  \\
        A   & E_s
    \end{array}
    \right)
    \longrightarrow
    \left(
    \begin{array}{cc}
        E_n & O         \\
        O   & E_s - AA'
    \end{array}
    \right).
\end{equation*}

所以$r \left(
    \begin{array}{cc}
            E_n - A'A & O   \\
            O         & E_s
        \end{array}
    \right)
    =
    r \left(
    \begin{array}{cc}
            E_n & O         \\
            O   & E_s - AA'
        \end{array}
    \right)$,即$s + r \left(E_n - A'A\right) = n + r \left(E_s - AA'\right)$,即
\begin{equation*}
    r \left(E_n - A'A\right) = n + r \left(E_s - AA'\right) = n - s.
\end{equation*}

{\heiti 例题4.27.}{\kaishu 已知$A$是$n$级可逆矩阵,$\alpha$ ,$\beta$ 是任意两个$n$维列向量,则$r \left(A + \alpha \beta '\right) \ge n - 1$.}

\vspace{2ex}
\noindent{{\heiti \zihao{4} 4.8 \quad 等价标准形}}

\vspace{1ex}
在讨论矩阵的等价标准形之前,我们先给出了一个重要的引理,它在矩阵分解中有非常重要的应用,可以达到出奇制胜的效果.

\vspace{1ex}
{\heiti 引理4.1.} $s \times n$的矩阵$\left(\begin{array}{cc}
            E_r & O \\
            O   & O
        \end{array}\right)$有一种极其重要的分解:
\begin{equation*}
    \left(
    \begin{array}{cc}
        E_r & O \\
        O   & O
    \end{array}
    \right)
    =
    \left(
    \begin{array}{c}
        E_r \\
        O
    \end{array}
    \right)
    \left(
    \begin{array}{cc}
        E_r & O
    \end{array}
    \right).
\end{equation*}
当$\left(
    \begin{array}{cc}
            E_r & O \\
            O   & O
        \end{array}
    \right)$是方阵时,还有另外一种分解:
\begin{equation*}
    \left(
    \begin{array}{cc}
        E_r & O \\
        O   & O
    \end{array}
    \right)
    =
    \left(
    \begin{array}{cc}
        E_r & O \\
        O   & O
    \end{array}
    \right)
    \left(
    \begin{array}{cc}
        E_r & O \\
        O   & O
    \end{array}
    \right).
\end{equation*}

{\heiti Theorem 4.9} $\left(\text{等价标准形}\right)$. {\kaishu 设$A$是一个秩为$r$的$s \times n$矩阵,则存在$s$级可逆矩阵$P$与$n$级可逆矩阵$Q$使得}
\begin{equation*}
    A = P \left(
    \begin{array}{cc}
        E_r & O \\
        O   & O
    \end{array}
    \right)Q.
\end{equation*}

{\heiti 命题4.6.} {\kaishu 任意一个非零矩阵都可以分解成一个列满秩矩阵与一个行满秩矩阵的乘积.}

{\heiti 证明.} 设矩阵$A$是$s \times n$矩阵,且$r \left(A\right) = r > 0$,则存在$s$级可逆矩阵$P$与$n$级可逆矩阵$Q$使得
\begin{equation*}
    A=P\left(\begin{array}{cc}
            E_{r} & O \\
            O     & O
        \end{array}\right)
    \underline{Q=P\left(\begin{array}{c}
            E_{r} \\
            O
        \end{array}\right)}
    \underline{\left(\begin{array}{ll}
            E_{r} & O
        \end{array}\right) Q.}
\end{equation*}
现在记$P_1 = P \left(
    \begin{array}{c}
            E_r \\
            O
        \end{array}
    \right)$,$Q_1 = \left(
    \begin{array}{cc}
            E_r & O
        \end{array}
    \right)Q$,则$A = P_1Q_1$,且$P_1$是$s \times r$的列满秩矩阵,$Q_1$是$r \times n$的行满秩矩阵.

注. 上面解答过程中,不能写$P \left(
    \begin{array}{c}
            E_r \\
            O
        \end{array}
    \right)
    =
    \left(
    \begin{array}{c}
            P \\
            O
        \end{array}
    \right)$,原因是$P$不能与$E_r$做乘法.

{\heiti 例题4.28.}{\kaishu 设$B_1$,$B_2$都是数域$P$上的$s \times n$的列满秩矩阵,证明:存在数域$P$上的$s$级可逆矩阵$C$使得$B_2 = CB_1$.}

{\heiti 证明.} 由于$B_1$,$B_2$都列满秩,所以存在可逆的$s$级矩阵$Q$,$R$使得
\begin{equation*}
    Q B_{1}=\left(\begin{array}{c}
            E_{r} \\
            O
        \end{array}\right)=R B_{2}.
\end{equation*}

所以$B_2 = R^{-1}QB_1$,即取$C = R^{-1}Q$即可.

{\heiti 例题4.29.}{\kaishu 都是数域$P$上的$s \times n$的行满秩矩阵,证明:存在数域$P$上的$n$级可逆矩阵$D$使得$C_2 = C_1D$.}

{\heiti 例题4.30.} {\kaishu 任意秩为$r \left(r > 0\right)$的矩阵都可以分解成$r$个秩为1的矩阵之和.}

{\heiti 证明.}设$A_{s \times n}$是一个秩为$r$的矩阵,则存在$s$级与$n$级可逆矩阵$P_1Q$使得
\begin{equation*}
    A=P\left(\begin{array}{cc}
            E_{r} & O \\
            O     & O
        \end{array}\right) Q=P\left(E_{11}+E_{22}+\cdots+E_{r r}\right) Q=P E_{11} Q+P E_{22} Q+\cdots+P E_{r r} Q.
\end{equation*}

由于$E_{i i} \left(i = 1,2, \cdots , r\right)$的秩为1,所以$PE_{i i}Q$的秩也为1.

{\heiti 例题4.31.} {\kaishu 已知$A$是一个秩为$r$的$s \times n$矩阵,求矩阵方程$AXA = A$的通解.}

{\heiti 证明.} 由于$r \left(A\right) = r$,所以存在可逆矩阵$P_1Q$使得$A = P \left(
    \begin{array}{cc}
            E_r & O \\
            O   & O
        \end{array}
    \right)$,代入到$AXA = A$就有
\begin{equation*}
    P\left(\begin{array}{cc}
        E_{r} & O \\
        O     & O
    \end{array}\right) Q X P\left(\begin{array}{cc}
        E_{r} & O \\
        O     & O
    \end{array}\right) Q=P\left(\begin{array}{cc}
        E_{r} & O \\
        O     & O
    \end{array}\right) Q.
\end{equation*}
消去 $P, Q$ 就有
\begin{equation*}
    \left(\begin{array}{cc}
        E_{r} & O \\
        0     & O
    \end{array}\right) Q X P\left(\begin{array}{cc}
        E_{r} & O \\
        0     & O
    \end{array}\right)=\left(\begin{array}{cc}
        E_{r} & O \\
        0     & O
    \end{array}\right).
\end{equation*}
现在对 $Q X P$ 分块, 设 $Q X P=\left(\begin{array}{cc}H & B \\ C & D\end{array}\right)$, 代入上式就有
\begin{equation*}
    \left(\begin{array}{cc}
        E_{r} & O \\
        O     & O
    \end{array}\right)\left(\begin{array}{cc}
        H & B \\
        C & D
    \end{array}\right)\left(\begin{array}{cc}
        E_{r} & O \\
        O     & O
    \end{array}\right)=\left(\begin{array}{cc}
        E_{r} & O \\
        O     & O
    \end{array}\right).
\end{equation*}
化简即得
\begin{equation*}
    \left(\begin{array}{ll}
        H & O \\
        O & O
    \end{array}\right)=\left(\begin{array}{cc}
        E_{r} & O \\
        O     & O
    \end{array}\right).
\end{equation*}
从而得到$H = E_r$,而$B$,$C$,$D$可以任意取,所以$Q X P = \left(
    \begin{array}{cc}
            E_r & B \\
            C   & D
        \end{array}
    \right)$,解出$X$就有
\begin{equation*}
    X=Q^{-1}\left(\begin{array}{cc}
            E_{r} & B \\
            C     & D
        \end{array}\right) P^{-1}.
\end{equation*}
其中 $B, C, D$ 分别是任意的 $r \times(s-r),(n-r) \times r,(n-r) \times(s-r)$ 矩阵.

\vspace{1ex}
{\heiti 扬哥经验: 一个矩阵题目中,如果什么条件都没有或者只告诉了矩阵的秩,记得考虑一下等价标准形.}

\vspace{2ex}
\noindent{{\heiti \zihao{4} \quad 4.9 矩阵的迹与幂零矩阵}}
\vspace{2ex}

一个方阵$A$的所有主对角元素之和称为$A$的迹,记为$t r \left(A\right)$.迹常用的性质总结为如下定理:

{\heiti Throrem 4.10.}
{\kaishu 已知$A$,$B$是两个$n$级矩阵,$k$是一个常数,则

(1)$tr \left(kA\right) =ktr \left(A\right)$.

(2)$tr \left(A + B\right) = tr \left(A\right) + tr \left(B\right)$.

(3)$tr \left(AB\right) = tr \left(BA\right)$.

(4)如果$A$是一个实方阵,则$A = O \Longleftrightarrow tr \left(A'A\right) = 0$.}

\vspace{1ex}
{\heiti 证明.} (1),(2)是显然的.

(3)就是求和号交换顺序,读者自己证.注意本条是迹最重要的性质.

(4)注意到$tr \left(A'A\right)$等于$A$的所有元素的平方和.

\vspace{1ex}
{\heiti 例题4.32} {\kaishu 已知$A$是一个$n$级实对称方阵,且$A^2 = O$,则$A = O$.}

{\heiti 证明.} 由于$A' = A$,所以$tr \left(A'A\right) = tr \left(A^2\right) = 0$,所以$A = O$.

利用相似的知识,我们知道

{\heiti Theorem 4.11.} (1)相似的矩阵有相同的迹.

(2)数域$h'$上$n$级矩阵$A$的迹等于其$n$个复特征值之和.

\vspace{1ex}
对于一个$n$级矩阵$A$,如果存在整数$l$使得$A' = O$则称$A$是一个幂零矩阵.如果还有$A^{l - 1} \neq O$,则称$l$为$A$的幂零指数.一个等价的命题就是:如果方阵$A$的所有特征值都为零,则称$A$是一个幂零矩阵.根据哈密顿-凯莱定理,这也是显然的.

下面,我们给出关于幂零矩阵的一个常识性的命题:

\vspace{1ex}
{\heiti 命题4.7.} {\kaishu 已知$A$是数域$P$上的$n$级矩阵,则$A$是幂零矩阵的充要条件是对任意的正整数$k$都有$tr \left(A^k\right) = 0$.}

{\heiti 证明.} 必要性. 显然当$A$是幂零矩阵时,$A$只有零特征值,与是$A^k$也只有零特征值,当然有$tr \left(A^k\right) = 0$.

充分性. 假设 $\lambda_{1}, \lambda_{2}, \cdots, \lambda_{n}$ 是 $A$ 的 $n$ 个复特征值, 我们取 $k=1,2, \cdots, n$, 根哲 $\mathrm{tr}\left(A^k\right) = 0$ 可得
\begin{equation*}
    \left\{\begin{array}{c}
        \lambda_{1}+\lambda_{2}+\cdots+\lambda_{n}=0             \\
        \lambda_{1}^{2}+\lambda_{2}^{2}+\cdots+\lambda_{n}^{2}=0 \\
        \vdots                                                   \\
        \lambda_{1}^{n}+\lambda_{2}^{n}+\cdots+\lambda_{n}^{n}=0.
    \end{array}\right.
\end{equation*}
由 32 页的例题 $2.29$ 得 $\lambda_{1}=\lambda_{2}=\cdots=\lambda_{n}=0$, 这说明 $A$ 是幂零矩阵.

\vspace{1ex}
{\heiti 例题 4.33.} {\kaishu 已知 $M_{n}(\mathbb{C})$ 表示所有 $n$ 级复矩阵组成的线性空间, $\sigma: M_{n}(\mathbb{C}) \longrightarrow \mathbb{C}$ 是一个线性映射, 并 且满足对任意的 $A, B \in M_{n}(\mathbb{C})$, 都有 $\sigma(A B)=\sigma(B A)$, 证明存在 $\lambda \in \mathbb{C}$ 使得对任意的 $A \in M_{n}(\mathbb{C})$, 都有
    $\sigma(A)=\lambda \operatorname{tr}(A)$}

\vspace{1ex}
{\heiti 证明.} 我们用基本矩阵 $E_{i j}$ 的性质来解答:

由于 $\sigma$ 是一个线性映射, 所以对任意的 $A, B \in M_{n}(\mathbb{C})$, 有 $\sigma(A B-B A)=\sigma(A B)-\sigma(B A)=0$, 现在取
$A=E_{i j}, B=E_{j i}$, 就有
\begin{equation*}
    \sigma\left(E_{i j} E_{j i}\right)-\sigma\left(E_{j i} E_{i j}\right)=\sigma\left(E_{i i}-E_{j j}\right)=\sigma\left(E_{i i}\right)-\sigma\left(E_{j j}\right)=0 .
\end{equation*}
所以对任意的 $i, j$ 都有 $\sigma\left(E_{i i}\right)=\sigma\left(E_{j j}\right)$, 我们设 $\sigma\left(E_{11}\right)=\sigma\left(E_{22}\right)=\cdots=\sigma\left(E_{n n}\right)=\lambda .$
当 $i \neq j$ 时, 有 $\sigma\left(E_{i j}\right)=\sigma\left(E_{i k} E_{k j}\right)=\sigma\left(E_{k j} E_{i k}\right)=\sigma(O)=0$, 再结合线代性质, 对任意的矩阵$A=\left(a_{i j}\right)=\sum_{i, j=1}^{n} a_{i j} E_{i j}$,有
\begin{equation*}
    \sigma(A)=\sigma\left(\sum_{i, j=1}^{n} a_{i j} E_{i j}\right)=\sum_{i, j=1}^{n} a_{i j} \sigma\left(E_{i j}\right)=\sum_{i=1}^{n} a_{i i} \sigma\left(E_{i i}\right)=\lambda \sum_{i=1}^{n} a_{i i}=\lambda \operatorname{tr}(A).
\end{equation*}

{\heiti 注.} 过程中我们用到基本矩阵的性质: $E_{i k} E_{k j}=E_{i j}$, 当 $k \neq l$ 时, 有 $E_{i k} E_{l j}=O .$

接下来是一个不怎么用到的命题, 也算是一个常识:

{\heiti 命题 4.8.} {\kaishu $U$ 是 $P^{n \times n}$ 空间中所有形如 $A B-B A$ 的矩阵生成的子空间, $W$ 是 $P^{n \times n}$ 中所有迹为零的矩
阵生成的线性子空间, 则 $U=W$.}

{\heiti 证明.} 显然 $U$ 中的矩阵迹都为零, 所以 $U \subseteq W .$ 与此同时, 我们知道有 $E_{i j}(i \neq j)$ 及 $E_{i i}-E_{11}$ $(i=2,3, \cdots, n)$ 是 $W$ 的一组基, 而利用基本矩阵的性质, 有
\begin{equation*}
    \left\{\begin{array}{l}
        E_{i j}=E_{i i} E_{i j}-E_{i j} E_{i i} \in U,(i \neq j) ; \\
        E_{i i}-E_{11}=E_{i 1} E_{1 i}-E_{1 i} E_{i 1} \in U,(i=2,3, \cdots, n).
    \end{array}\right.
\end{equation*}
于是 $W \subseteq U$, 所以 $U=W$.

    {\heiti 注.} 以上说明 $E_{i j}(i \neq j)$ 及 $E_{i i}-E_{11}(i=2,3, \cdots, n)$ 是 $U=W$ 的一组基, 且 $\operatorname{dim} U=\operatorname{dim} W=n^{2}-1$.

接下来我们考虑一类问题:

\vspace{1ex}
{\heiti 例题 4.34.} {\kaishu 证明: 在有限维线性空问上, 不存在满足 $\mathscr{A} \mathscr{B}-\mathscr{B} \mathscr{A}=\mathscr{E}$ 的线性变换 $\mathscr{A}, \mathscr{B} .$ 但在无限维线性空间与无限维线性空问在本质上是有区别的. 幸好我们大多研究的是有限维线性空间.}

{\heiti 证明.} 利用线代变换与矩阵的一一对应性, 我们只需证明不存在 $n$ 级矩阵 $A, B$ 使得 $A B-B A=E$ 即可, 而这是显然的: 因为 $\operatorname{tr}(A B-B A)=0$, 而 $\operatorname{tr}(E)=n$, 所以必然有 $A B-B A \neq E$.

\vspace{1ex}
{\heiti 例题 4.35.} {\kaishu 已知数域 $K$ 上的两个 $n$ 级矩阵 $A, B$ 满足 $A B-B A=A$, 则 $A$ 不可逆.}

\vspace{1ex}
{\heiti 证明.} 反证法. 假设 $A$ 可逆, 那么由 $A B-B A=A$ 两边左乘 $A^{-1}$ 可得 $B-A^{-1} B A=E$, 而这是不可能
的,因为
\begin{equation*}
    \operatorname{tr}\left(B-A^{-1} B A\right)=\operatorname{tr}(B)-\operatorname{tr}\left(A^{-1} \underline{B A}\right)=\operatorname{tr}(B)-\operatorname{tr}\left(\underline{B A} A^{-1}\right)=0 \neq n=\operatorname{tr}(E)
\end{equation*}

{\heiti 例题4.36.} {\kaishu 已知数域 $K$ 上的两个 $n$ 级矩阵 $A, B$ 满足 $A B-B A=A$, 则对任意的正整数 $k$, 都有$\operatorname{tr}\left(A^{k}\right)=0$ (即 $A$ 是幂零矩阵).}

\vspace{1ex}
{\heiti 证明.} 套路:
\begin{equation*}
    A^{k}=A^{k-1} A=A^{k-1}(A B-B A)=A^{k} B-A^{k-1} B A.
\end{equation*}
于是
\begin{equation*}
    \operatorname{tr}\left(A^{k}\right)=\operatorname{tr}\left(A^{k} B\right)-\operatorname{tr}\left(\underline{A^{k-1} B} A\right)=\operatorname{tr}\left(A^{k} B\right)-\operatorname{tr}\left(A A^{k-1} B\right)=0.
\end{equation*}
同时, 据命题 4.7 可得 $A$ 是幂零矩阵.

{\heiti 例题 4.37 .} {\kaishu 设 $A, B, C$ 是数域 $K$ 上的 $n$ 级矩阵, 满足 $A B-B A=C$, 且 $A C=C A$,证明: 对任意的正整数$k$,都有$tr \left(C^k\right) = 0$ (即$C$是幂零矩阵).}

\vspace{1ex}
{\heiti 证明.} 首先, 注意套路: 如果 $A C=C A$, 则对任意的多项式 $f(x), g(x)$ 都有 $f(A) g(C)=g(C) f(A)$, 特别地, $A C^{k-1}=C^{k-1} A(k \geq 2)$, 于是
\begin{equation*}
    \operatorname{tr}\left(C^{k}\right)=\operatorname{tr}\left(C^{k-1} C\right)=\operatorname{tr}\left(C^{k-1}(A B-B A)\right)=\operatorname{tr}\left(\underline{C^{k-1} A B}\right)-\operatorname{tr}\left(C^{k-1} B A\right)=\operatorname{tr}\left(\underline{A C^{k-1}} B\right)-\operatorname{tr}\left(A C^{k-1} B\right)=0.
\end{equation*}
据命题 4.7 可得 $C$ 是幂零矩阵.

\vspace{1ex}
{\heiti 例题4.38.} {\kaishu 已知数域 $K$ 上的两个 2 级矩阵 $A, B$ 满足 $A B - B A = A$,则$A^2 = O$.}

\vspace{1ex}
{\heiti 证明.} 由例 4.35 知道 $A$ 不可逆, 从而 $A$ 的秩为 1 或 0, 结合 $4.4$ 节的知识点得 $A^{2}=\operatorname{tr}(A) A$, 而由例 4.36 可知 $\operatorname{tr}(A)=0$, 于是 $A^{2}=O .$

\vspace{2ex}
\noindent{{\heiti \zihao{4} 4.10 \quad 分块乘法与初等变换的强调与应用}}

\vspace{2ex}
我们知道在数字矩阵中, 做一次初等行(列)变换其实就对应左(右)乘一个初等矩阵, 这个性质对分块矩阵也是适用的. 只不过分块矩阵里面, 左乘右乘是严格区分的, 拿第三类变换来说:

我们知道
\begin{equation*}
    \left(\begin{array}{cc}
        E_{m} & O     \\
        P     & E_{n}
    \end{array}\right)\left(\begin{array}{cc}
        A & B \\
        C & D
    \end{array}\right)=\left(\begin{array}{cc}
        A     & B     \\
        C+P A & D+P B
    \end{array}\right).
\end{equation*}
这里就要求 $P A$ 与 $P B$ 有意义, 同时万万不能写成 $A P$ 或者 $B P .$ 这就体现了行变换对应左乘(第 1 行左乘 $P$ 倍加到第 2 行 $),$ 也说明 $\left(\begin{array}{cc}A & B \\ C & D\end{array}\right)$ 与 $\left(\begin{array}{cc}A & B \\ C+P A & D+P B\end{array}\right)$ 是等价的.

同样地, 我们给 $\left(\begin{array}{cc}A & B \\ C & D\end{array}\right)$ 右边乘一个广义初等矩阵 $\left(\begin{array}{cc}E_{m} & Q \\ 0 & E_{n}\end{array}\right)$ 得到 $\left(\begin{array}{cc}A & B+A Q \\ C & D+C Q\end{array}\right) .$ 这体现了
列变换对应右乘(第 1 列右乘 $Q$ 倍加到第 2 列), 总体写出来就是
\begin{equation*}
    \left(\begin{array}{cc}
        A     & B     \\
        C+P A & D+P B
    \end{array}\right) \leftarrow\left(\begin{array}{cc}
        A & B \\
        C & D
    \end{array}\right) \longrightarrow\left(\begin{array}{cc}
        A & B+A Q \\
        C & D+C Q
    \end{array}\right).
\end{equation*}
我们考虑一些特殊的问题, 称之为打洞原理, 这是玩转矩阵{\heiti 最重要}的手段!
1. 当 $A$ 可逆时, 可以通过行变换干掉 $C$ 如下
\begin{equation*}
    \left(\begin{array}{ll}
        A & B \\
        C & D
    \end{array}\right) \longrightarrow\left(\begin{array}{cc}
        A & B            \\
        O & D-C A^{-1} B
    \end{array}\right).
\end{equation*}
写成矩阵等式就是
\begin{equation*}
    \left(\begin{array}{cc}
        E         & O \\
        -C A^{-1} & E
    \end{array}\right)\left(\begin{array}{cc}
        A & B \\
        C & D
    \end{array}\right)=\left(\begin{array}{cc}
        A & B            \\
        O & D-C A^{-1} B
    \end{array}\right).
\end{equation*}
同时, 可以继续对 $\left(\begin{array}{cc}A & B \\ O & D-C A^{-1} B\end{array}\right)$ 做列变换用 $A$ 干掉 $B$, 即
\begin{equation*}
    \left(\begin{array}{cc}
        A & B            \\
        O & D-C A^{-1} B
    \end{array}\right)\left(\begin{array}{cc}
        E & -A^{-1} B \\
        O & E
    \end{array}\right)=\left(\begin{array}{cc}
        A & O            \\
        O & D-C A^{-1} B
    \end{array}\right).
\end{equation*}

同理, 当 $D$ 可逆时, 就可以通过 $D$ 干掉 $B, C$, 至于先干掉哪个就看读者的心情了. 扬哥写一个, 读者写出
对应的矩阵等式:
\begin{equation*}
    \left(\begin{array}{cc}
        A & B \\
        C & D
    \end{array}\right) \longrightarrow\left(\begin{array}{cc}
        A-B D^{-1} C & B \\
        O            & D
    \end{array}\right) \longrightarrow\left(\begin{array}{cc}
        A-B D^{-1} C & O \\
        O            & D
    \end{array}\right).
\end{equation*}
当然, 以上也说明 $\left|\begin{array}{ll}A & B \\ C & D\end{array}\right|=|A|\left|D-C A^{-1} B\right|=|D|\left|A-B D^{-1} C\right| .$

2. 特别地,对于对称矩阵$\left(\begin{array}{cc}
            A  & B \\
            B' & D
        \end{array}\right)$,其中$A,D$也是对称方阵,则$A$可逆时,可以通过合同变换干掉$B, B^{\prime}$, 直接有
\begin{equation*}
    \left(\begin{array}{cc}
        A          & B \\
        B^{\prime} & D
    \end{array}\right) \longrightarrow\left(\begin{array}{cc}
        A & O                     \\
        O & D-B^{\prime} A^{-1} B
    \end{array}\right).
\end{equation*}
写出矩阵等式就是
$\left(\begin{array}{cc}E & -A^{-1} B \\ O & E\end{array}\right)^{\prime}\left(\begin{array}{cc}A & B \\ B^{\prime} & D\end{array}\right)\left(\begin{array}{cc}E & -A^{-1} B \\ O & E\end{array}\right)=\left(\begin{array}{cc}A & O \\ O & D-B^{\prime} A^{-1} B\end{array}\right) \cdot($ 请重视 $)$
3. 对 $A=\left(\begin{array}{cc}A_{1} & \alpha \\ \beta^{\prime} & a_{n n}\end{array}\right)$, 假设 $A_{1}$ 可逆, 则直接有
\begin{equation*}
    \left(\begin{array}{cc}
        A_{1}          & \alpha  \\
        \beta^{\prime} & a_{n n}
    \end{array}\right) \longrightarrow\left(\begin{array}{cc}
        A_{1} & \alpha                                   \\
        0     & a_{n n}-\beta^{\prime} A_{1}^{-1} \alpha
    \end{array}\right).
\end{equation*}
写成矩阵等式就是
\begin{equation*}
    \left(\begin{array}{cc}
        E_{n-1}                    & 0 \\
        -\beta^{\prime} A_{1}^{-1} & 1
    \end{array}\right)\left(\begin{array}{cc}
        A_{1}          & \alpha  \\
        \beta^{\prime} & a_{n n}
    \end{array}\right)=\left(\begin{array}{cc}
        A_{1} & \alpha                                   \\
        0     & a_{n n}-\beta^{\prime} A_{1}^{-1} \alpha
    \end{array}\right).
\end{equation*}

以上这种将分块乘法与初等交换结合是矩阵运算中{\heiti 极端重要}的手段,同时,这种手段还往往与数学归纳法紧密结合在一起,我们这里先举几个例子,请读者{\heiti 务必反复书写},做到{\heiti 熟练}.

\vspace{1ex}
{\heiti 例题4.39.} {\kaishu 已知n级方阵A的所有顺序主子式非零,证明存在n级下三角矩阵B使得BA为上三角矩阵.}

{\heiti 证明.} 对级数 $n$ 做数学归纳法. $n=1$ 是显然的.

设命题对 $n-1$ 级矩阵成立, 当 $A$ 是 $n$ 级矩阵时, 我们可以分块设 $A=\left(\begin{array}{cc}A_{1} & \alpha \\ \beta^{\prime} & a_{n n}\end{array}\right)$, 这样分块后, 显然$A_1$的顺序主子式都是$A$的顺序主子式,所以非零,可以利用归纳假设: 存在$n - 1$级的下三角矩阵$B_1$使得 $B_{1} A_{1}$ 为上三角矩阵. 同时, 由于 $\left|A_{1}\right| \neq 0$, 所以 $A_{1}$ 可逆, 这样可以对 $A$ 通过行变换于掉 $\beta^{\prime}$, 具体为
\begin{equation*}
    \left(\begin{array}{cc}
        E_{n-1}                    & 0 \\
        -\beta^{\prime} A_{1}^{-1} & 1
    \end{array}\right)\left(\begin{array}{cc}
        A_{1}          & \alpha  \\
        \beta^{\prime} & a_{n n}
    \end{array}\right)=\left(\begin{array}{cc}
        A_{1} & \alpha                               \\
        0     & a_{n n}-\beta^{\prime} A^{-1} \alpha
    \end{array}\right).
\end{equation*}
再结合 $B_{1} A_{1}$ 为上三角, 那么
\begin{equation*}
    \left(\begin{array}{cc}
        B_{1} & 0 \\
        0     & 1
    \end{array}\right)\left(\begin{array}{cc}
        A_{1} & \alpha                               \\
        0     & a_{n n}-\beta^{\prime} A^{-1} \alpha
    \end{array}\right)=\left(\begin{array}{cc}
        B_{1} A_{1} & B_{1} \alpha                         \\
        0           & a_{n n}-\beta^{\prime} A^{-1} \alpha
    \end{array}\right).
\end{equation*}
是一个上三角矩阵, 两次运算结合起来, 记 $B=\left(\begin{array}{cc}B_{1} & 0 \\ 0 & 1\end{array}\right)\left(\begin{array}{cc}E_{n-1} & 0 \\ -\beta^{\prime} A_{1}^{-1} & 1\end{array}\right)$, 则显然 $B$ 是一个下三角矩阵,使得 $B A$ 为上三角矩阵.

{\heiti 例题 4.40 .} {\kaishu 已知 $A$ 是 $n$ 级实方阵, $A$ 的所有顺床主子式都大于零, 且非主对角上的元素都为负数, 则 $A^{-1}$ 的元索都为正数.}

{\heiti 证明.} 依旧是对 $A$ 用数学归纳法, $n=1$ 是显然的. 假设命题对 $n-1$ 级矩阵成立, 现在设 $A$ 是 $n$ 级矩阵, 对 $A$ 分块有
\begin{equation*}
    A=\left(\begin{array}{cc}
        A_{n-1}        & \alpha  \\
        \beta^{\prime} & a_{n n}
    \end{array}\right).
\end{equation*}
利用 $A_{n-1}$ 可逆, 千掉 $\alpha, \beta^{\prime}$ 就是
\begin{equation*}
    \left(\begin{array}{cc}
        E_{n-1}                      & 0 \\
        -\beta^{\prime} A_{n-1}^{-1} & 1
    \end{array}\right) A\left(\begin{array}{cc}
        E_{n-1} & -A_{n-1}^{-1} \alpha \\
        0       & 1
    \end{array}\right)=\left(\begin{array}{cc}
        A_{n-1} & 0                                          \\
        0       & a_{n n}-\beta^{\prime} A_{n-1}^{-1} \alpha
    \end{array}\right).
\end{equation*}
两边取逆后化简, 马上可得
\begin{equation*}
    A^{-1}=\left(\begin{array}{cc}
            E_{n-1} & -A_{n-1}^{-1} \alpha \\
            0       & 1
        \end{array}\right)\left(\begin{array}{cc}
            A_{n-1}^{-1} & 0                                                            \\
            0            & \left(a_{n n}-\beta^{\prime} A_{n-1}^{-1} \alpha\right)^{-1}
        \end{array}\right)\left(\begin{array}{cc}
            E_{n-1}                      & 0 \\
            -\beta^{\prime} A_{n-1}^{-1} & 1
        \end{array}\right).
\end{equation*}
注意 $|A|>0$ 可得 $a_{n n}-\beta^{\prime} A_{n-1}^{-1} \alpha>0$, 同时, 我们还知道 $\alpha, \beta$ 的元素都为负数, $A_{n-1}^{-1}$ 的元素都为正数, 于是 个矩阵 $\left(\begin{array}{cc}E_{n-1} & -A_{n-1}^{-1} \alpha \\ 0 & 1\end{array}\right)$ 与 $\left(\begin{array}{cc}A_{n-1}^{-1} & 0 \\ 0 & \left(a_{n n}-\beta^{\prime} A_{n-1}^{-1} \alpha\right)^{-1}\end{array}\right)$ 及 $\left(\begin{array}{cc}E_{n-1} & 0 \\ -\beta^{\prime} A_{n-1}^{-1} & 1\end{array}\right)$ 的所有元素都是正数, 从而 $A^{-1}$ 的所有元素也都是正数.

\vspace{1ex}
{\heiti 例题 4.41 . } {\kaishu 设 $A$ 为 $n$ 级实对称矩阵, 且 $A$ 的所有顺序主子式 $D_{1}, D_{2}, \cdots, D_{n}$ 都非零, 则 $A$ 合同于对角矩阵}
\begin{equation*}
    \operatorname{diag}\left\{D_{1}, \frac{D_{2}}{D_{1}}, \cdots, \frac{D_{n}}{D_{n-1}}\right\}.
\end{equation*}

{\heiti 证明.} 对级数 $n$ 做数学归纳法. $n=1$ 是显然的. 假设命题对 $n-1$ 级矩阵成立, 则当 $A$ 是 $n$ 级矩阵时,
对 $A$ 分块
\begin{equation*}
    A=\left(\begin{array}{cc}
        A_{1}           & \alpha  \\
        \alpha^{\prime} & a_{n n}
    \end{array}\right).
\end{equation*}
我们知道 $A_{1}$ 的顺序主子式为 $D_{1}, D_{2}, \cdots, D_{n-1}$, 结合 $A_{1}$ 是实对称矩阵, 利用假设, 我们知道存在 $n-1$ 级可
逆矩阵 $P_{1}$ 使得
\begin{equation*}
    P_{1}^{\prime} A_{1} P_{1}=\operatorname{diag}\left\{D_{1}, \frac{D_{2}}{D_{1}}, \cdots, \frac{D_{n-1}}{D_{n-2}}\right\}.
\end{equation*}

现在我用利用合同变换干掉 $A$ 中的 $\alpha$ 与 $\alpha^{\prime}:$ 自然是选取 $Q=\left(\begin{array}{cc}E_{n-1} & -A_{1}^{-1} \alpha \\ 0 & 1\end{array}\right)$, 就有
\begin{equation*}
    Q^{\prime} A Q=\left(\begin{array}{cc}
            A_{1} & 0                                         \\
            0     & a_{n n}-\alpha^{\prime} A_{1}^{-1} \alpha
        \end{array}\right).
\end{equation*}
注意到这里 $|Q|=1$, 所以上式两边取行列式可得 $|A|=\left|A_{1}\right|\left(a_{n n}-\alpha^{\prime} A_{1}^{-1} \alpha\right)$, 即 $a_{n n}-\alpha^{\prime} A_{1}^{-1} \alpha=\frac{D_{n}}{D_{n-1}} .$ 再取$P=\left(\begin{array}{cl}
            P_{1} & 0 \\
            0     & 1
        \end{array}\right)$, 则有
\begin{equation*}
    \begin{aligned}
         & P^{\prime} Q^{\prime} A Q P=\left(\begin{array}{cc}
                P_{1}^{\prime} A_{1} P_{1} & 0                                         \\
                0                          & a_{n n}-\alpha^{\prime} A_{1}^{-1} \alpha
            \end{array}\right)=\operatorname{diag}\left\{D_{1}, \frac{D_{2}}{D_{1}}, \cdots, \frac{D_{n-1}}{D_{n-2}}, \frac{D_{n}}{D_{n-1}}\right\}.
    \end{aligned}
\end{equation*}
记 $T=Q P$ 可知 $T^{\prime} A T=\operatorname{diag}\left\{D_{1}, \frac{D_{2}}{D_{1}}, \cdots, \frac{D_{n}}{D_{n-1}}\right\}$.

\vspace{1ex}
下一个例题和此例题是一回事儿, 写出来是为了让读者加深印象.

{\heiti 例题 4.42 .}{\kaishu 利用结论 " $A$ 是实矩阵, 则 $A$ 正定的充要条件是存在实可逆拒阵 $P$ 使得 $P^{\prime} A P$ 为主对角元 全大于零的对角矩阵"证明: 如果 $n$ 级实对称矩阵 $A$ 的所有顺序主子式大于零, 则 $A$ 是正定矩阵.}

{\heiti 证明.} 对级数做数学归纳法, $n=1$ 显然. 假设命题对 $n-1$ 级矩阵成立, 则对于 $n$ 级实对称矩阵 $A$, 分声
有
\begin{equation*}
    A=\left(\begin{array}{cc}
        A_{1}           & \alpha  \\
        \alpha^{\prime} & a_{n n}
    \end{array}\right).
\end{equation*}
则 $A_{1}$ 的所有顺序主子式大于零, 利用假设知道 $A_{1}$ 正定, 于是存在 $n-1$ 级可逆矩阵 $P_{1}$ 使得 $P_{1}^{\prime} A P_{1}$ 为主对 角元都为正的对角矩阵. 同时, 利用 $A_{1}$ 可逆, $A$ 可以通过合同变换可以千掉 $\alpha$ 与 $\alpha^{\prime}$, 即
\begin{equation*}
    \left(\begin{array}{cc}
        E_{n-1} & -A_{1}^{-1} \alpha \\
        0       & 1
    \end{array}\right)^{\prime}\left(\begin{array}{cc}
            A_{1}           & \alpha  \\
            \alpha^{\prime} & a_{n n}
        \end{array}\right)\left(\begin{array}{cc}
            E_{n-1} & -A_{1}^{-1} \alpha \\
            0       & 1
        \end{array}\right)=\left(\begin{array}{cc}
            A_{1} & 0                                         \\
            0     & a_{n n}-\alpha^{\prime} A_{1}^{-1} \alpha
        \end{array}\right).
\end{equation*}
再取 $Q=\left(\begin{array}{cc}P_{1} & 0 \\ 0 & 1\end{array}\right)$ 就有
\begin{equation*}
    Q^{\prime}\left(\begin{array}{cc}
            A_{1} & 0                                         \\
            0     & a_{n n}-\alpha^{\prime} A_{1}^{-1} \alpha
        \end{array}\right) Q=\left(\begin{array}{cc}
            P^{\prime} A_{1} P & 0                                         \\
            0                  & a_{n n}-\alpha^{\prime} A_{1}^{-1} \alpha
        \end{array}\right).
\end{equation*}
为主对角元都为正的对角矩阵.

两个过程合起来, 记 $P=\left(\begin{array}{cc}E_{n-1} & -A_{1}^{-1} \alpha \\ 0 & 1\end{array}\right) Q$, 就有 $P^{\prime} A P$ 为主对角元都为正的对角矩阵, 从而 $A$ 是
一个正定矩阵.

进一步, 我们还有下面的例题:

\vspace{1ex}
{\heiti 例题 4.43 .} {\kaishu 主对角元素都为 1 的上三角矩阵称为特殊上三角矩阵, 已知 $n$ 级对称矩阵 $A$ 的各阶顺序主子式都不为零, 则存在特殊的上三角矩阵 $T$ 使得 $T^{\prime} A T$ 为对角矩阵, 且 $A$ 与 $T^{\prime} A T$ 有完全相同的顺序主子式.}

\vspace{1ex}
证明. 依旧是对 $n$ 做数学归纳法. $n=1$ 是显然的, 假设命题对 $n-1$ 级矩阵成立, 现在讨论 $n$ 级矩阵的情形: 对 $A$ 分块, 设为
\begin{equation*}
    A=\left(\begin{array}{cc}
        A_{n-1}         & \alpha  \\
        \alpha^{\prime} & a_{n n}
    \end{array}\right).
\end{equation*}
显然 $A_{n-1}$ 的顺序主子式都是 $A$ 的顺序主子式, 从而都非零, 于是存在 $n-1$ 级特殊上三角矩阵 $T_{1}$ 使得 $T_{1}^{\prime} A_{n-1} T_{1}$ 为对角矩阵. 下面对 $A$ 进行合同变换干掉 $\alpha$ 与 $\alpha^{\prime}$, 即取 $P=\left(\begin{array}{cc}E_{n-1} & -A_{n-1}^{-1} \alpha \\ 0 & 1\end{array}\right)$ 就有
\begin{equation*}
    P^{\prime} A P=\left(\begin{array}{cc}
            A_{n-1} & 0                                           \\
            0       & a_{n n}-\alpha^{\prime} A_{n-1}^{-1} \alpha
        \end{array}\right).
\end{equation*}
再记 $T=P\left(\begin{array}{cc}T_{1} & 0 \\ 0 & 1\end{array}\right)$ 就有
\begin{equation*}
    T^{\prime} A T=\left(\begin{array}{cc}
            T_{n-1}^{\prime} & 0 \\
            0                & 1
        \end{array}\right) P^{\prime} A P\left(\begin{array}{cc}
            T_{n-1} & 0 \\
            0       & 1
        \end{array}\right)=\left(\begin{array}{cc}
            T_{1}^{\prime} A_{n-1} T_{1} & 0                                           \\
            0                            & a_{n n}-\alpha^{\prime} A_{n-1}^{-1} \alpha
        \end{array}\right).
\end{equation*}
是对角矩阵, 同时 $T$ 显然是特殊三角矩阵.

最后, 我们证明 $A$ 与 $T^{\prime} A T$ 有完全相同的顺序主子式: 对任意的 $k(1 \leq k \leq n)$ 我们做分块
\begin{equation*}
    A=\left(\begin{array}{ll}
        A_{1} & A_{2} \\
        A_{3} & A_{4}
    \end{array}\right), T=\left(\begin{array}{cc}
        T_{1} & T_{2} \\
        O     & T_{4}
    \end{array}\right).
\end{equation*}
其中 $A_{1}, T_{1}$ 都是 $k$ 级方阵, 且 $T_{1}, T_{4}$ 都是特殊上三角矩阵. 于是直接有
\begin{equation*}
    T^{\prime} A T=\left(\begin{array}{cc}
            T_{1}^{\prime} & O              \\
            T_{2}^{\prime} & T_{4}^{\prime}
        \end{array}\right)\left(\begin{array}{cc}
            A_{1} & A_{2} \\
            A_{3} & A_{4}
        \end{array}\right)\left(\begin{array}{cc}
            T_{1} & T_{2} \\
            0     & T_{4}
        \end{array}\right)=\left(\begin{array}{cc}
            T_{1}^{\prime} A_{1} & * \\
            *                    & *
        \end{array}\right)\left(\begin{array}{cc}
            T_{1} & T_{2} \\
            0     & T_{4}
        \end{array}\right)=\left(\begin{array}{cc}
            T_{1}^{\prime} A_{1} T_{1} & *^{\prime} \\
            *^{\prime}                 & *^{\prime}
        \end{array}\right).
\end{equation*}
这说明 $T^{\prime} A T$ 的 $k$ 级顺序主子式为 $\left|T_{1}^{\prime} A_{1} T_{1}\right|=\left|A_{1}\right|$, 即 $A$ 与 $T^{\prime} A T$ 有完全相同的顺序主子式.

{\heiti 注 1 .} 如果设 $A$ 的各阶顺序主子式为 $D_{k}(k=1,2, \cdots, n)$ 且非零, 则对角矩阵
\begin{equation*}
    T^{\prime} A T=\operatorname{diag}\left\{D_{1}, \frac{D_{2}}{D_{1}}, \cdots, \frac{D_{n}}{D_{n-1}}\right\}.
\end{equation*}
是唯一确定的.

{\heiti 注 2 .} 当 $A$ 的顺序主子式都非零时, 可得 $T$ 也是唯一确定的: 我们不妨设 $T_{1}$ 也是满足 $T_{1}^{\prime} A T_{1}=B$ 的特 殊上三角矩阵, 则
\begin{equation*}
    T^{\prime} A T=B=T_{1}^{\prime} A T_{1}
\end{equation*}
于是
\begin{equation*}
    A=\left(T^{\prime}\right)^{-1} B T^{-1}=\left(T_{1}^{\prime}\right)^{-1} B T_{1}^{-1}
\end{equation*}
即有
\begin{equation*}
    T_{1}^{\prime}\left(T^{\prime}\right)^{-1} B=B T_{1}^{-1} T
\end{equation*}
注意到 $T_{1}^{\prime}\left(T^{\prime}\right)^{-1} B$ 是一个下三角矩阵, 而 $B T_{1}^{-1} T$ 是一个上三角矩阵, 所以 $T_{1}^{\prime}\left(T^{\prime}\right)^{-1} B=B T_{1}^{-1} T$ 就是一个对
角矩阵, 且主对角元与 $B$ 对应相等, 于是
\begin{equation*}
    T_{1}^{\prime}\left(T^{-1}\right)^{\prime} B=B T_{1}^{-1} T=B
\end{equation*}
这就得到 $T_{1}^{-1} T=E$, 即 $T_{1}=T$, 即满足条件的 $T$ 也是唯一确定的.

有了分块矩阵的初等变换, 我们可以像数字矩阵那样求分块矩阵的逆, 只不过这个过程要注意左乘与右乘的事儿:

{\heiti 例题 4.44 .} {\kaishu 已知 $A, C$ 分别是 $m, n$ 级可逆矩阵, 求矩阵 $\left(\begin{array}{cc}A & B \\ O & C\end{array}\right)$ 的逆矩阵.}

\vspace{1ex}
{\heiti 证明.} 和 2 级矩阵一样, 只要注意行变换对应左乘就可以, 如下:
\begin{equation*}
    \left(\begin{array}{cccc}
        A & B & E & O \\
        0 & C & O & E
    \end{array}\right) \rightarrow\left(\begin{array}{cccc}
        E & A^{-1} B & A^{-1} & O      \\
        O & E        & O      & C^{-1}
    \end{array}\right) \rightarrow\left(\begin{array}{cccc}
        E & O & A^{-1} & -A^{-1} B C^{-1} \\
        O & E & O      & C^{-1}
    \end{array}\right).
\end{equation*}
于是 $\left(\begin{array}{ll}A & B \\ O & C\end{array}\right)$ 的逆矩阵为 $\left(\begin{array}{cc}A^{-1} & -A^{-1} B C^{-1} \\ 0 & C^{-1}\end{array}\right) .$

{\heiti 注.} 同样的方法, 我们可以求出来 $\left(\begin{array}{cc}A & B \\ C & D\end{array}\right)$ 的逆, 只不过这个结果很麻烦, 读者可以自己试着写一下.

{\heiti 例题 4.45 .} {\kaishu 已知 $A, B$ 都是可逆方阵, 求 $\left(\begin{array}{cc}O & A \\ B & O\end{array}\right)$ 的逆矩阵.}

\vspace{1ex}
再次强调: 本节{\heiti 极其重要,} 必须学会左乘与右乘, 掌握打洞原理, 对每一个例题都做到滚瓜烂熟.

\end{document}