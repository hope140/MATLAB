\documentclass{article}
\usepackage{ctex}                               % 显示中文,更改字体
\usepackage{indentfirst}                        % 首行缩进
\usepackage{setspace}                           % 调整页行距
\usepackage{geometry}                           % 设置页边距
\usepackage{makecell}                           % 处理表格
\usepackage{amsmath}                            % 公式处理
\usepackage{amssymb}                            % 公式处理
\usepackage{amsthm}                             % 数学字符
\usepackage{mathrsfs}                           % 数学字符
\usepackage{enumerate}                          % 编号排版
\usepackage{algorithmic}

\renewcommand{\baselinestretch}{1.5}
\geometry{left=1.5cm,right=1.5cm,top=2cm,bottom=2cm}
%设置页边距,此处参考word默认间距

\begin{document}
\setlength{\parindent}{2em}                       % 首行缩进两字符
% \setcounter{page}{87}                           % 文章页码从87开始重新编排
{\heiti 1.}已知4阶行列式${D}$的第3行元素分别是-1,0,2,4,第4行元素对应的余子式依次是5,10,$a$,4,求$a$的值。
\begin{enumerate}[\qquad 解:]
    \item 因为$a_{31}A_{41}+a_{32}A_{42}+a_{33}A_{43}+a_{34}A_{44}=0$,
          这里 $a_{i j}$ 和 $A$ 分别是第 ${i}$ 行第 $j$ 列处的元素和该元素的  \\
          代数余子式,所以有 $-1 \times(-5)+0 \times 10+2 \times(-a)+4 \times 4=0$,
          可得 $a=\frac{21}{2}$
\end{enumerate}

\vspace{1ex}
{\heiti 2.} 已知矩阵 $A, B$ 满足关系 $A B-B=A$, 其中 $B=\left(
    \begin{array}{ccc}
            1 & -2 & 0 \\
            2 & 1  & 0 \\
            0 & 0  & 2
        \end{array}
    \right)$
, 求矩阵$A$。
\begin{enumerate}[\qquad 解:]
    \item 因为 ${A B-A=B}$, 所以 $A(B-E)=B$, $A=B(B-E)^{-1}=\left(
              \begin{array}{ccc}
                      1            & \frac{1}{2} & 0 \\
                      -\frac{1}{2} & 1           & 0 \\
                      0            & 0           & 2
                  \end{array}
              \right)$
\end{enumerate}

\vspace{1ex}
{\heiti 3.}设 $A^{*}$ 为 $3$ 阶方阵 $A$ 的伴随矩阵, $|A|=2$, 计算行列式 $|(3 A)^{-1}-\left.\frac{1}{2} A^{*}\right|$
\begin{enumerate}[\qquad 解:]
    \item $|(3 A)^{-1}-\left.\frac{1}{2} A^{*}\right|=|\frac{1}{3}A^{-1}-A^{-1}|=|-\frac{2}{3}A^{-1}|=(-\frac{2}{3})^{3}|A^{-1}|=-\frac{4}{27}$
\end{enumerate}

\vspace{1ex}
{\heiti 4.}如果多项式 $f(x), g(x)$ 不全为零,证明 $: \frac{f(x)}{(f(x), g(x))}$ 与 $\frac{g(x)}{(f(x), g(x))}$ 互素。

\begin{enumerate}[\qquad 证:]
    \item 证: 存在多项式 $u(x), v(x),$ 使 $(f(x), g(x))=u(x) f(x)+v(x) g(x)$ \\
          因而 $u(x) \frac{f(x)}{(f(x), g(x))}+v(x) \frac{g(x)}{(f(x), g(x))}=1$ \\
          由定理 $3, \left(\frac{f(x)}{(f(x),g(x))},\frac{g(x)}{(f(x), g(x))}\right)=1$
\end{enumerate}

\vspace{1ex}
{\heiti 5.}证明: $x_{0}$ 是 $f(x)$ 的 $k$ 重根的充分必要条件是 $f\left(x_{0}\right)=f^{\prime}\left(x_{0}\right)=\cdots=f^{k-1}\left(x_{0}\right)=0$ 而$f^{k}\left(x_{0}\right) \neq 0$

\begin{enumerate}[\qquad 证:]
    \item {\heiti 必要性:} 设 $x_{0}$ 是 $f(x)$ 的 $k$ 重根。 那么 $x_{0}$ 是 $f^{\prime}(x)$ 的 $k-1$ 重根, $\cdots \cdots$, 是 $f^{t-1}(x)$ 的1 重根, 是 $f^{\prime}(x)$ 的 0 重根, 即不是 $f^{*}(x)$ 的根 \\
          所以 $f\left(x_{0}\right)=f^{\prime}\left(x_{0}\right)=\cdots=f^{k-1}\left(x_{0}\right)=0$,而 $f^{k}\left(x_{0}\right) \neq 0 .$ \\
          {\heiti 充分性: } 设 $f(x_{0})=f^{\prime}\left(x_{0}\right)=\cdots=f^{t-1}\left(x_{0}\right)=0$ 而 $f^{\prime}\left(x_{0}\right) \neq 0 .$ 设 $x_{0}$ 是 $f(x)$ 的l重根 \\
          由必要性的证明 $f\left(x_{0}\right)=f^{\prime}\left(x_{0}\right)=\cdots=f^{\prime-1}\left(x_{0}\right)=0$ 而 $f^{\prime}\left(x_{0}\right) \neq 0 .$ 从而 $l=k .$
\end{enumerate}

\vspace{1ex}
{\heiti 6.}二次型 $f\left(x_{1}, x_{2}, x_{3}\right)=\left(x_{1}+x_{2}\right)^{2}+\left(x_{2}+x_{3}\right)^{2}-\left(x_{3}-x_{1}\right)^{2}$ 的正贯性指数与负惯性指数依次为
$(A) 2,0 $ $(B) 1,1 $ $(C) 2,1 $ $(D) 1,2 $

\begin{enumerate}[\qquad 解:]
    \item $f\left(x_{1}, x_{2}, x_{3}\right)=\left(x_{1}+x_{2}\right)^{2}+\left(x_{2}+x_{3}\right)^{2}-\left(x_{3}-x_{1}\right)^{2}=2 x_{2}^{2}+2 x_{1} x_{2}+2 x_{2} x_{3}+2 x_{1} x_{3}$  \\
          所以 $A=\left(\begin{array}{lll}0 & 1 & 1  \\ 1 & 2 & 1  \\ 1 & 1 & 0\end{array}\right)$, 故特征多项式为
          $$
              |\lambda E-A|=\left|\begin{array}{rrr}
                  \lambda & -1 & -1      \\
                  -1      & -2 & -1      \\
                  -1      & -1 & \lambda
              \end{array}\right|=(\lambda+1)(\lambda-3) \lambda
          $$  \\
          令上式等于零,故特征值为$-1,3,0$,故该二次型的正惯性指数为1,负惯性指数为1,故选$B$.
\end{enumerate}

\vspace{1ex}
{\heiti 7.} 设 3 阶矩阵 $\boldsymbol{A}=\left(\alpha_{1}, \alpha_{2}, \alpha_{3}\right), \quad B=\left(\beta_{1}, \beta_{2}, \beta_{3}\right)$, 若向量组 $\alpha_{1}, \alpha_{2}, \alpha_{3}$ 可以由向量组 $\beta_{1}, \beta_{2}$线性表出,则
\begin{enumerate}[\qquad \quad (A)]
    \item $A x=0$ 的解均为 $B x=0$ 的解.
    \item $A^{T} x=0$ 的解均为 $B^{T} x=0$ 的解.
    \item $B x=0$ 的解均为 $A x=0$ 的解.
    \item $B^{T} x=0$ 的解均为 $A^{T} x=0$ 的解.
\end{enumerate}

\begin{enumerate}[\qquad 解:]
    \item 令 $A=\left(a_{1}, a_{2}, a_{3}\right), B=\left(\beta_{1}, \beta_{2}, \beta_{3}\right)$, 由题 $a_{1}, a_{2}, a_{3}$ 可由 $\beta_{1}, \beta_{2}, \beta_{3}$ 线性表示,即存在矩阵$A$,使得 $B P=A$, \\
          则当 $B^{T} x_{0}=0$ 时, $A^{T} x_{0}=(B P)^{T} x_{0}=p^{T} B^{T} x_{0}=0 .$ 恒成立,即选 $\mathrm{D}$.
\end{enumerate}

\vspace{1ex}
{\heiti 8.}已知 $\alpha_{1}=\left(\begin{array}{l}1  \\ 0  \\ 1\end{array}\right), \alpha_{2}=\left(\begin{array}{l}1  \\ 2  \\ 1\end{array}\right), \alpha_{3}=\left(\begin{array}{l}3  \\ 1  \\ 2\end{array}\right)$, 记 $\beta_{1}=\alpha_{1}, \beta_{2}=\alpha_{2}-k \beta_{1}, \beta_{3}=\alpha_{3}-l_{1} \beta_{1}-l_{2} \beta_{2}$,
若 $\beta_{1}, \beta_{2},\beta_{3}$ 两两正交,则 $l_{1}, l_{2}$ 依次为 \\
$$\begin{array}{llll}\text { (A) } \frac{5}{2}, \frac{1}{2} . & \text { (B) }-\frac{5}{2}, \frac{1}{2} . & \text { (C) } \frac{5}{2},-\frac{1}{2} . & \text { (D) }-\frac{5}{2},-\frac{1}{2} \text { . }\end{array}$$
解: 利用斯密特正交化方法知
$$\begin{array}{c}
        \beta_{2}=\alpha_{2}-\frac{\left[\alpha_{2}, \beta_{1}\right]}{\left[\beta_{1}, \beta_{1}\right]} \beta_{1}=\left(\begin{array}{l}
                0 \\
                2 \\
            \end{array}\right),                                                \\
        \beta_{3}=\alpha_{3}-\frac{\left[\alpha_{3}, \beta_{1}\right]}{\left[\beta_{1}, \beta_{1}\right]} \beta_{1}-\frac{\left[\alpha_{3}, \beta_{2}\right]}{\left[\beta_{2}, \beta_{2}\right]} \beta_{2}, \\
        \text { 故 } l_{1}=\frac{\left[\alpha_{3}, \beta_{1}\right]}{\left[\beta_{1}, \beta_{1}\right]}=\frac{5}{2}, l_{2}=\frac{\left[\alpha_{3}, \beta_{2}\right]}{\left[\beta_{2}, \beta_{2}\right]}=\frac{1}{2}, \text { 故选 } \mathrm{A} .
    \end{array}$$ \\

\vspace{1ex}
{\heiti 9.} 已知行列式 $D=\left|\begin{array}{llll}-1 & 2 & -11 & 4  \\ 1 & 2 & \quad6 & 2  \\ 1 & 1 & \quad2 & 1  \\ 4 & 7 & \quad8 & 3\end{array}\right| .$ 求 $A_{13}+A_{23}+A_{33}+A_{43}$, 其中 $A_{ij}$ 是元素
$a_{ij}$ 的代数余子式。

解: 考虑行列式 $C=\left|\begin{array}{cccc}-1 & 2 & 1 & 4  \\ 1 & 2 & 1 & 2  \\ 1 & 1 & 1 & 1  \\ 4 & 7 & 1 & 3\end{array}\right|$, 按它的第三列展开。由于 $c$ 和 $D$ 除了第三
列外均相同,故 $C=A_{13}+A_{23}+A_{33}+A_{43}$, 而计算可得
$C=\left|\begin{array}{llll}-1 & 2 & 1 & 4  \\ 1 & 2 & 1 & 2  \\ 1 & 1 & 1 & 1  \\ 4 & 7 & 1 & 3\end{array}\right|=2 .$ 所以 $A_{13}+A_{23}+A_{33}+A_{t,}=2 .$ \\

{\heiti 10.}计算 $n$ 阶行列式 $\left|\begin{array}{llllll}a_{1} & b_{1} & 0 & \mathrm{~L} & 0 & 0  \\ 0 & a_{2} & b_{2} & \mathrm{~L} & 0 & 0  \\ \mathrm{M} & \mathrm{M} & \mathrm{M} & & \mathrm{M} & \mathrm{M}  \\ 0 & 0 & 0 & \mathrm{~L} & a_{n-1} & b_{n+1}  \\ b_{*} & 0 & 0 & \mathrm{~L} & 0 & a_{*}\end{array}\right|$. \\
解: 按第一列展开得:
$\left|\begin{array}{cccccc}a_{1} & b_{1} & 0 & \cdots & 0 & 0  \\ 0 & a_{2} & b_{2} & \cdots & 0 & 0  \\ \vdots & \vdots & \vdots & & \vdots & \vdots  \\ 0 & 0 & 0 & \cdots & a_{s-1} & b_{z-1}  \\ b_{n} & 0 & 0 & \cdots & 0 & a_{n}\end{array}\right|$
$=a_{1}\left|\begin{array}{ccccc}a_{2} & b_{2} & \cdots & 0 & 0  \\ \vdots & \vdots & & \vdots & \vdots  \\ 0 & 0 & \cdots & a_{n-1} & b_{n-1}  \\ 0 & 0 & \cdots & 0 & a_{n}\end{array}\right|+(-1)^{n+1} b\left|\begin{array}{ccccc}b_{1} & 0 & \cdots & 0 & 0  \\ a_{2} & b_{2} & \cdots & 0 & 0  \\ \vdots & \vdots & & \vdots & \vdots  \\ 0 & 0 & \cdots & a_{n-1} & b_{n-1}\end{array}\right|$
$=a_{1} a_{2} \cdots a_{n}+(-1)^{n+1} b_{1} b_{2} \cdots b_{n}$

\noindent (中山大学2015年)
\\ 已知实多项式$f(x)=x^4+2x^3-x^2-4x-2$,$g(x)=x^4+x^3-x^2-2x-2$.
\\ (1)求$f(x)$的全部有理根及相应重数.
\\ (2)求$f(x)$与$g(x)$的首一的最大公因式$(f,g)$.
\\ 解:(1)令$x=-1$,有$f(-1)=0$,说明$x+1\mid f(x)$.
\\ 可得$f(x)=(x+1)(x^3+x^2-2x-2)$,同样可以验证$x+1\mid x^3+x^2-2x-2$,
\\ 所以$f(x)=(x+1)^2(x^2-2)$,故$f(x)$有二重有理根$x=-1$.
\\ (2)因为
$$g(x)=x^4-2x^2+x^3-2x+x^2-1=x^2(x^2-2)+x(x^2-2)+(x^2-2)$$
$$=(x^2-2)(x^2+x+1)$$
因为$((x+1)^2,x^2+x+1)=1$,所以
$$(f(x),g(x))=x^2-2.$$
\\ \\ (中山大学2015年)
\\ 设3阶复矩阵
$A=
    \left(
    \begin{matrix}
            2  & 3   & 2  \\
            1  & 8   & 2  \\
            -2 & -14 & -3 \\
        \end{matrix}
    \right)$,
定义$C^3$上的线性变换$\sigma$为:$\sigma(\alpha)=A\alpha$,
对任意的$\alpha\in C^3$.求$\sigma$的最小多项式以及Jordan标准形.
\\ 解:取$C^3$的自然基$e_1,e_2,e_3$,
则有$\sigma(e_1)=Ae_1,\sigma(e_2)=Ae_2,\sigma(e_3)=Ae_3$,
那么$(\sigma(e_1),\sigma(e_2),\sigma(e_3))=(e_1,e_2,e_3)A$,
所以$\sigma$在基$e_1,e_2,e_3$下的矩阵是$A$.
所以只需求$A$的极小多项式和若当标准形.
\\ 先求$\lambda E-A$的初等因子,注意到$(\lambda E-A)$存在二阶行列式
$$
    \begin{vmatrix}
        -3        & -2 \\
        \lambda-8 & -2 \\
    \end{vmatrix}
    =2\lambda-10,
    \begin{vmatrix}
        -1 & \lambda-8 \\
        2  & 14        \\
    \end{vmatrix}
    =2-2\lambda
$$
并且$(\lambda-5,1-\lambda)=1$,从而$(\lambda E-A)$的二阶不变因子是1.又
$$\begin{vmatrix}\lambda E-A\end{vmatrix}=(\lambda-1)(\lambda-3)^2$$
所以$(\lambda E-A)$的标准形为
$$
    \left(
    \begin{matrix}
            1 & 0 & 0                        \\
            0 & 1 & 0                        \\
            0 & 0 & (\lambda-1)(\lambda-3)^2 \\
        \end{matrix}
    \right)
$$
从而$A$的初等因子为$\lambda-1,(\lambda-3)^2$.
\\ 故$A$的若当标准形为
$$
    \left(
    \begin{matrix}
            1     & \quad & \quad \\
            \quad & 3     & \quad \\
            \quad & 1     & 3     \\
        \end{matrix}
    \right)
$$
其最小多项式
$$m_A(\lambda)=\lambda-1,(\lambda-3)^2.$$
\\ \\ (中山大学2015年)
\\ 记$R[x]_5$为次数小于5的实多项式全体构成的向量空间,在$R[x]_5$上定义双线性函数如下
$$(f(x),g(x))=\int_{-1}^1f(x)g(x)dx$$
\\ (1)证明:上式定义了$R[x]_5$上一个正定的对称双线性函数.
\\ (2)用Gram-Schmidt方法由$1,x,x^2,x^3$求$R[x]_5$的一个正交向量组.
\\ (3)求一个形如$f(x)=a+bx^2-x^4$的多项式,使它与所有次数低于4的实多项式正交.
\\ 证明:(1)$\forall f(x),g(x)\in R[x]_5$,有
$$(f(x),g(x))=\int_{-1}^1f(x)g(x)dx=\int_{-1}^1g(x)f(x)dx=(g(x),f(x))$$
其二次型
$$q(f(x))=(f(x),f(x))=\int_{-1}^1f(x)^2 dx>0$$
对任意的$f(x)\in R[x]_5$且$f(x)\neq0$成立,
因而$(f(x),g(x))$是正定的对称双线性函数.
\\ (2)利用Gram-Schmidt方法和根据内积的运算,容易算出正交向量组,
$$1,x,x^2,-\frac{1}{3},x^3,-\frac{3}{5}x.$$
(3)取次数低于4的实多项式的一组基$1,x,x^2,x^3$,只需要$f(x)$与$1,x,x^2,x^3$都正交即可.
注意到$f(x)$是偶函数,所以只要与$1,x^2$正交即可.
$$(1,f(x))=\int_{-1}^1f(x)dx=2a+\frac{2}{3}b-\frac{2}{5}=0$$
$$(x^2,f(x))=\int_{-1}^1x^2f(x)dx=\frac{2}{3}a+\frac{2}{5}b-\frac{2}{7}=0$$
解得$a=-\frac{1}{5},b=\frac{6}{7}$.
\\ \\ (中山大学2015年)
\\ 设$A,B\in M_n(C)$为幂等矩阵,即$A^2=A,B^2=B$.
\\ (1)证明:$A-B$是幂等矩阵当且仅当$AB=BA=B$.
\\ (2)证明:若$AB=BA$,则$AB$为幂等矩阵.
反之,若$AB$为幂等矩阵,是否必有$AB=BA$?试证明或给出反例.
\\ 证明:充分性
\\ $A-B$是幂等矩阵,则$(A-B)^2=(A-B)$,有
$$(A-B)^2=A^2-AB-BA+B^2=A-B$$
依题意,$A^2=A,B^2=B$,有
$$2B=AB+BA$$
两边左乘以$A$有
$$2AB=A^2B+ABA=AB+ABA\Rightarrow AB=ABA$$
两边右乘以$A$有
$$2BA=ABA+BA^2=ABA+BA\Rightarrow BA=ABA$$
从而$B=AB=BA$.
\\ 必要性的证明是显然的.
\\ (2)若$AB=BA$,则
$$(AB)^2=ABAB=A(BA)B=A(AB)B=A^2B^2=AB$$
从而幂等.
\\ 取
$$
    A=\left(
    \begin{matrix}
            1 & 1 \\
            0 & 0 \\
        \end{matrix}
    \right),
    B=\left(
    \begin{matrix}
            1 & 0 \\
            0 & 0 \\
        \end{matrix}
    \right),
$$
那么
$$
    AB=\left(
    \begin{matrix}
            1 & 0 \\
            0 & 0 \\
        \end{matrix}
    \right),
    (AB)^2={\left(
    \begin{matrix}
            1 & 0 \\
            0 & 0 \\
        \end{matrix}
    \right)}^2
    =\left(
    \begin{matrix}
            1 & 0 \\
            0 & 0 \\
        \end{matrix}
    \right)
$$
从而$AB$是满足幂等的.但是
$$
    BA=\left(
    \begin{matrix}
            1 & 0 \\
            0 & 0 \\
        \end{matrix}
    \right)
    \left(
    \begin{matrix}
            1 & 1 \\
            0 & 0 \\
        \end{matrix}
    \right)
    =\left(
    \begin{matrix}
            1 & 1 \\
            0 & 0 \\
        \end{matrix}
    \right)
    \neq AB.
$$
\\ \\ (中山大学2015年)
\\ 设$A_1,\cdots,A_m$为$m$个两两可换的互不相同的$n$阶实对称矩阵,且
$$tr(A_i A_j)=\delta_{ij},1\leq i,j\leq n,$$
这里$tr(A)$表示矩阵$A$的迹,即它的对角元之和,试证明$m\leq n$.
\\ 证明:因为$A_1,A_2,\cdots,A_m$是$m$个两两可换的互不相同的$n$阶实对称矩阵,
所以存在一个$n$阶正交矩阵$P$使得${P^T}{A_i}P$都是对角阵.
\\ 记${P^T}{A_i}P=diag(\lambda_{i1},\lambda_{i2},\cdots,\lambda_{in})$,
其中$i=1,2,\cdots,m$.根据题意
$$
    tr(A_i A_j)=\delta_{ij}=
    \left\{
    \begin{array}{ccc}
        1,i=j     \\
        0,i\neq j \\
    \end{array}
    \right.
$$
由$tr$的性质,
$$tr(A_i A_j)=tr({A_i}{P^T}P{A_j}P{P^T})=tr({P^T}{A_i}P{P^T}{A_j}P)=\sum\limits_{k=1}^{n}\lambda_{ik}\lambda_{jk},$$
记$u_i=(\lambda_{i1},\lambda_{i2},\cdots,\lambda_{in})$,则有
$$
    (u_i,u_j)=
    \left\{
    \begin{array}{ccc}
        1,i=j     \\
        0,i\neq j \\
    \end{array}
    \right.
$$
因此$u_1,u_2,\cdots,u_m$是一组标准正交组,从而其个数小于$R^n$的维数,即有
$$m\leq n.$$

\end{document}