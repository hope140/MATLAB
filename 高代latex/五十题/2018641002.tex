\documentclass{article}
\usepackage{ctex}                               % 显示中文,更改字体
\usepackage{indentfirst}                        % 首行缩进
\usepackage{setspace}                           % 调整页行距
\usepackage{geometry}                           % 设置页边距
\usepackage{makecell}                           % 处理表格
\usepackage{amsmath}                            % 公式处理
\usepackage{amssymb}                            % 公式处理
\usepackage{amsthm}                             % 数学字符
\usepackage{mathrsfs}                           % 数学字符
\usepackage{enumerate}                          % 编号排版
\usepackage{algorithmic}

\renewcommand{\baselinestretch}{1.5}
\geometry{left=1.5cm,right=1.5cm,top=2cm,bottom=2cm}
%设置页边距,此处参考word默认间距

\begin{document}
\setlength{\parindent}{2em}                       % 首行缩进两字符
% \setcounter{page}{87}                           % 文章页码从87开始重新编排
{\heiti 1.}已知4阶行列式${D}$的第3行元素分别是-1,0,2,4,第4行元素对应的余子式依次是5,10,$a$,4,求$a$的值。
\begin{enumerate}[\qquad 解:]
    \item 因为$a_{31}A_{41}+a_{32}A_{42}+a_{33}A_{43}+a_{34}A_{44}=0$,
          这里 $a_{i j}$ 和 $A$ 分别是第 ${i}$ 行第 $j$ 列处的元素和该元素的  \\
          代数余子式,所以有 $-1 \times(-5)+0 \times 10+2 \times(-a)+4 \times 4=0$,
          可得 $a=\dfrac{21}{2}$
\end{enumerate}

\vspace{1ex}
{\heiti 2.} 已知矩阵 $A, B$ 满足关系 $A B-B=A$, 其中 $B=\left(
    \begin{array}{ccc}
            1 & -2 & 0 \\
            2 & 1  & 0 \\
            0 & 0  & 2
        \end{array}
    \right)$
, 求矩阵$A$。
\begin{enumerate}[\qquad 解:]
    \item 因为 ${A B-A=B}$, 所以 $A(B-E)=B$, $A=B(B-E)^{-1}=\left(
              \begin{array}{ccc}
                      1             & \dfrac{1}{2} & 0 \\
                      -\dfrac{1}{2} & 1            & 0 \\
                      0             & 0            & 2
                  \end{array}
              \right)$
\end{enumerate}

\vspace{1ex}
{\heiti 3.}设 $A^{*}$ 为 $3$ 阶方阵 $A$ 的伴随矩阵, $|A|=2$, 计算行列式 $|(3 A)^{-1}-\left.\dfrac{1}{2} A^{*}\right|$
\begin{enumerate}[\qquad 解:]
    \item $|(3 A)^{-1}-\left.\dfrac{1}{2} A^{*}\right|=|\dfrac{1}{3}A^{-1}-A^{-1}|=|-\dfrac{2}{3}A^{-1}|=(-\dfrac{2}{3})^{3}|A^{-1}|=-\dfrac{4}{27}$
\end{enumerate}

\vspace{1ex}
{\heiti 4.}如果多项式 $f(x), g(x)$ 不全为零,证明 $: \dfrac{f(x)}{(f(x), g(x))}$ 与 $\dfrac{g(x)}{(f(x), g(x))}$ 互素。

\begin{enumerate}[\qquad 证:]
    \item 证: 存在多项式 $u(x), v(x),$ 使 $(f(x), g(x))=u(x) f(x)+v(x) g(x)$ \\
          \vspace{1ex}
          因而 $u(x) \dfrac{f(x)}{(f(x), g(x))}+v(x) \dfrac{g(x)}{(f(x), g(x))}=1$ \\
          \vspace{1ex}
          由定理 $3, \left(\dfrac{f(x)}{(f(x),g(x))},\dfrac{g(x)}{(f(x), g(x))}\right)=1$
\end{enumerate}

\vspace{1ex}
{\heiti 5.}证明: $x_{0}$ 是 $f(x)$ 的 $k$ 重根的充分必要条件是 $f\left(x_{0}\right)=f^{\prime}\left(x_{0}\right)=\cdots=f^{k-1}\left(x_{0}\right)=0$ 而$f^{k}\left(x_{0}\right) \neq 0$

\begin{enumerate}[\qquad 证:]
    \item {\heiti 必要性:} 设 $x_{0}$ 是 $f(x)$ 的 $k$ 重根。 那么 $x_{0}$ 是 $f^{\prime}(x)$ 的 $k-1$ 重根, $\cdots \cdots$, 是 $f^{t-1}(x)$ 的1 重根, 是 $f^{\prime}(x)$ 的 0 重根, 即不是 $f^{*}(x)$ 的根 \\
          所以 $f\left(x_{0}\right)=f^{\prime}\left(x_{0}\right)=\cdots=f^{k-1}\left(x_{0}\right)=0$,而 $f^{k}\left(x_{0}\right) \neq 0 .$ \\
          {\heiti 充分性: } 设 $f(x_{0})=f^{\prime}\left(x_{0}\right)=\cdots=f^{t-1}\left(x_{0}\right)=0$ 而 $f^{\prime}\left(x_{0}\right) \neq 0 .$ 设 $x_{0}$ 是 $f(x)$ 的l重根 \\
          由必要性的证明 $f\left(x_{0}\right)=f^{\prime}\left(x_{0}\right)=\cdots=f^{\prime-1}\left(x_{0}\right)=0$ 而 $f^{\prime}\left(x_{0}\right) \neq 0 .$ 从而 $l=k .$
\end{enumerate}

\vspace{1ex}
{\heiti 6.}二次型 $f\left(x_{1}, x_{2}, x_{3}\right)=\left(x_{1}+x_{2}\right)^{2}+\left(x_{2}+x_{3}\right)^{2}-\left(x_{3}-x_{1}\right)^{2}$ 的正贯性指数与负惯性指数依次为
$(A) 2,0 $ $(B) 1,1 $ $(C) 2,1 $ $(D) 1,2 $

\begin{enumerate}[\qquad 解:]
    \item $f\left(x_{1}, x_{2}, x_{3}\right)=\left(x_{1}+x_{2}\right)^{2}+\left(x_{2}+x_{3}\right)^{2}-\left(x_{3}-x_{1}\right)^{2}=2 x_{2}^{2}+2 x_{1} x_{2}+2 x_{2} x_{3}+2 x_{1} x_{3}$  \\
          所以 $A=\left(\begin{array}{ccc}
                      0 & 1 & 1 \\
                      1 & 2 & 1 \\
                      1 & 1 & 0
                  \end{array}\right)$, 故特征多项式为
          \begin{equation*}
              |\lambda E-A|=\left|\begin{array}{rrr}
                  \lambda & -1 & -1      \\
                  -1      & -2 & -1      \\
                  -1      & -1 & \lambda
              \end{array}\right|=(\lambda+1)(\lambda-3) \lambda
          \end{equation*}
          令上式等于零,故特征值为$-1,3,0$,故该二次型的正惯性指数为1,负惯性指数为1,故选$B$.
\end{enumerate}

\vspace{1ex}
{\heiti 7.} 设 3 阶矩阵 $\boldsymbol{A}=\left(\alpha_{1}, \alpha_{2}, \alpha_{3}\right), \quad B=\left(\beta_{1}, \beta_{2}, \beta_{3}\right)$, 若向量组 $\alpha_{1}, \alpha_{2}, \alpha_{3}$ 可以由向量组 $\beta_{1}, \beta_{2}$线性表出,则
\begin{enumerate}[\qquad \quad (A)]
    \item $A x=0$ 的解均为 $B x=0$ 的解.
    \item $A^{T} x=0$ 的解均为 $B^{T} x=0$ 的解.
    \item $B x=0$ 的解均为 $A x=0$ 的解.
    \item $B^{T} x=0$ 的解均为 $A^{T} x=0$ 的解.
\end{enumerate}

\begin{enumerate}[\qquad 解:]
    \item 令 $A=\left(a_{1}, a_{2}, a_{3}\right), B=\left(\beta_{1}, \beta_{2}, \beta_{3}\right)$, 由题 $a_{1}, a_{2}, a_{3}$ 可由 $\beta_{1}, \beta_{2}, \beta_{3}$ 线性表示,即存在矩阵$A$,使得 $B P=A$, \\
          则当 $B^{T} x_{0}=0$ 时, $A^{T} x_{0}=(B P)^{T} x_{0}=p^{T} B^{T} x_{0}=0 .$ 恒成立,即选 $\mathrm{D}$.
\end{enumerate}

\vspace{1ex}
\begin{enumerate}[\qquad {\heiti 8.}]
    \item 已知 $\alpha_{1}=\left(\begin{array}{l}1  \\ 0  \\ 1\end{array}\right), \alpha_{2}=\left(\begin{array}{l}1  \\ 2  \\ 1\end{array}\right), \alpha_{3}=\left(\begin{array}{l}3  \\ 1  \\ 2\end{array}\right)$, 记 $\beta_{1}=\alpha_{1}, \beta_{2}=\alpha_{2}-k \beta_{1}, \beta_{3}=\alpha_{3}-l_{1} \beta_{1}-l_{2} \beta_{2}$,
          若 $\beta_{1}, \beta_{2},\beta_{3}$ 两两正交,则 $l_{1}, l_{2}$ 依次为
          \begin{equation*}
              \begin{array}{cccc}
                  \text { (A) } \dfrac{5}{2}, \dfrac{1}{2} . & \text { (B) }-\dfrac{5}{2}, \dfrac{1}{2} . & \text { (C) } \dfrac{5}{2},-\dfrac{1}{2} . & \text { (D) }-\dfrac{5}{2},-\dfrac{1}{2} \text { . }
              \end{array}
          \end{equation*}
\end{enumerate}
\begin{enumerate}[\qquad 解: ]
    \item 利用斯密特正交化方法知
          \begin{equation*}
              \begin{array}{c}
                  \beta_{2}=\alpha_{2}-\dfrac{\left|\alpha_{2}, \beta_{1}\right|}{\left|\beta_{1}, \beta_{1}\right|} \beta_{1}=\left(\begin{array}{l}
                          0 \\
                          2 \\
                      \end{array}\right) \\
                  \vspace{1ex}
                  \beta_{3}=\alpha_{3}-\dfrac{\left|\alpha_{3}, \beta_{1}\right|}{\left|\beta_{1}, \beta_{1}\right|} \beta_{1}-\dfrac{\left|\alpha_{3}
                  \beta_{2}\right|}{\left[\beta_{2}, \beta_{2}\right]} \beta_{2}                                                                                       \\
                  \text { 故 } l_{1}=\dfrac{\left|\alpha_{3}, \beta_{1}\right|}{\left|\beta_{1}, \beta_{1}\right|}=\dfrac{5}{2}, l_{2}=\dfrac{\left|\alpha_{3}, \beta_{2}\right|}{\left|\beta_{2}, \beta_{2}\right|}=\dfrac{1}{2}, \text { 故选 } \mathrm{A}
              \end{array}
          \end{equation*}
\end{enumerate}

\vspace{1ex}
{\heiti 9.} 已知行列式 $D=\left|\begin{array}{cccc}
        -1 & 2 & -11    & 4 \\
        1  & 2 & \quad6 & 2 \\
        1  & 1 & \quad2 & 1 \\ 4 & 7 & \quad8 & 3
    \end{array}\right| $ .求 $A_{13}+A_{23}+A_{33}+A_{43}$, 其中 $A_{ij}$ 是元素
$a_{ij}$ 的代数余子式.

\begin{enumerate}[\qquad 解: ]
    \item 考虑行列式 $C=\left|\begin{array}{cccc}
                  -1 & 2 & 1 & 4 \\
                  1  & 2 & 1 & 2 \\
                  1  & 1 & 1 & 1 \\
                  4  & 7 & 1 & 3
              \end{array}\right|$, 按它的第三列展开。由于 $c$ 和 $D$ 除了第三
          列外均相同,\\
          故 $C=A_{13}+A_{23}+A_{33}+A_{43}$, 而计算可得
          $C=\left|\begin{array}{cccc}
                  -1 & 2 & 1 & 4 \\
                  1  & 2 & 1 & 2 \\
                  1  & 1 & 1 & 1 \\
                  4  & 7 & 1 & 3
              \end{array}\right|=2 .$
          所以 $A_{13}+A_{23}+A_{33}+A_{t,}=2 .$ \\
\end{enumerate}

\vspace{1ex}
{\heiti 10.}计算 $n$ 阶行列式 $\left|\begin{array}{cccccc}
        a_{1}      & b_{1}      & 0          & \mathrm{~L} & 0          & 0          \\
        0          & a_{2}      & b_{2}      & \mathrm{~L} & 0          & 0          \\
        \mathrm{M} & \mathrm{M} & \mathrm{M} &             & \mathrm{M} & \mathrm{M} \\
        0          & 0          & 0          & \mathrm{~L} & a_{n-1}    & b_{n+1}    \\
        b_{*}      & 0          & 0          & \mathrm{~L} & 0          & a_{*}
    \end{array}\right|$
\begin{enumerate}[\qquad 解: ]
    \item 按第一列展开得:
          $\left|\begin{array}{cccccc}
                  a_{1}  & b_{1}  & 0      & \cdots & 0       & 0       \\
                  0      & a_{2}  & b_{2}  & \cdots & 0       & 0       \\
                  \vdots & \vdots & \vdots &        & \vdots  & \vdots  \\
                  0      & 0      & 0      & \cdots & a_{s-1} & b_{z-1} \\
                  b_{n}  & 0      & 0      & \cdots & 0       & a_{n}
              \end{array}\right|$
          $=a_{1}\left|\begin{array}{ccccc}
                  a_{2}  & b_{2}  & \cdots & 0       & 0       \\
                  \vdots & \vdots &        & \vdots  & \vdots  \\
                  0      & 0      & \cdots & a_{n-1} & b_{n-1} \\
                  0      & 0      & \cdots & 0       & a_{n}
              \end{array}\right| \\
              +(-1)^{n+1}b \left|\begin{array}{ccccc}
                  b_{1}  & 0      & \cdots & 0       & 0       \\
                  a_{2}  & b_{2}  & \cdots & 0       & 0       \\
                  \vdots & \vdots &        & \vdots  & \vdots  \\
                  0      & 0      & \cdots & a_{n-1} & b_{n-1}
              \end{array}\right|$
          $=a_{1} a_{2} \cdots a_{n}+(-1)^{n+1} b_{1} b_{2} \cdots b_{n}$
\end{enumerate}

\vspace{1ex}
\begin{enumerate}[\qquad {\heiti 11.}]
    \item 已知实多项式$f(x)=x^4+2x^3-x^2-4x-2$,$g(x)=x^4+x^3-x^2-2x-2$. \\
          (1)求$f(x)$的全部有理根及相应重数. \\
          (2)求$f(x)$与$g(x)$的首一的最大公因式$(f,g)$.
\end{enumerate}
\begin{enumerate}[\qquad 解:]
    \item
          (1)令$x=-1$,有$f(-1)=0$,说明$x+1\mid f(x)$.\\
          可得$f(x)=(x+1)(x^3+x^2-2x-2)$,同样可以验证$x+1\mid x^3+x^2-2x-2$,\\
          所以$f(x)=(x+1)^2(x^2-2)$,故$f(x)$有二重有理根$x=-1$.\\
          (2)因为$$g(x)=x^4-2x^2+x^3-2x+x^2-1=x^2(x^2-2)+x(x^2-2)+(x^2-2)=(x^2-2)(x^2+x+1)$$
          因为$((x+1)^2,x^2+x+1)=1$,所以 \\
          $$(f(x),g(x))=x^2-2.$$
\end{enumerate}

\vspace{1ex}
{\heiti 12.}设3阶复矩阵$A=
    \left(
    \begin{matrix}
            2  & 3   & 2  \\
            1  & 8   & 2  \\
            -2 & -14 & -3 \\
        \end{matrix}
    \right)$,
定义$C^3$上的线性变换$\sigma$为:$\sigma(\alpha)=A\alpha$,
对任意的$\alpha\in C^3$.求$\sigma$的最小多项式以及Jordan标准形.
\begin{enumerate}[\qquad 解:]
    \item    取$C^3$的自然基$e_1,e_2,e_3$,
          则有$\sigma(e_1)=Ae_1,\sigma(e_2)=Ae_2,\sigma(e_3)=Ae_3$,
          那么$(\sigma(e_1),\sigma(e_2),\sigma(e_3))=(e_1,e_2,e_3)A$,
          所以$\sigma$在基$e_1,e_2,e_3$下的矩阵是$A$.
          所以只需求$A$的极小多项式和若当标准形.
          \\ 先求$\lambda E-A$的初等因子,注意到$(\lambda E-A)$存在二阶行列式
          \begin{equation*}
              \begin{vmatrix}
                  -3        & -2 \\
                  \lambda-8 & -2 \\
              \end{vmatrix}
              =2\lambda-10,\quad
              \begin{vmatrix}
                  -1 & \lambda-8 \\
                  2  & 14        \\
              \end{vmatrix}
              =2-2\lambda
          \end{equation*}
          并且$(\lambda-5,1-\lambda)=1$,从而$(\lambda E-A)$的二阶不变因子是1.又
          \begin{equation*}
              \begin{vmatrix}\lambda E-A\end{vmatrix}=(\lambda-1)(\lambda-3)^2
          \end{equation*}
          所以$(\lambda E-A)$的标准形为
          \begin{equation*}
              \left(
              \begin{matrix}
                  1 & 0 & 0                        \\
                  0 & 1 & 0                        \\
                  0 & 0 & (\lambda-1)(\lambda-3)^2 \\
              \end{matrix}
              \right)
          \end{equation*}
          从而$A$的初等因子为$\lambda-1,(\lambda-3)^2$
          \\ 故$A$的若当标准形为
          \begin{equation*}
              \left(
              \begin{matrix}
                  1     & \quad & \quad \\
                  \quad & 3     & \quad \\
                  \quad & 1     & 3     \\
              \end{matrix}
              \right)
          \end{equation*}
          其最小多项式
          \begin{equation*}
              m_A(\lambda)=\lambda-1,(\lambda-3)^2
          \end{equation*}
\end{enumerate}

\vspace{1ex}
{\heiti 13.}记$R[x]_5$为次数小于5的实多项式全体构成的向量空间,在$R[x]_5$上定义双线性函数如下
\begin{equation*}
    (f(x),g(x))=\int_{-1}^1f(x)g(x)dx
\end{equation*}
\begin{enumerate}[\qquad (1)]
    \item 证明:上式定义了$R[x]_5$上一个正定的对称双线性函数.
    \item 用Gram-Schmidt方法由$1,x,x^2,x^3$求$R[x]_5$的一个正交向量组.
    \item 求一个形如$f(x)=a+bx^2-x^4$的多项式,使它与所有次数低于4的实多项式正交.
\end{enumerate}

\begin{enumerate}[\qquad 证明:]
    \item (1)$\forall f(x),g(x)\in R[x]_5$,有
          $$(f(x),g(x))=\int_{-1}^1f(x)g(x)dx=\int_{-1}^1g(x)f(x)dx=(g(x),f(x))$$
          其二次型
          $$q(f(x))=(f(x),f(x))=\int_{-1}^1f(x)^2 dx>0$$
          对任意的$f(x)\in R[x]_5$且$f(x)\neq0$成立,
          因而$(f(x),g(x))$是正定的对称双线性函数.\\
          (2)利用Gram-Schmidt方法和根据内积的运算,容易算出正交向量组,
          $$1,x,x^2,-\dfrac{1}{3},x^3,-\dfrac{3}{5}x.$$ \\
          (3)取次数低于4的实多项式的一组基$1,x,x^2,x^3$,只需要$f(x)$与$1,x,x^2,x^3$都正交即可.
          注意到$f(x)$是偶函数,所以只要与$1,x^2$正交即可.
          $$(1,f(x))=\int_{-1}^1f(x)dx=2a+\dfrac{2}{3}b-\dfrac{2}{5}=0$$
          $$(x^2,f(x))=\int_{-1}^1x^2f(x)dx=\dfrac{2}{3}a+\dfrac{2}{5}b-\dfrac{2}{7}=0$$
          解得$a=-\dfrac{1}{5},b=\dfrac{6}{7}$.
\end{enumerate}

\vspace{1ex}
{\heiti 14.}设$A,B\in M_n(C)$为幂等矩阵,即$A^2=A,B^2=B$.
\begin{enumerate}[\qquad (1)]
    \item 证明:$A-B$是幂等矩阵当且仅当$AB=BA=B$.
    \item 证明:若$AB=BA$,则$AB$为幂等矩阵.
\end{enumerate}
\qquad 反之,若$AB$为幂等矩阵,是否必有$AB=BA$?试证明或给出反例.

\begin{enumerate}[\qquad 证明:]
    \item (1)充分性
          \\ $A-B$是幂等矩阵,则$(A-B)^2=(A-B)$,有
          $$(A-B)^2=A^2-AB-BA+B^2=A-B$$
          依题意,$A^2=A,B^2=B$,有
          $$2B=AB+BA$$
          两边左乘以$A$有
          $$2AB=A^2B+ABA=AB+ABA\Rightarrow AB=ABA$$
          两边右乘以$A$有
          $$2BA=ABA+BA^2=ABA+BA\Rightarrow BA=ABA$$
          从而$B=AB=BA$.
          \\ 必要性的证明是显然的.
          \\ (2)若$AB=BA$,则
          $$(AB)^2=ABAB=A(BA)B=A(AB)B=A^2B^2=AB$$
          从而幂等.
          \\ 取
          $$
              A=\left(
              \begin{matrix}
                      1 & 1 \\
                      0 & 0 \\
                  \end{matrix}
              \right),
              B=\left(
              \begin{matrix}
                      1 & 0 \\
                      0 & 0 \\
                  \end{matrix}
              \right),
          $$
          那么
          $$
              AB=\left(
              \begin{matrix}
                      1 & 0 \\
                      0 & 0 \\
                  \end{matrix}
              \right),
              (AB)^2={\left(
              \begin{matrix}
                      1 & 0 \\
                      0 & 0 \\
                  \end{matrix}
              \right)}^2
              =\left(
              \begin{matrix}
                      1 & 0 \\
                      0 & 0 \\
                  \end{matrix}
              \right)
          $$
          从而$AB$是满足幂等的.但是
          $$
              BA=\left(
              \begin{matrix}
                      1 & 0 \\
                      0 & 0 \\
                  \end{matrix}
              \right)
              \left(
              \begin{matrix}
                      1 & 1 \\
                      0 & 0 \\
                  \end{matrix}
              \right)
              =\left(
              \begin{matrix}
                      1 & 1 \\
                      0 & 0 \\
                  \end{matrix}
              \right)
              \neq AB.
          $$
\end{enumerate}

\vspace{1ex}
{\heiti 15.}设$A_1,\cdots,A_m$为$m$个两两可换的互不相同的$n$阶实对称矩阵,且
$$tr(A_i A_j)=\delta_{ij},1\leq i,j\leq n,$$
\qquad 这里$tr(A)$表示矩阵$A$的迹,即它的对角元之和,试证明$m\leq n$.
\begin{enumerate}[\qquad 证明:]
    \item 因为$A_1,A_2,\cdots,A_m$是$m$个两两可换的互不相同的$n$阶实对称矩阵,
          所以存在一个$n$阶正交矩阵$P$使得${P^T}{A_i}P$都是对角阵.\\
          记${P^T}{A_i}P=diag(\lambda_{i1},\lambda_{i2},\cdots,\lambda_{in})$,
          其中$i=1,2,\cdots,m$.根据题意
          \begin{equation*}
              tr(A_i A_j)=\delta_{ij}=
              \left\{
              \begin{array}{ccc}
                  1,i=j     \\
                  0,i\neq j \\
              \end{array}
              \right.
          \end{equation*}
          由$tr$的性质,
          $$tr(A_i A_j)=tr({A_i}{P^T}P{A_j}P{P^T})=tr({P^T}{A_i}P{P^T}{A_j}P)=\sum\limits_{k=1}^{n}\lambda_{ik}\lambda_{jk},$$
          记$u_i=(\lambda_{i1},\lambda_{i2},\cdots,\lambda_{in})$,则有
          \begin{equation*}
              (u_i,u_j)=
              \left\{
              \begin{array}{ccc}
                  1,i=j     \\
                  0,i\neq j \\
              \end{array}
              \right.
          \end{equation*}
          因此$u_1,u_2,\cdots,u_m$是一组标准正交组,从而其个数小于$R^n$的维数,即有
          $$m\leq n$$
\end{enumerate}

\vspace{1ex}
{\heiti 16.}求下列 $n$ 阶实矩阵的行列式:

(1) $A=\left(a_{i j}\right)$, 其中 $a_{i j}=\left\{\begin{array}{lr}
        1, & i \neq j, \text { 且 } i \text { 或 } j=1 \\
        2, & i=j                                       \\
        0, & \text { 其他. }
    \end{array}\right.$

(2) $B=\left(b_{i j}\right)$, 其中 $b_{i j}=f_{j}\left(a_{i}\right), f_{j}(x)$ 为首一的 $j-1$ 次实系数多项式,$a_{1}, \cdots, a_{n}$ 为两两不同的实数.
\begin{enumerate}[\qquad 解:]
    \item (1)根据题意写出矩阵A.
          $$
              A=\left[\begin{array}{ccccc}
                      2      & 1      & 1      & \ldots & 1      \\
                      1      & 2      & 0      & \ldots & 0      \\
                      \vdots & \vdots & \vdots & \ddots & \vdots \\
                      1      & 0      & 0      & \ldots & 2
                  \end{array}\right]
          $$
          则
          $$
              |A|=\left|\begin{array}{ccccc}
                  2      & 1      & 1      & \ldots & 1      \\
                  1      & 2      & 0      & \ldots & 0      \\
                  \vdots & \vdots & \vdots & \ddots & \vdots \\
                  1      & 0      & 0      & \ldots & 2
              \end{array}\right|
          $$
          从第二行开始,每一行乘上$\left(-\dfrac{1}{2}\right)$加到第一行,
          $$
              |A|=\left|\begin{array}{ccccc}
                  2-\dfrac{n-1}{2} & 0      & 0      & \ldots & 0      \\
                  1                & 2      & 0      & \ldots & 0      \\
                  \vdots           & \vdots & \vdots & \ddots & \vdots \\
                  1                & 0      & 0      & \ldots & 2
              \end{array}\right|=\dfrac{5-n}{2} \cdot 2^{n-1}
          $$
          (2)根据题意,设出$f_j \left( x \right)$的表达式,
          $$
              \begin{array}{c}
                  f_{1}(x)=1                     \\
                  f_{2}(x)=x+c_{21}              \\
                  f_{3}(x)=x^{2}+c_{31} x+c_{32} \\
                  \vdots                         \\
                  f_{n}(x)=x^{n-1}+c_{n 1} x^{n-2}+\cdots+c_{n, n-1}
              \end{array}
          $$
          写出矩阵 $B$.
          $$
              B=\left[\begin{array}{cccc}
                      f_{1}\left(a_{1}\right) & f_{2}\left(a_{1}\right) & \ldots & f_{n}\left(a_{1}\right) \\
                      f_{1}\left(a_{2}\right) & f_{2}\left(a_{2}\right) & \ldots & f_{n}\left(a_{2}\right) \\
                      \vdots                  & \vdots                  & \ddots & \vdots                  \\
                      f_{1}\left(a_{n}\right) & f_{2}\left(a_{n}\right) & \ldots & f_{n}\left(a_{n}\right)
                  \end{array}\right]
          $$
          所以
          $$
              |B|=\left|\begin{array}{lllc}
                  1      & a_{1}+c_{21} & \ldots & a_{1}^{n-1}+c_{n 1} a_{1}^{n-2}+\cdots+c_{n, n-1} \\
                  1      & a_{2}+c_{21} & \ldots & a_{2}^{n-1}+c_{n 1} a_{2}^{n-2}+\cdots+c_{n, n-1} \\
                  \vdots & \vdots       & \ddots & \vdots                                            \\
                  1      & a_{n}+c_{21} & \ldots & a_{n}^{n-1}+c_{n 1} a_{n}^{n-2}+\cdots+c_{n, n-1}
              \end{array}\right|
          $$
          根据行列式的性质,按逐列展开,最后利用范德蒙德行列式,
          $$
              |B|=\left|\begin{array}{cccc}
                  1      & a_{1}  & \ldots & a_{1}^{n-1} \\
                  1      & a_{2}  & \ldots & a_{2}^{n-1} \\
                  \vdots & \vdots & \ddots & \vdots      \\
                  1      & a_{n}  & \ldots & a_{n}^{n-1}
              \end{array}\right|=\prod_{1 \leq i<j \leq n}\left(a_{j}-a_{i}\right) .
          $$
\end{enumerate}

\vspace{1ex}
{\heiti 17.}设 $n$ 阶实矩阵 $\boldsymbol{A}=\left(a_{i j}\right)$ 半正定.
\begin{enumerate}[\qquad (1)]
    \item 证明: 存在 $\alpha_{i} \in R^{n}, i=1, \cdots, n$, 使得 $a_{i j}$ 等于 $\alpha_{i}$ 与 $\alpha_{j}$ 的内积;
    \item 证明: $2 n$ 阶矩阵 $\left(\begin{array}{cc}A & A \\ A & A\end{array}\right)$ 半正定
    \item 若实矩阵 $B=\left(b_{i j}\right)$ 也半正定, 令 $d_{i j}=a_{i j} b_{i j} .$ 证明: 矩阵 $D=\left(d_{i j}\right)$ 半正定.
\end{enumerate}

\begin{enumerate}[\qquad 证明:]
    \item (1) 证明因为 $A$ 是半正定矩阵, 故存在实矩阵 $C$ 使 $A=C^{T} C .$ 取 $R^{n}$ 的一 组标准正交基 $\varepsilon_{1}, \varepsilon_{2}, \ldots, \varepsilon_{n}$, 记 $\left(\alpha_{1}, \ldots, \alpha_{n}\right)=\left(\varepsilon_{1}, \varepsilon_{2}, \ldots, \varepsilon_{n}\right) C$, 那么
          $$
              \begin{array}{c}
                  G\left(\alpha_{1}, \ldots, \alpha_{n}\right)=\left[\begin{array}{c}
                          \alpha_{1}^{T} \\
                          \alpha_{2}^{T} \\
                          \vdots         \\
                          \alpha_{n}^{T}
                      \end{array}\right]\left[\alpha_{1}, \alpha_{2}, \ldots, \alpha_{n}\right]= \\
                  C^{T}\left[\begin{array}{c}
                          \varepsilon_{1}^{T} \\
                          \varepsilon_{2}^{T} \\
                          \vdots              \\
                          \varepsilon_{n}^{T}
                      \end{array}\right]\left[\varepsilon_{1}, \varepsilon_{2}, \ldots, \varepsilon_{n}\right]=C^{T} C=A
              \end{array}
          $$
          所以有 $a_{i j}=\alpha_{i}^{T} \alpha_{j} $. \\
          (2) 因为 $A$ 是半正定的, 则 $A^{T}=A .$ 故
          $$
              \left[\begin{array}{ll}
                      A & A \\
                      A & A
                  \end{array}\right]^{T}=\left[\begin{array}{ll}
                      A^{T} & A^{T} \\
                      A^{T} & A^{T}
                  \end{array}\right]=\left[\begin{array}{ll}
                      A & A \\
                      A & A
                  \end{array}\right]
          $$
          又因为
          $$
              \left[\begin{array}{cc}
                      E  & 0 \\
                      -E & E
                  \end{array}\right]\left[\begin{array}{ll}
                      A & A \\
                      A & A
                  \end{array}\right]\left[\begin{array}{cc}
                      E & -E \\
                      0 & E
                  \end{array}\right]=\left[\begin{array}{cc}
                      A & 0 \\
                      0 & 0
                  \end{array}\right]
          $$
          所以 $\left[\begin{array}{ll}A & A \\ A & A\end{array}\right]$ 合同于 $\left[\begin{array}{cc}A & 0 \\ 0 & 0\end{array}\right]$, 并且 $A$ 是半正定的, 从而 $\left[\begin{array}{cc}A & 0 \\ 0 & 0\end{array}\right]$ 半正定, 所以
          $\left[\begin{array}{ll}\boldsymbol{A} & \boldsymbol{A} \\ \boldsymbol{A} & \boldsymbol{A}\end{array}\right]$ 半正定. \\
          (3) 因为 $A, B$ 半正定,所以对称. 因此 $D$ 也对称. 根据定义,对任意的 $x=$
          $\left(x_{1}, x_{2}, \ldots, x_{n}\right)^{T}$, 有
          $$
              x^{T} A x=\sum_{j, k=1}^{n} a_{j k} x_{j} x_{k} \geq 0, x^{T} B x=\sum_{j, k=1}^{n} b_{j k} x_{j} x_{k} \geq 0
          $$
          因为 $B$ 半正定, 所以存在 $T=r_{i j}$, 使得 $B=T^{T} T .$ 所以 $b_{j k}=\sum_{l=1}^{n} r_{l j} r_{l k}$, 所以
          \begin{equation*}
              x^{T} D x=\sum_{j, k=1}^{n} a_{j k} b_{j k} x_{j} x_{k}=\sum_{j, k=1}^{n} a_{j k}\left(\sum_{l=1}^{n} r_{l j} r_{l k}\right) x_{j} x_{k}=
              \sum_{l=1}^{n} \sum_{j, k=1}^{n} a_{j k}\left(x_{j} r_{l j}\right)\left(x_{k} r_{l k}\right)
          \end{equation*}
          记 $y_{l}=\left(x_{1} r_{l 1}, \ldots, x_{n} r_{l n}\right)^{T}$, 则
          $$
              \sum_{j, k=1}^{n} a_{j k}\left(x_{j} r_{l j}\right)\left(x_{k} r_{l k}\right)=y_{l}^{T} A y_{l} \leq 0
          $$
          从而 $D$ 是半正定矩阵.
\end{enumerate}
\vspace{1ex}
{\heiti 18.}设 $A \in M_{n}(C)$. 定义 $M_{n}(C)$ 上的线性变换 $\sigma, \tau$ 为 $\sigma(X)=A X, \tau(X)=$ $\boldsymbol{A} \boldsymbol{X}-\boldsymbol{X} \boldsymbol{A}$, 对任意的 $\boldsymbol{X} \in \boldsymbol{M}_{n}(\boldsymbol{C})$

(1) 设 $\mathrm{A}$ 的秩为 $r$, 求 $\operatorname{dim}$ ker $\sigma$

(2) 若 $B \in M_{n}(C)$, 满足 $\tau(B)=B .$ 证明 :$B$ 的特征值都是零,且矩阵 $A$ 与
$B$ 至少有一个公共的特征向量.
\begin{enumerate}[\qquad 证明:]
    \item (1) 解 \\
          $\operatorname{ker}(\sigma)=\left\{X \mid \sigma(X)=0, x \in M_{n}(C)\right\} .$ 记 $V=\left\{x \mid A x=0, x \in C^{n}\right\} .$ 则对
          任意的 $X \in \operatorname{ker}(\sigma)$, 记 $X=\left(\alpha_{1}, \ldots, \alpha_{n}\right)$, 由于 $A X=0 \Rightarrow A \alpha_{i}=0$, 从而
          $\alpha_{i} \in V$. 因为 $r(A)=r$, 所以 $\operatorname{dim}(\boldsymbol{V})=n-r .$ 假设 $V$ 的一组基是 $\eta_{1}, \ldots, \eta_{n-r}$,
          那么所有的满足 $\boldsymbol{X} \in \operatorname{ker}(\sigma)$ 的 $\boldsymbol{X}$, 它的每个列向量必然是 $\eta_{1}, \ldots, \eta_{n-r}$ 的
          线性组合. 令 $E_{11}=\left(\eta_{1}, 0, \ldots, 0\right), E_{12}=\left(0, \eta_{1}, 0, \ldots, 0\right), \ldots, E_{n-r, n}=$
          $\left(0,0, \ldots, \eta_{n-r}\right)$, 容易证明它是 $\operatorname{ker}(\sigma)$ 的一组基, 从而 $\operatorname{dim}(\operatorname{ker}(\sigma))=n(n-r) .$
          (2) 记 $B$ 的不为 0 的互不相同的特征值是 $\lambda_{1}, \lambda_{2}, \ldots, \lambda_{s}$, 相应的重数是 $k_{1}, \ldots, k_{s} $ . \\
          则
          $$
              \operatorname{tr} B=\operatorname{tr}(A B-B A)=k_{1} \lambda_{1}+\cdots+k_{s} \lambda_{s}=0
          $$
          因为 $B^{2}=B(A B-B A)=B A B-B^{2} A=(B A) B-B(B A)$, 所以
          $$
              \operatorname{tr}\left(B^{2}\right)=\operatorname{tr}\left(B A B-B^{2} A\right)=k_{1} \lambda_{1}^{2}+\cdots+k_{s} \lambda_{s}^{2}=0
          $$
          类似的,
          $$
              \operatorname{tr}\left(B^{s}\right)=k_{1} \lambda_{1}^{s}+\cdots+k_{s} \lambda_{s}^{s}=0
          $$
          所以
          $$
              \left[\begin{array}{cccc}
                      \lambda_{1}     & \lambda_{2}     & \ldots & \lambda_{s}     \\
                      \vdots          & \vdots          & \ddots & \vdots          \\
                      \lambda_{1}^{s} & \lambda_{2}^{s} & \ldots & \lambda_{s}^{s}
                  \end{array}\right]\left[\begin{array}{c}
                      k_{1}  \\
                      k_{2}  \\
                      \vdots \\
                      k_{n}
                  \end{array}\right]=\left[\begin{array}{c}
                      0      \\
                      0      \\
                      \vdots \\
                  \end{array}\right]
          $$
          有非零解. 但是 $\left|\begin{array}{cccc}\lambda_{1} & \lambda_{2} & \ldots & \lambda_{s} \\ \vdots & \vdots & \ddots & \vdots \\ \lambda_{1}^{s} & \lambda_{2}^{s} & \ldots & \lambda_{s}^{s}\end{array}\right|=\lambda_{1} \lambda_{2} \ldots \lambda_{s}\left|\begin{array}{cccc}1 & 1 & \ldots & 1 \\ \lambda_{1} & \lambda_{2} & \ldots & \lambda_{s} \\ \vdots & \vdots & \ddots & \vdots \\ \lambda_{1}^{s-1} & \lambda_{2}^{s-1} & \ldots & \lambda_{s}^{s-1}\end{array}\right|$ \\
          因为 $\lambda_{i}$ 是互不相同的, 所以上面的行列式不等于 0, 从而线性方程组只有 0 解, 矛 盾了. 所以 $B$ 不存在不等于 0 的特征值. 从而 $B$ 的特征值全部是 $0 $. \\
          (2) 证明 \\
          设 $A \alpha=\lambda \alpha$, 其中 $\alpha \neq 0$ 是 $A$ 的特征向量, 因为 $A B=B+B A$, 所以 $A B \alpha=B \alpha+B A \alpha=(\lambda+1) B \alpha$, 如果 $B \alpha=0$, 那么 $\alpha$ 是 $A, B$ 的公共特征
          向量, 如果 $B \alpha \neq 0$, 那么 $\lambda+1$ 是 $A$ 的一个特征值, 而 $B \alpha$ 是 $A$ 的一个特征向
          量; 进一步地
          $$
              A B^{2} \alpha=+B A B \alpha=(\lambda+2) B^{2} \alpha
          $$
          如果 $B^{2} \alpha=0$ 那么 $B \alpha$ 是 $A, B$ 的公共特征向量, 如果 $B \alpha \neq 0$, 那么 $\lambda+2$ 是 $A$ 的一个特征值, 而 $B^{2} \alpha$ 是 $A$ 的一个特征向量; \\
          如此下去, 必然存在 $B^{k} \alpha=0$ 而 $B^{k-1} \alpha \neq 0$ 是 $A, B$ 的公共特征向量. 这是因 为,若 $B^{k} \alpha \neq 0$, 那么 $\lambda+k$ 都是 $A$ 的特征值, 这是不可能的.
\end{enumerate}

\vspace{1ex}
{\heiti 19.} 计算行列式
$$
    \left|\begin{array}{cccc}
        1 & 1 & 1 & 1  \\
        2 & 1 & 1 & -3 \\
        1 & 2 & 2 & 5  \\
        4 & 3 & 2 & 1
    \end{array}\right|
$$

解答. 直接计算可得
$$
    \left|\begin{array}{cccc}
        1 & 1 & 1 & 1  \\
        2 & 1 & 1 & -3 \\
        1 & 2 & 2 & 5  \\
        4 & 3 & 2 & 1
    \end{array}\right|=\left|\begin{array}{cccc}
        1 & 1  & 1  & 1  \\
        0 & -1 & -1 & -5 \\
        0 & 1  & 1  & 4  \\
        0 & -1 & -2 & -3
    \end{array}\right|=\left|\begin{array}{cccc}
        1 & 1  & 1  & 1  \\
        0 & -1 & -1 & -5 \\
        0 & 0  & 0  & -1 \\
        0 & 0  & -1 & 2
    \end{array}\right|=-\left|\begin{array}{cccc}
        1 & 1  & 1  & 1  \\
        0 & -1 & -1 & -5 \\
        0 & 0  & -1 & 2  \\
        0 & 0  & 0  & -1
    \end{array}\right|=1
$$

\vspace{1ex}
{\heiti 20.}计算行列式
$$
    \left|\begin{array}{cccc}
        x_{1}-m & x_{2}   & \cdots & x_{n}   \\
        x_{1}   & x_{2}-m & \cdots & x_{n}   \\
        \vdots  & \vdots  &        & \vdots  \\
        x_{1}   & x_{2}   & \cdots & x_{n}-m
    \end{array}\right|_{n \times n}
$$

解答. 由于
现根据行列式的性质将上述行列式拆分为 $2^{n}$ 个行列式之和, 其中每个行列式的第 $i(i=1,2, \cdots, n)$

行要么为 $\left(x_{i}, x_{i}, \cdots, x_{i}\right)$, 要么为 $(0, \cdots, 0,-m, 0, \cdots, 0)$, 将这 $2^{n}$ 个行列式分为如下三类:
\begin{enumerate}[\qquad (i) ]
    \item 至少有两行 $\left(\right.$ 设为 $i, j(i \neq j)$ 行)的元素分别取 $\left(x_{i}, x_{i}, \cdots, x_{i}\right),\left(x_{j}, x_{j}, \cdots, x_{j}\right)$, 由于每个行列式至么 有两行元素成比例, 所以此类行列式均为零, 其和自然也为零.
    \item 有且仅有一行 $\left(\right.$ 设为第 $i(i=1,2, \cdots, n)$ 行)元素取的是 $\left(x_{i}, x_{i}, \cdots, x_{i}\right)$, 此类行列式共有 $n$ 个, 并且 根据第 $i$ 列展开, 可知
          $$
              \sum_{i = 1}^{n}\left|\begin{array}{llllllll}
                  -m    &        &        &       &        &        &       \\
                        & \ddots &        &       &        &        &       \\
                        &        & -m     &       &        &        &       \\
                  x_{i} & x_{i}  & \cdots & x_{i} & \cdots & x_{i}  & x_{i} \\
                        &        &        &       & -m     &        &       \\
                        &        &        &       &        & \ddots &       \\
                        &        &        &       &        &        & -m
              \end{array}\right|=\sum_{i=1}^{n}(-1)^{i+i} x_{i}
              \left|\begin{array}{cccc}
                  -m &    &        &    \\
                     & -m &        &    \\
                     &    & \ddots &    \\
                     &    &        & -m
              \end{array} \right| =(-m)^{n-1} \sum_{i=1}^{n} x_{i} .
          $$
          (iii) 每一行均取形如 $(0, \cdots, 0,-m, 0, \cdots, 0)$ 这样的元素, 此类行列式仅有一个, 即
          $$
              \left|\begin{array}{llll}
                  -m &    &        &    \\
                     & -m &        &    \\
                     &    & \ddots &    \\
                     &    &        & -m
              \end{array}\right|=(-m)^{n}
          $$
          于是
          $$
              \left|\begin{array}{ccccc}
                  x_{1}-m & x_{2}   & \cdots & x_{n}   \\
                  x_{1}   & x_{2}-m & \cdots & x_{n}   \\
                  \vdots  & \vdots  &        & \vdots  \\
                  x_{1}   & x_{2}   & \cdots & x_{n}-m
              \end{array}\right|_{n \times n}=(-m)^{n}+(-m)^{n-1} \sum_{i=1}^{n} x_{i}
          $$
\end{enumerate}

\vspace{1ex}
{\heiti 21.}证明高斯引理: 两个本原多项式的乘积还是本原多项式.

解答. 设 $f(x)=a_{n} x^{n}+\cdots+a_{1} x+a_{0}$ 与 $g(x)=b_{m} x^{m}+\cdots+b_{1} x+b_{0}$ 为两个本原多项式, 其中 $a_{n} b_{m} \neq 0$,

记
$$
    h(x)=f(x) g(x)=c_{m+n} x^{m+n}+\cdots+c_{1} x+c_{0}
$$

若 $h(x)$ 不是本原多项式, 则存在素数 $p$ 使得 $p \mid c_{i}(i=0,1, \cdots, m+n)$, 而由于 $f(x), g(x)$ 均为本原多项式.

所以 $p$ 不可能整除所有的 $a_{i}(i=0,1, \cdots, n)$, 也不可能整除所有的 $b_{i}(i=0,1, \cdots, m)$, 不妨设
$$
    p\left|a_{0}, a_{1}, \cdots, a_{i-1}, p \nmid a_{i} ; p\right| b_{0}, b_{1}, \cdots, b_{j-1}, p \nmid b_{j}
$$

此时考虑 $c_{i+j}$, 由于
$$
    \begin{aligned}
        c_{i+j}=a_{i} b_{j} & +a_{i+1} b_{j-1}+a_{i+2} b_{j-2}+\cdots \\
                            & +a_{i-1} b_{j+1}+a_{i-2} b_{j+2}+\cdots
    \end{aligned}
$$

由上述假设可知 $p \nmid a_{i} b_{j}$, 但 $p\left|a_{i+1} b_{j-1}+a_{i+2} b_{j-2}+\cdots, p\right| a_{i-1} b_{j+1}+a_{i-2} b_{j+2}+\cdots$, 于是 $p \nmid c_{i+j}$,

这与原假设矛盾. 所以 $h(x)$ 一定是本原多项式

\vspace{1ex}
{\heiti 22.}设多项式 $f(x)=x^{3} \mathcal{F}(1+t) x^{2}+4 x+k, g(x)=x^{3}+t x^{2}+k$, 常数 $t$ 和 $k$ 为多少时, 最大公 因式 $(f(x), g(x))$ 是二次多项式?

解答. 注意到 $f(x)$ 与 $g(x)$ 作带余除法可得
$$
    f(x)=g(x)+x^{2}+4 x
$$

所以由辗转相除法可知
$$
    (f(x), g(x))=(g(x), f(x)-g(x))=\left(g(x), x^{2}+4 x\right) .
$$

于是若 $(f(x), g(x))$ 是二次多项式, 则必为 $x^{2}+4 x=x(x+4)$, 即有 $x(x+4) \mid g(x)$, 从而
$$
    \left\{\begin{array}{l}
        0=g(0)=k ; \\
        0=g(-4)=(-4)^{3}+t(-4)^{2}+k
    \end{array}\right.
$$

由此解得 $t=4, k=0$

\vspace{1ex}
{\heiti 23.}当常数 $a, b, c$ 满足什么条件时, 下列线性方程组有解? 并在有解的条件下求出全部解

(用特解和相应齐次线性方程组的基础解系表示).
$$
    \left\{\begin{array}{l}
        x_{1}+2 x_{2}+x_{3}-x_{4}+x_{5}-2 x_{6}+3 x_{7}=1    \\
        2 x_{1}+4 x_{2}+3 x_{3}+5 x_{5}-3 x_{6}+7 x_{7}=a ;  \\
        -3 x_{1}-6 x_{2}-2 x_{3}+5 x_{4}+8 x_{6}-7 x_{7}=b ; \\
        -x_{1}-2 x_{2}+x_{3}+5 x_{4}+5 x_{5}+5 x_{6}=c
    \end{array}\right.
$$

解答. 首先对方程组的增广矩阵进行初等行变换, 化为阶梯形:
$$
    \begin{array}{l}
        \left(\begin{array}{cccccccc}
                1  & 2  & 1  & -1 & 1 & -2 & 3  & 1 \\
                2  & 4  & 3  & 0  & 5 & -3 & 7  & a \\
                -3 & -6 & -2 & 5  & 0 & 8  & -7 & b \\
                -1 & -2 & 1  & 5  & 5 & 5  & 0  & c
            \end{array}\right) \rightarrow\left(\begin{array}{cccccccc}
                1 & 2 & 1 & -1 & 1 & -2 & 3 & 1   \\
                0 & 0 & 1 & 2  & 3 & 1  & 1 & a-2 \\
                0 & 0 & 1 & 2  & 3 & 2  & 2 & b+3 \\
                0 & 0 & 2 & 4  & 6 & 3  & 3 & c+1
            \end{array}\right) \\
        \rightarrow\left(\begin{array}{cccccccc}
                1 & 2 & 1 & -1 & 1 & -2 & 3 & 1       \\
                0 & 0 & 1 & 2  & 3 & 1  & 1 & a-2     \\
                0 & 0 & 0 & 0  & 0 & 1  & 1 & b-a+5   \\
                0 & 0 & 0 & 0  & 0 & 1  & 1 & c-2 a+5
            \end{array}\right) \rightarrow\left(\begin{array}{cccccccc}
                1 & 2 & 1 & -1 & 1 & -2 & 3 & 1     \\
                0 & 0 & 1 & 2  & 3 & 1  & 1 & a-2   \\
                0 & 0 & 0 & 0  & 0 & 1  & 1 & b-a+5 \\
                0 & 0 & 0 & 0  & 0 & 0  & 0 & c-b-a
            \end{array}\right) .
    \end{array}
$$

由此可知 $c-b-a=0$ 时, 方程组有解. 并且由上述阶梯形可知
$$
    \eta_{1}=(-2,1,0,0,0,0,0)^{\prime}, \eta_{2}=(3,0,-2,1,0,0,0)^{\prime}, \eta_{3}=(2,0,-3,0,1,0,0)^{\prime}, \eta_{4}=(-5,0,0,0,0,-1,1)^{\prime}
$$

为方程组导出组的基础解系, 同时
$$
    X_{0}=(3 b-4 a+18,0,2 a-b-7,0,0, b-a+5,0)^{\prime}
$$

为方程组的一个特解, 所以方程组的通解为
$$
    X_{0}+k_{1} \eta_{1}+k_{2} \eta_{2}+k_{3} \eta_{3}+k_{4} \eta_{4} .
$$

其中 $k_{1}, k_{2}, k_{3}, k_{4}$ 为任意数.

\vspace{1ex}
{\heiti 24.}设矩阵 $A, B$ 分别是数域 $P$ 上的 $m \times n$ 和 $s \times n$ 矩阵, 证明: 线性方程组 $A X=0$ 与 $B X=0$ 同解的充分必要条件是存在矩阵 $T_{1}, T_{2}$, 使得 $A=T_{1} B, B=T_{2} A$.

解答. 充分性. 已知存在矩阵 $T_{1}, T_{2}$, 使得 $A=T_{1} B, B=T_{2} A$, 那么当 $A X=0$ 时, 显然有 $B X=$ $T_{2} A X=0$, 同理, 当 $B X=0$ 时, 有 $A X=T_{1} B X=0$, 这说明方程组 $A X=0$ 与 $B X=0$ 同解.

必要性. 已知方程组 $A X=0$ 与 $B X=0$ 店解, 则 $A X=0$ 时, 有 $A X=B X=0$, 反之, 若 $A X=B X=0$,显然也有$A X=0$,这说明方程组 $A X=0$ 止 $\left(\begin{array}{c}A \\ B\end{array}\right) X=0$ 同解, 进而 $r(A)=r\left(\begin{array}{l}A \\ B\end{array}\right) .$ 注意到 $A$ 的行向量均为$\left(\begin{array}{c}A \\ B\end{array}\right)$ 的行向量, 所以结合它们的秩相同可知 $A$ 行向量组的极大线性无关组也是 $\left(\begin{array}{c}A \\ B\end{array}\right)$ 行向量组 的极大线性无关组, 这说明 $B$ 的行向量作为 $\left(\begin{array}{c}A \\ B\end{array}\right)$ 的行向量均可以 $A$ 行向量组的极大线性无关组线性表 出, 即 $B$ 的行向量均可以由 $A$ 的行向量组线性表出, 所以存在矩阵 $T_{2}$ 使得 $B=T_{2} A$.

同理, 根据已知可得 $B X=0$ 与 $\left(\begin{array}{c}A \\ B\end{array}\right) X=0$ 同解, 由此可知 $A$ 的行向量均可由 $B$ 的行向量组线性表 出, 即存在矩阵 $T_{1}$ 使得 $A=T_{1} B .$

\vspace{1ex}
{\heiti 25.}设矩阵 $A, D$ 分别为 $n$ 阶和 $m$ 阶可逆矩阵, $B, C$ 分别为 $n \times m$ 和 $m \times n$ 的矩阵. 证明:

1. $\left|\begin{array}{ll}A & B \\ C & D\end{array}\right|=|A|\left|D-C A^{-1} B\right|$.

2. $r\left(A-B D^{-1} C\right)-r\left(D-C A^{-1} B\right)=n-m$.

解答. 1. 由于 $A$ 可逆, 所以
$$
    \left(\begin{array}{cc}
            E_{n}     & O     \\
            -C A^{-1} & E_{m}
        \end{array}\right)\left(\begin{array}{ll}
            A & B \\
            C & D
        \end{array}\right)=\left(\begin{array}{cc}
            A & B            \\
            O & D-C A^{-1} B
        \end{array}\right) .
$$

上述矩阵等式两边取行列式可得
$$
    \left|\begin{array}{cc}
        A & B \\
        C & D
    \end{array}\right|=|A|\left|D-C A^{-1} B\right|
$$

2. 由于 $A$ 可逆, 所以有
$$
    \left(\begin{array}{cc}
            E_{n}     & O     \\
            -C A^{-1} & E_{m}
        \end{array}\right)\left(\begin{array}{cc}
            A & B \\
            C & D
        \end{array}\right)\left(\begin{array}{cc}
            E_{n} & -A^{-1} B \\
            O     & E_{m}
        \end{array}\right)=\left(\begin{array}{cc}
            A & O            \\
            O & D-C A^{-1} B
        \end{array}\right) .
$$

于是
\begin{equation}
    \left(\begin{array}{cc}A & B \\ C & D\end{array}\right)=r\left(\begin{array}{cc}A & O \\ O & D-C A^{-1} B\end{array}\right)=r(A)+r\left(D-C A^{-1} B\right)=n+r\left(D-C A^{-1} B\right)
    \tag{1}
\end{equation}

同时,又由于 $D$ 可逆, 所以有
$$
    r\left(\begin{array}{cc}
            E_{n} & -B D^{-1} \\
            O     & E_{m}
        \end{array}\right)\left(\begin{array}{cc}
            A & B \\
            C & D
        \end{array}\right)\left(\begin{array}{cc}
            E_{n}     & O     \\
            -D^{-1} C & E_{m}
        \end{array}\right)=\left(\begin{array}{cc}
            A-B D^{-1} C & O \\
            O            & D
        \end{array}\right) .
$$

于是
\begin{equation}
    r\left(\begin{array}{cc}
            A & B \\
            C & D
        \end{array}\right)=r\left(\begin{array}{cc}
            A-B D^{-1} C & O \\
            O            & D
        \end{array}\right)=r(D)+r\left(A-B D^{-1} C\right)=m+r\left(A-B D^{-1} C\right)
    \tag{2}
\end{equation}

进而由 (1) 式与 (2) 式可知 $n+r\left(D-C A^{-1} B\right)=m+r\left(A-B D^{-1} C\right)$, 即
$$
    r\left(A-B D^{-1} C\right)-r\left(D-C A^{-1} B\right)=n-m .
$$

\vspace{1ex}
{\heiti 26.}已知 $n$ 阶矩阵 $M_{n}=\left(\dfrac{1-a_{i}^{n} a_{j}^{n}}{1-a_{i} a_{j}}\right)$, 证明: 当 $a_{1}, a_{2}, \cdots, a_{n}$ 为互不相同的实数时, $M_{n}$ 为正 定矩阵.
解答. 显然 $M_{n}$ 为实对称矩阵. 另外, 注意到
$$
    \dfrac{1-a_{i}^{n} a_{j}^{n}}{1-a_{i} a_{j}}=1+a_{i} a_{j}+a_{i}^{2} a_{j}^{2}+\ldots+a_{i}^{n-1} a_{j}^{n-1}
$$

于是
\begin{equation*}
    \begin{aligned}
        M_{n} & =\left(\begin{array}{ccc}
                1+a_{1} a_{1}+a_{1}^{2} a_{1}^{2}+\cdots+a_{1}^{n-1} a_{1}^{n-1} & \cdots & 1+a_{1} a_{n}+a_{1}^{2} a_{n}^{2}+\cdots+a_{1}^{n-1} a_{n}^{n-1} \\
                \vdots                                                           &        & \vdots                                                           \\
                1+a_{n} a_{1}+a_{n}^{2} a_{1}^{2}+\cdots+a_{n}^{n-1} a_{1}^{n-1} & \cdots & 1+a_{n} a_{n}+a_{n}^{2} a_{n}^{2}+\cdots+a_{n}^{n-1} a_{n}^{n-1}
            \end{array}\right)                                                        \\
              & =\left(\begin{array}{cccc}
                1      & a_{1}  & \cdots & a_{1}^{n-1}      \\
                1      & a_{2}  & \cdots & \Delta_{2}^{n-1} \\
                \vdots & \vdots &        & \vdots           \\
                1      & a_{n}  & \cdots & a_{n}^{n-1}
            \end{array}\right)\left(\begin{array}{cccc}
                1           & 1           & \cdots & 1           \\
                a_{1}       & a_{2}       & \cdots & a_{n}       \\
                \vdots      & \vdots      &        & \vdots      \\
                a_{1}^{n-1} & a_{2}^{n-1} &        & a_{n}^{n-1}
            \end{array}\right)=C^{\prime} C .
    \end{aligned}
\end{equation*}

其中
$$
    C=\left(\begin{array}{cccc}
            1           & 1           & \cdots & 1           \\
            a_{1}       & a_{2}       & \cdots & a_{n}       \\
            \vdots      & \vdots      &        & \vdots      \\
            a_{1}^{n-1} & a_{2}^{n-1} &        & a_{n}^{n-1}
        \end{array}\right)
$$

由范德蒙行列式的性质可知
$$
    |C|=\prod_{1 \leq i<j \leq n}\left(a_{j}-a_{i}\right) \neq 0 .
$$

所以 $C$ 为可逆实矩阵. 于是对任意的 $X \in \mathbb{R}^{n}$, 总有
$$
    X^{\prime} M_{n} X=X^{\prime} C^{\prime} C X=(C X)^{\prime}(C X) \geq 0 .
$$

并且当 $X^{\prime} M_{n} X=(C X)^{\prime}(C X)=0$ 时, 有 $C X=0$, 再结合 $C$ 可逆知 $X=0 .$ 这说明 $M_{n}$ 为正定矩阵.

\vspace{1ex}
{\heiti 27.}设 $\alpha$ 为实线性空间 $\mathbb{R}^{3}$ 上的线性变换, $\mathscr{E}$ 为恒等变换, $\propto$ 的特征多项式为 $\lambda^{3}-1$, 令 $V_{1}=\{\alpha \mid(\mathscr{\alpha}-\mathscr{\delta}) \alpha=0\}, V_{2}=\left\{\alpha \mid\left(\mathscr{\alpha}^{2}+\mathscr{A}+8\right) \alpha=0\right\} .$ 证明:

1. $V_{1}$ 和 $V_{2}$ 都是 $\propto$ 的不变子空间.

2. $\mathbb{R}^{3}=V_{1} \oplus V_{2}$

解答. $1 .$ 若 $\alpha \in V_{1}$, 则有 $(\mathscr{A}-\mathscr{E}) \alpha=0$, 于是
$$
    (\alpha-\mathscr{}) \mathscr{A} \alpha=\mathscr{A}(\mathscr{A}-\mathscr{E}) \alpha=0 .
$$

这说明 $\mathscr{A} \alpha \in V_{1}$, 即 $V_{1}$ 为 $\mathscr{A} \mathscr{\text { 的不变子空间. }}$

同理, 若 $\alpha \in V_{2}$, 则有 $\left(\mathscr{A}^{2}+\mathscr{A}+\mathscr{E}\right) \alpha=0$, 于是
\begin{equation*}
    \left(\mathscr{A}^{2}+\mathscr{A}+\mathscr{E}\right)\mathscr{A} \alpha=\mathscr{A}\left(\mathscr{A}^{2}+\mathscr{A}+\mathscr{E}\right) \alpha=0
\end{equation*}

这说明 $\mathscr{A} \alpha \in V_{2}$, 即 $V_{2}$ 也为 $\mathscr{A}$ 的不变子空间.

2. 由于 $\lambda^{3}-1=(\lambda-1)\left(\lambda^{2}+\lambda+1\right)$, 并且根据 $\lambda^{2}+\lambda+1$ 不以 1 为根可知 $\left(\lambda-1, \lambda^{2}+\lambda+1\right)=1$, 即存
在 $u(\lambda), v(\lambda) \in \mathbb{R}[x]$, 使得
$$
    u(\lambda)(\lambda-1)+v(\lambda)\left(\lambda^{2}+\lambda+1\right)=1
$$

进而
$$
    u(\mathscr{A})(\mathscr{A}-\mathscr{S})+v(\mathscr{A})\left(\mathscr{A}^{2}+\mathscr{A}+\mathscr{E}\right)=\mathcal{E} .
$$

所以对任意的 $\alpha \in \mathbb{R}^{3}$, 均有
\begin{equation}
    \alpha=u(\mathscr{A})(\mathscr{A}-\mathscr{E}) \alpha+v(\mathscr{A})\left(\mathscr{A}^{2}+\mathscr{A}+\mathscr{E}\right) \alpha=\alpha_{1}+\alpha_{2}
    \tag{3}
\end{equation}

其中 $\alpha_{1}=u(\mathscr{A})(\mathscr{A}-\mathscr{S}) \alpha, \alpha_{2}=v(\mathscr{A})\left(\mathscr{A}^{2}+\mathscr{A}+\mathscr{E}\right) \alpha .$

由于 $\lambda^{3}-1$ 为 $d$ 的特征多项式, 所以 $\alpha^{3}-\mathscr{E}=\mathscr{O}$, 于是
\begin{gather*}
    \left(\mathscr{A}^{2}+\mathscr{A}+\mathscr{E}\right)\alpha_1=u(\mathscr{A})(\mathscr{A}-\mathscr{E})(\mathscr{A})\left(\mathscr{A}^{2}+\mathscr{A}+\mathscr{E}\right) \alpha=u(\mathscr{A})(\mathscr{A}^3-\mathscr{E})\alpha=0 \\
    \left(\mathscr{A}-\mathscr{E}\right)\alpha_2=v(\mathscr{A})(\mathscr{A}-\mathscr{E})(\mathscr{A})\left(\mathscr{A}^{2}+\mathscr{A}+\mathscr{E}\right) \alpha=u(\mathscr{A})(\mathscr{A}^3-\mathscr{E})\alpha=0
\end{gather*}

这说明 $\alpha_{1} \in V_{2}, \alpha_{2} \in V_{1}$, 从而结合 $V_{1}, V_{2}$ 为 $\mathbb{R}^{3}$ 的线性子空伯可知
$$\mathbb{R}^{3}=V_{1}+V_{2}$$

另外, 对任意的 $\alpha \in V_{1} \cap V_{2}$, 则有 $(\propto-8) \alpha=\left(d^{2}+\mathscr{A}+\delta\right) \alpha=0$, 从而代入到(3) 式可知
\begin{equation*}
    \alpha=u(\mathscr{A})(\mathscr{A}-\mathscr{E}) \alpha+v(\mathscr{A})\left(\mathscr{A}^{2}+\mathscr{A}+\mathscr{E}\right) \alpha=0
\end{equation*}

即 $V_{1} \cap V_{2}=\{0\}$, 进而
$$
    \mathbb{R}^{3}=V_{1} \oplus V_{2}
$$

\vspace{1ex}
{\heiti 28.}设 $A, B$ 都是 $n$ 阶实对称矩阵, 证明: 存在正交矩阵 $U$, 使得 $U^{-1} A U$ 和 $U^{-1} B U$ 同时为对角 矩阵的充分必要条件是 $A B=B A$.

解答. 必要性. 若存在正交矩阵 $U$, 使得 $U^{-1} A U$ 和 $U^{-1} B U$ 同时为对角矩阵. 则有
$$U^{-1} A U U^{-1} B U=U^{-1} B U U^{-1} A U$$

于是 $A B=B A$.

充分性. 已知 $n$ 阶实对称矩阵 $A, B$ 满足 $A B=B A$, 首先存在正交矩阵 $P$ 使得
$$
    P^{-1} A P=\left(\begin{array}{cccc}
            \lambda_{1} E_{r_{1}} &                       &        &                       \\
                                  & \lambda_{2} E_{r_{2}} &        &                       \\
                                  &                       & \ddots &                       \\
                                  &                       &        & \lambda_{s} E_{r_{r}}
        \end{array}\right) .
$$

其中 $\lambda_{1}, \lambda_{2}, \cdots, \lambda_{s}$ 是 $A$ 的所有互异特征值, $E_{r_{1}}, E_{r_{2}}, \cdots, E_{r}$, 分别是 $r_{1}, r_{2}, \cdots, r_{s}$ 级单位矩阵. 由 $A B=B A$

可得
$$P^{-1} A P P^{-1} B P=P^{-1} B P P^{-1} A P$$

现在记 $P^{-1} B P=\left(B_{i j}\right)$, 其中 $B_{i j}$ 为 $r_{i} \times r_{j}$ 阶矩阵, 代入到上式可知
$$
    \lambda_{i} B_{i j}=\lambda_{j} B_{i j}
$$

于是当 $i \neq j$ 时, 有 $B_{i j}=O$, 即
$$
    P^{-1} B P=\left(\begin{array}{cccc}
            B_{11} &        &        &         \\
                   & B_{22} &        &         \\
                   &        & \ddots &         \\
                   &        &        & B_{s s}
        \end{array}\right) .
$$

由于 $B$ 为实对称矩阵, 所以 $B_{11}, B_{22}, \cdots, B_{s s}$ 均为实对称矩阵, 于是对任意的 $i=1,2, \cdots, s$, 存在 $r_{i}$ 阶正交 矩阵 $Q_{i}$ 使得 $Q_{i}^{-1} B_{i i} Q_{i}$ 为对角矩阵, 取 $Q=\operatorname{diag}\left\{Q_{1}, Q_{2}, \cdots, Q_{s}\right\}$, 则 $Q$ 为 $n$ 阶正交矩阵, 且
$$
    Q^{-1} P^{-1} B P Q=\left(\begin{array}{llll}
            Q_{1}^{-1} B_{11} Q_{1} &                         &        &                          \\
                                    & Q_{2}^{-1} B_{22} Q_{2} &        &                          \\
                                    &                         & \ddots &                          \\
                                    &                         &        & Q_{s}^{-1} B_{s s} Q_{s}
        \end{array}\right)
$$

为对角矩阵. 同时
$$
    \begin{aligned}
        Q^{-1} P^{-1} A P Q & =\left(\begin{array}{cccc}
                Q_{1}^{-1}\left(\lambda_{1} E_{r_{1}}\right) Q_{1} &                                                    &        &                                                    \\
                                                                   & Q_{2}^{-1}\left(\lambda_{2} E_{r_{2}}\right) Q_{2} &        &                                                    \\
                                                                   &                                                    & \ddots &                                                    \\
                                                                   &                                                    &        & Q_{S}^{-1}\left(\lambda_{s} E_{r_{s}}\right) Q_{2}
            \end{array}\right) \\
                            & =\left(\begin{array}{cccc}
                \lambda_{1} E_{r_{1}} &                       &        &                       \\
                                      & \lambda_{2} E_{r_{2}} &        &                       \\
                                      &                       & \ddots &                       \\
                                      &                       &        & \lambda_{s} E_{r_{0}}
            \end{array}\right)
    \end{aligned}
$$

仍为对角矩阵. 所以取 $U=P Q$ 就有 $U^{-1} A U, U^{-1} B U$ 同时为对角矩阵, 并且 $U$ 也为正交矩阵.

\vspace{1ex}
{\heiti 29.}若 $A$ 为 $n \times n$ 的半正定矩阵, $B$ 为 $n \times n$ 的实矩阵. 若存在正整数 $s$, 使得 $A^{s} B=B A^{s}$, 证明: $A B=B A .$
由 $A$ 半正定知存在正交阵 $P$ 使得
$$
    P^{\mathrm{T}} A P=\Lambda=\operatorname{diag}\left(\lambda_{1} E_{n_{1}}, \cdots, \lambda_{t} E_{n_{t}}\right), \lambda_{i} \geq 0 \text { 互异 }
$$

于是
$$
    \begin{aligned}
        A^{s} B=B A^{s} \Rightarrow & P^{\mathrm{T}} A^{s} P \cdot P^{\mathrm{T}} B P=P^{\mathrm{T}} B P \cdot P^{\mathrm{T}} A^{s} P                                                                                                               \\
        \Rightarrow                 & \operatorname{diag}\left(\lambda_{1}^{s} E_{n_{1}}, \cdots, \lambda_{t}^{s} E_{n_{t}}\right) \tilde{B}=\tilde{B} \operatorname{diag}\left(\lambda_{1}^{s} E_{n_{1}}, \cdots, \lambda_{t}^{s} E_{n_{t}}\right) \\
                                    & \left(\tilde{B}=P^{\mathrm{T}} B P=\left(\tilde{B}_{i j}\right), \text { 分块如 } \Lambda\right)                                                                                                              \\
        \Rightarrow                 & \lambda_{i}^{s} \tilde{B}_{i j}=\tilde{B}_{i j} \lambda_{j}^{s} \Rightarrow \forall i \neq j, \tilde{B}_{i j}=0                                                                                               \\
        \Rightarrow                 & \tilde{B}=\operatorname{diag}\left(\tilde{B}_{11}, \cdots, \tilde{B}_{t t}\right)                                                                                                                             \\
        \Rightarrow                 & \operatorname{diag}\left(\lambda_{1} E_{n_{1}}, \cdots, \lambda_{t} E_{n_{t}}\right) \tilde{B}=\tilde{B} \operatorname{diag}\left(\lambda_{1} E_{n_{1}}, \cdots, \lambda_{t} E_{n_{t}}\right)                 \\
        \Rightarrow                 & P \operatorname{diag}\left(\lambda_{1} E_{n_{1}}, \cdots, \lambda_{t} E_{n_{t}}\right) P^{T} \cdot P \tilde{B} P^{\mathrm{T}}                                                                                 \\
                                    & =P \tilde{B} P^{\mathrm{T}} \cdot P \operatorname{diag}\left(\lambda_{1} E_{n_{1}}, \cdots, \lambda_{t} E_{n_{t}}\right) P^{\mathrm{T}}                                                                       \\
        \Rightarrow                 & A B=B A .
    \end{aligned}
$$

\vspace{1ex}
{\heiti 30.}已知复数域上的两个 $n$ 阶矩阵
$$
    A=\left(\begin{array}{ccccc}
            0 & 1 &        &        &   \\
              & 0 & 1      &        &   \\
              &   & \ddots & \ddots &   \\
              &   &        & 0      & 1 \\
            1 &   &        &        & 0
        \end{array}\right), B=\operatorname{diag}\left(\xi_{1}, \xi_{2}, \cdots, \xi_{n}\right)
$$

其中 $\xi_{i}(i=1,2, \cdots, n)$ 为全部 $n$ 次单位根, 证明: $A$ 与 $B$ 相似.
$$
    \begin{aligned}
        |\lambda E-A|= & \left|\begin{array}{ccccc}
            \lambda & -1      &        &         &    \\
                    & \lambda & -1     &         &    \\
                    &         & \ddots & \ddots  &    \\
                    &         &        & \lambda & -1
        \end{array}\right|                                \\
        =              & \lambda \cdot    \lambda^{n-1} + (-1)^{n+1} \cdot (-1) \cdot (-1)^{n-1} \\
        =              & \lambda^{n}-1=\prod_{k=1}^{n}\left(\lambda-\xi_{k}\right)
    \end{aligned}
$$

知 $A$ 有 $n$ 个互异特征值, 而有 $n$ 个线性无关的特征向量, 故 $A$ 可可对角 化, 即存在正交阵 $P$, 使得 $P^{\mathrm{T}} A P=B .$

\vspace{1ex}
{\heiti 31.}设 $V_{1}, V_{2}$ 是有限维欧氏空间 $V$ 的子空间, 且 $\operatorname{dim} V_{1}<\operatorname{dim} V_{2}$, 证明:$V_{2}$ 中必有非零向量垂直与 $V_{1}$ 中的一切向量.
$$
    \begin{aligned}
        \operatorname{dim}\left(V_{2} \cap V_{1}^{\perp}\right) & =\operatorname{dim} V_{2}+\operatorname{dim} V_{1}^{\perp}-\operatorname{dim}\left(V_{2} \cap V_{1}^{\perp}\right)                     \\
                                                                & =\operatorname{dim} V_{2}+\left(n-\operatorname{dim} V_{1}\right)-\operatorname{dim}\left(V_{2} \cap V_{1}^{\perp}\right)              \\
                                                                & =\left(\operatorname{dim} V_{2}-\operatorname{dim} V_{1}\right)+\left[n-\operatorname{dim}\left(V_{2} \cap V_{1}^{\perp}\right)\right] \\
                                                                & \geq \operatorname{dim} V_{2}-\operatorname{dim} V_{1}>0
    \end{aligned}
$$

知 $\exists 0 \neq \alpha \in V_{2} \cap V_{1}^{\perp} .$ 此 $\alpha$ 即满足题意.

\vspace{1ex}
{\heiti 32.}设矩阵 $A=\left(\begin{array}{lll}2 & 1 & 0 \\ 1 & 2 & 0 \\ 1 & a & b\end{array}\right)$ 仅有两个不同的特征值. 若 $A$ 相似于对角矩阵,求 $a, b$ 的值,并求可
逆矩阵 $P$, 使 $P^{-1} A P$ 为对角矩阵.

解: 由 $|\lambda E-A|=\left|\begin{array}{ccc}\lambda-2 & -1 & 0 \\ -1 & \lambda-2 & 0 \\ -1 & -a & \lambda-b\end{array}\right|=(\lambda-b)(\lambda-3)(\lambda-1)=0$

当 $b=3$ 时,由 $A$ 相似对角化可知,二重根所对应特征值至少存在两个线性无关的特征向量,

则$(3 E-A)=\left(\begin{array}{ccc}1 & -1 & 0 \\ -1 & 1 & 0 \\ -1 & -a & 0\end{array}\right)$ 知, $\quad a=-1$

此时,$\lambda_{1}=\lambda_{2}=3$ 所对应特征向量为 $\alpha_{1}=\left(\begin{array}{l}1 \\ 1 \\ 0\end{array}\right), \alpha_{2}=\left(\begin{array}{l}0 \\ 0 \\ 1\end{array}\right)$,

$\lambda_{3}=1$ 所对应的特征向量为 $\alpha_{3}=\left(\begin{array}{c}-1 \\ 1 \\ 1\end{array}\right)$, 则 $P^{-1} A P=\left(\begin{array}{lll}3 & & \\ & 3 & \\ & & 1\end{array}\right)$

当 $b=1$ 时,由 $A$ 相似对角化可知,二重根所对应特征值至少存在两个线性无关的特征向量,

则 $(E-A)=\left(\begin{array}{rrr}-1 & -1 & 0 \\ -1 & -1 & 0 \\ -1 & -a & 0\end{array}\right)$, 知 $a=1$,

此时, $\lambda_{1}=\lambda_{2}=1$ 所对应特征向量为 $\beta_{1}=\left(\begin{array}{c}-1 \\ 1 \\ 0\end{array}\right), \beta_{2}=\left(\begin{array}{l}0 \\ 0 \\ 1\end{array}\right)$,

$\lambda_{3}=3$ 所对应的特征向量为 $\alpha_{3}=\left(\begin{array}{l}1 \\ 1 \\ 1\end{array}\right)$, 则 $P^{-1} A P=\left(\begin{array}{lll}1 & & \\ & 1 & \\ & & 3\end{array}\right)$

\vspace{1ex}
{\heiti 33.}曲线$\left(x^{2}+y^{2}\right)^{2}=x^{2}-y^{2}(x \geq 0, y \geq 0) $与  $x $ 轴围成的区域为 $D$, 求$ \iint_{D} x y dxdy $

解: $$
    \begin{aligned}
        \iint_{D} x y d x d y & =\int_{0}^{-\frac{\pi}{4}} d \theta \int_{0}^{\sqrt{\cos 2 \theta}} r^{3} \sin \theta \cos \theta d r \\
                              & =\int_{0}^{\frac{\pi}{4}} \dfrac{1}{4} \cos ^{2} 2 \theta \sin \theta \cos \theta d \theta            \\
                              & =-\int_{0}^{\frac{\pi}{4}} \dfrac{1}{16} \cos ^{2} 2 \theta d \cos 2 \theta                           \\
                              & =-\left.\dfrac{1}{48} \cos ^{3} 2 \theta\right|_{0} ^{\frac{\pi}{4}}=\dfrac{1}{48} .
    \end{aligned}
$$

\vspace{1ex}
{\heiti 33.}函数 $y=y(x)$ 的微分方程 $x y^{\prime}-6 y=-6$, 满足 $y(\sqrt{3})=10$,

(1)求 $y(x)$;

(2) $P$ 为曲线 $y=y(x)$ 上的一点,曲线 $y=y(x)$ 在点 $P$ 的法线在 $y$ 轴上的截距为 $I_{y}$, 为使 $I_{y}$ 最小,求 $P$ 的坐标.

解:(1) $y^{\prime}-\dfrac{6}{x} y=-\dfrac{6}{x}, \therefore y=e^{\int \frac{6}{x} x}\left[\int\left(-\frac{6}{x}\right) e^{-\int_{x}^{6} d x} d x+C\right]=x^{6}\left(\dfrac{1}{x^{6}}+C\right)=1+C x^{6}$

将 $y(\sqrt{3})=10$ 代入, $C=\dfrac{1}{3}, \quad \therefore y(x)=1+\dfrac{x^{6}}{3}$.

(2)设 $P(x, y)$, 则过 $P$ 点的切线方程为 $Y-y=2 x^{5}(X-x)$,

法线方程为 $Y-y=-\dfrac{1}{2 x^{5}}(X-x)$,

令 $X=0, \quad \therefore Y=I_{y}=1+\dfrac{x^{6}}{3}+\dfrac{1}{2 x^{4}}$, 偶函数,为此仅考虑 $(0,+\infty)$

令 $\left(I_{y}\right)^{\prime}=2 x^{5}-\dfrac{2}{x^{5}}=0, \quad x=1$

$\therefore x \in(0,1),\left(I_{y}\right)^{\prime}<0, I_{y}>I_{y}(1)=\dfrac{11}{6} ; \quad x \in(1,+\infty),\left(I_{y}\right)>0, \quad I_{y}>I_{y}(1)=\dfrac{11}{6}$

$\therefore P\left(\pm 1, \dfrac{4}{3}\right)$ 时, $I_{y}$ 有最小值 $\dfrac{11}{6} .$

\vspace{1ex}
{\heiti 34.}$f(x)$满足$\int \dfrac{f(x)}{\sqrt{x}} d x=\dfrac{1}{6} x^{2}-x+C$, $L$ 为曲线$y=f(x)(4 \leq x \leq 9)$, $L$的弧长为s, $L$绕 $x$ 轴旋转一周所形成的曲面的面积为$A$ ,求$s$和$A$.

解:$\dfrac{f(x)}{\sqrt{x}}=\dfrac{1}{3} x-1, \quad f(x)=\dfrac{1}{3} x^{\dfrac{3}{2}}-x^{\dfrac{1}{2}}$

曲线的弧长 $ s=\int_{4}^{9} \sqrt{1+y^{\prime 2}} d x=\int_{4}^{9} \sqrt{\dfrac{1}{2}+\dfrac{x}{4}+\dfrac{1}{4 x}} d x=\dfrac{22}{3}.$

曲面的侧面积$A=2 \pi \int_{4}^{9} y \sqrt{1+y^{\prime 2}} d x=2 \pi \int_{4}^{9}\left(\dfrac{1}{3} x^{\frac{3}{2}}-x^{\frac{1}{2}}\right) \sqrt{x+\dfrac{1}{x}+2} d x =\dfrac{425 \pi}{9}$

\vspace{1ex}
{\heiti 35.}求极限$\lim _{x \rightarrow 0}\left(\dfrac{1+\int_{0}^{x} e^{t^{2}} d t}{e^{x}-1}-\dfrac{1}{\sin x}\right)$

解: $\lim _{x \rightarrow 0}\left(\dfrac{1+\int_{0}^{x} e^{t^{2}} d t}{e^{x}-1}-\dfrac{1}{\sin x}\right)=\lim _{x \rightarrow 0} \dfrac{\sin x-1-\int_{0}^{x} e^{t^{2}} d t}{\left(e^{x}-1\right) \sin x} $

又因为$\int_{0}^{x} e^{t^{2}} d t=\int_{0}^{x}\left(1+t^{2}+o\left(t^{2}\right)\right) d t=x+\dfrac{1}{3} x^{3}+o\left(x^{3}\right),$故

原式=$\lim _{x \rightarrow 0} \dfrac{\left(x-\dfrac{1}{3 !} x^{3}+o\left(x^{3}\right)\right)\left(1+x+\dfrac{1}{3 !} x^{3}+o\left(x^{3}\right)\right)-x-\dfrac{1}{2} x^{2}+o\left(x^{2}\right)}{x^{2}}$

$\qquad =\lim _{x \rightarrow 0} \dfrac{\dfrac{1}{2} x^{2}+o\left(x^{2}\right)}{x^{2}}=\dfrac{1}{2} $

\vspace{1ex}
{\heiti 36.}在区间 $(0,2)$ 上随机取一点,将该区间分成两段,较短的一段长度记为 $X$,较长的一段长度记为 $Y$,令 $Z=\dfrac{Y}{X}$.

(1)求 $X$ 的概率密度;

(2)求 $Z$ 的概率密度.

(3)求 $E\left(\frac{X}{Y}\right)$.

解:

(1)由题知: $\quad X \sim f(x)=\left\{\begin{array}{c}1,0<x<1 \\ 0, \text { 其他 }\end{array}\right.$;

(2)由 $Y=2-X$, 即 $Z=\frac{2-X}{X}$, 先求 $Z$ 的分布函数:
$F_{Z}(z)=P\{Z \leq z\}=P\left\{\frac{2-X}{X} \leq z\right\}=P\left\{\frac{2}{X}-1 \leq z\right\}$

当 $z<1$ 时, $F_{Z}(z)=0$;

当 $z \geq 1$ 时, $F_{Z}(z)=P\left\{\frac{2}{X}-1 \leq z\right\}=1-P\left\{X \leq \frac{2}{z+1}\right\}=1-\int_{0}^{\frac{2}{z+1}} 1 d x=1-\frac{2}{z+1} ;$

$f_{Z}(z)=\left(F_{z}(z)\right)^{\prime}=\left\{\begin{array}{c}\frac{2}{(z+1)^{2}}, z \geq 1 \\ 0, \text { 其他 }\end{array} ;\right.$

(3) $E\left(\frac{X}{Y}\right)=E\left(\frac{X}{2-X}\right)=\int_{0}^{1} \frac{x}{2-x} \cdot 1 d x=-1+2 \ln 2$.

\vspace{1ex}
{\heiti 37.}已知曲线 $C:\left\{\begin{array}{l}x^{2}+2 y^{2}-z=6 \\ 4 x+2 y+z=30\end{array}\right.$, 求 $C$ 上的点到 $x o y$ 坐标面距离的最大值.

解: 设拉格朗日函数 $L(x, y, z, \lambda, \mu)=z^{2}+\lambda\left(x^{2}+2 y^{2}-z-6\right)+\mu(4 x+2 y+z-30)$
$$
    \begin{array}{l}
        L_{x}^{\prime}=2 x \lambda+4 u=0 \\
        L^{\prime} y=4 y \lambda+2 u=0   \\
        L_{z}^{\prime}=2 z-\lambda+u=0   \\
        x^{2}+2 y^{2}-z=6                \\
        4 x+2 y+z=30
    \end{array}
$$

解得驻点: $(4,1,12),(-8,-2,66)$

C上的点 $(-8,-2,66)$ 到 xoy 面距离最大为 $66$.

\vspace{1ex}
{\heiti 37.}设 $u_{n}(x)=e^{-n x}+\frac{1}{n(n+1)} x^{n+1}(n=1,2, \ldots), \text { 求级数 } \sum_{n=1}^{\infty} u_{n}(x) $ 的收敛域及和函数.

解:

$\begin{array}{l}
        S(x)=\sum_{n=1}^{\infty} u_{n}(x)=\sum_{n=1}^{\infty}\left[e^{-n x}+\frac{1}{n(n+1)} x^{n+1}\right] ,\text {收敛域 } (0,1]
        S_{1}(x)=\sum_{n=1}^{\infty} e^{-n x}=\frac{e^{-x}}{1-e^{-x}}, x \in(0,1]                                                                                       \\
        S_{2}(x)=\sum_{n=1}^{\infty} \frac{1}{n(n+1)} x^{n+1}=\sum_{n=1}^{\infty} \frac{x^{n+1}}{n}-\sum_{n=1}^{\infty} \frac{x^{n+1}}{n+1}=-x \ln (1-x)-[-\ln (1-x)-x] \\
        =(1-x) \ln (1-x)+x, \quad x \in(0,1)                                                                                                                            \\
        S_{2}(1)=\lim _{x \rightarrow 1^{-}} S_{2}(x)=1                                                                                                                 \\
        S(x)=\left\{\begin{array}{l}
            \frac{e^{-x}}{1-e^{-x}}+(1-x) \ln (1-x)+x, x \in(0,1) \\
            \frac{e}{e-1}, x=1
        \end{array}\right.
    \end{array}$

\vspace{1ex}
{\heiti 38.}求极限 $\lim _{x \rightarrow 0}\left(\dfrac{1+\int_{0}^{x} e^{t^{2}} d t}{e^{x}-1}-\dfrac{1}{\sin x}\right) .$

解: $\lim _{x \rightarrow 0}\left(\dfrac{1+\int_{0}^{x} e^{t^{2}} d t}{e^{x}-1}-\dfrac{1}{\sin x}\right)=\lim _{x \rightarrow 0} \dfrac{\sin x-1-\int_{0}^{x} e^{t^{2}} d t}{\left(e^{x}-1\right) \sin x}$

又因为 $\int_{0}^{x} e^{t^{2}} d t=\int_{0}^{x}\left(1+t^{2}+o\left(t^{2}\right)\right) d t=x+\dfrac{1}{3} x^{3}+o\left(x^{3}\right)$, 故

原式 $=\lim _{x \rightarrow 0} \dfrac{\left(x-\dfrac{1}{3 !} x^{3}+o\left(x^{3}\right)\right)\left(1+x+\dfrac{1}{3 !} x^{3}+o\left(x^{3}\right)\right)-x-\dfrac{1}{2} x^{2}+o\left(x^{2}\right)}{x^{2}}$

$=\lim _{x \rightarrow 0} \dfrac{\dfrac{1}{2} x^{2}+o\left(x^{2}\right)}{x^{2}}=\dfrac{1}{2} .$

\vspace{1ex}
{\heiti 38.}已知 $\lim _{x \rightarrow 0}\left[\alpha \arctan \frac{1}{x}+(1+|x|)^{\frac{1}{x}}\right]$ 存在,求 $\alpha$ 的值.

解: 要想极限存在,则左右极限相等;

又由于 $\lim _{x \rightarrow 0^{+}}\left[\alpha \arctan \dfrac{1}{x}+\left(1+|x|\right)^{\dfrac{1}{x}}\right]=\dfrac{\pi}{2} \alpha+e $

$\lim _{x \rightarrow 0^{-}}\left[\alpha \arctan \dfrac{1}{x}+(1+|x|)^{\dfrac{1}{x}}\right]=-\dfrac{\pi}{2} \alpha+\dfrac{1}{e} ;$

从而 $\dfrac{\pi}{2} \alpha+e=-\dfrac{\pi}{2} \alpha+\dfrac{1}{e}$, 即 $\alpha=\dfrac{1}{\pi}\left(\dfrac{1}{e}-e\right) .$

\vspace{1ex}
{\heiti 39.}在区间 $(0,2)$ 上随机取一点,将该区间分成两段,较短的一段长度记为 $X$,较长的一段长度记为$Y$, 令 $Z=\frac{Y}{X}$

(1)求 $X$ 的概率密度;

(2)求 $Z$ 的概率密度.

(3)求 $E\left(\frac{X}{Y}\right)$.

解:

(1) 由题知: $\quad x \sim f(x)=\left\{\begin{array}{c}
        1,0<x<1 \\
        0, \text { 其他 }
    \end{array} ;\right.$

(2)由 $y=2-x$, 即 $Z=\dfrac{2-X}{X}$, 先求 $Z$ 的分布函数:

$F_{z}(z)=P\{Z \leq z\}=P\left\{\dfrac{2-X}{X} \leq z\right\}=P\left\{\dfrac{2}{X}-1 \leq z\right\}$

当 $z<1$ 时, $F_{z}(z)=0$;

当 $z \geq 1$ 时, $F_{z}(z)=P\left\{\frac{2}{X}-1 \leq z\right\}=1-P\left\{X \leq \frac{2}{z+1}\right\}=1-\int_{0}^{\frac{2}{z+1}} 1 d x=1-\frac{2}{z+1} ;$

$f_{z}(z)=\left(F_{z}(z)\right)^{\prime}=\left\{\begin{array}{c}
        \dfrac{2}{(z+1)^{2}}, z \geq 1 \\
        0, \text { 其他 }
    \end{array}\right.$

(3) $E\left(\dfrac{X}{Y}\right)=E\left(\dfrac{X}{2-X}\right)=\int_{0}^{1} \dfrac{x}{2-x} \cdot 1 d x=-1+2 \ln 2$

\vspace{1ex}
{\heiti 40.}求函数 $f(x, y)=2 \ln |x|+\frac{(x-1)^{2}+y^{2}}{2 x^{2}}$ 的极值.

解:

(1) $\left\{\begin{array}{l}
        f_{x}^{\prime}=\dfrac{2 x^{2}+x-1-y^{2}}{x^{3}}=0 \\
        f_{y}^{\prime}=\dfrac{y}{x^{2}}=0
    \end{array}\right.$ 即 $\left\{\begin{array}{l}
        2 x^{2}+x-1-y^{2}=0 \\
        y=0
    \end{array}\right.$

得驻点 $(-1,0),\left(\dfrac{1}{2}, 0\right)$

(2)$\left\{\begin{array}{l}
        f_{x x}^{\prime \prime}=\dfrac{(4 x+1) x-3\left(2 x^{2}+x-1-y^{2}\right)}{x^{4}} \\
        f_{x y}^{\prime \prime}=\dfrac{-2 y}{x^{3}}                                      \\
        f_{y y}^{\prime \prime}=\dfrac{1}{x^{2}}
    \end{array}\right.$

(3) 驻点 $(-1,0)$ 处, $\mathrm{A}=3, \mathrm{~B}=0, \mathrm{C}=1, \quad A C-B^{2}=3>0, \quad A>0$

故 $f(x, y)$ 在 $(-1,0)$ 处取极小值 2 ;

驻点 $\left(\dfrac{1}{2}, 0\right)$ 处, $\mathrm{A}=24, \mathrm{~B}=0, \mathrm{C}=4, \quad A C-B^{2}=3>0, \quad A>0$

故 $f(x, y)$ 在 $\left(\dfrac{1}{2}, 0\right)$ 处取极小值 $\dfrac{1}{2}-2 \ln 2 .$

\vspace{1ex}
{\heiti 41.}求曲线 $y=\frac{x^{1+x}}{(1+x)^{x}}(x>0)$ 的斜渐近线方程.

解: $\lim _{x \rightarrow+\infty} \frac{y}{x}=\lim _{x \rightarrow+\infty} \frac{x^{1+x}}{(1+x)^{x} x}$

$=\lim _{x \rightarrow+\infty} \frac{x^{x}}{(1+x)}$

$=\lim _{x \rightarrow+\infty} \frac{\mathrm{e}^{x \ln x}}{\mathrm{e}^{x \tan (1+x)}}$

$=\lim _{x \rightarrow+\infty} \mathrm{e}^{x(\ln }$

$=\lim _{x \rightarrow+\infty} \mathrm{e}^{\mathrm{x} \operatorname{lin} \frac{x+1-1}{1+x}}$

$=\lim \mathrm{e}^{x \ln \left(1-\frac{1}{1+x}\right)}$

$=\lim _{x \rightarrow+\infty} e^{x\left(\frac{-1}{1+x}\right)_{/}^{-1}}={ }^{-1}$

$\lim _{x \rightarrow+\infty}\left(y-\mathrm{e}^{-1} x\right)$

$=\lim _{x \rightarrow+\infty}\left(\frac{x^{1+x}}{(1+x)^{x}}-\mathrm{e}^{-1} x\right)$

$=\lim _{x \rightarrow+\infty} x\left(\frac{x_{x}}{(1+x)^{x}}-\mathrm{e}-1\right)$

$=\lim _{x \rightarrow+\infty} x \cdot\left(e^{\frac{x \ln x}{1+x}-e^{-1}}\right)$

$=\lim _{x \rightarrow+\infty} e^{-1} x \cdot\left(x \ln \frac{x}{\underline{1+x}+1}\right) \mid$

$=\lim _{t \rightarrow 0^{+}} e^{-1} \frac{\frac{1}{t} \cdot \ln \frac{\frac{1}{t}}{1+\frac{1}{t}}+1}{t}$

$=\lim _{t \rightarrow 0^{+}} \mathrm{e}^{-1} \frac{\overline{t+1}}{t^{2}}$

$=\lim _{t \rightarrow 0^{+}} \frac{-1}{t^{2}}=\frac{1}{2} \mathrm{e}^{-1}$

$\therefore$ 曲线的斜渐近线方程为 $y=\mathrm{e}^{-1} x_{\pm}{ }^{1} \mathrm{e}^{-1}$

\vspace{1ex}
{\heiti 42.}已知函数 $f(x)$ 连续且$\lim _{x \rightarrow 0} \dfrac{f(x)}{x}=1, g(x)=\int_{0}^{1} f(x t) d t$, 求g '( $\left.x\right)$ 并证明 $g{ }^{\prime}(x)$ 在 $x=0$ 处连续.

解: 因为$\operatorname{lim } _{x \rightarrow 0 x}=1 \quad \therefore f(0)=\lim _{x \rightarrow 0} f(x)=0$
所以 $g(0)=\int_{0}^{1} f(0) d t=0$

因为 $g(x)=\int_{0}^{1} f(x t) d t \underline{x t=u} \dfrac{1}{x^{0}} \int_{0}^{x} f(u) d u$

当 $x \neq 0$ 时, $g^{\prime}(x)=\dfrac{x f(x)-\int_{0}^{x} f(u) d u}{x^{2}}$

当 $x=0$ 时,$\quad g^{\prime}(0)=\lim _{x \rightarrow 0} \dfrac{g(x)-g(0)}{x-0}=\lim _{x \rightarrow 0} \dfrac{\int_{0}^{x f(u) d u}}{x^{2}}=\dfrac{1}{2 x \rightarrow 0} \lim \dfrac{f(x)}{x}=\dfrac{1}{2}$

$g^{\prime}(x)=\left\{\begin{array}{cc}\dfrac{f f(u) d u}{x^{2}}, & x \neq 0 \\ \dfrac{1}{2}, & x=0\end{array}\right.$

又因为 $\lim _{x \rightarrow 0} g^{\prime}(x)=\lim _{x \rightarrow 0} \dfrac{x f(x)}{-x^{2}}-{ }_{0}^{x}(u) d u$

$=\lim _{x \rightarrow 0}\left[\dfrac{f(x)}{x}-\left.\dfrac{\int_{0}^{x} f(u) d u}{x \rightarrow 0} \dfrac{x^{x^{2}}}{x^{2}}\right|_{\mid\rfloor} ^{\mid}=1-\dfrac{1}{2}=\dfrac{1}{2}\right.$

$\therefore g^{\prime}(x)$ 在 $x=0$ 处连续

\vspace{1ex}
{\heiti 43.}求解行列式$\left|\begin{array}{cccc}a & 0 & -1 & 1 \\ 0 & a & 1 & -1 \\ -1 & 1 & a & 0 \\ 1 & -1 & 0 & a\end{array}\right|=$

解:
$\left|\begin{array}{cccc}a & 0 & -1 & 1 \\ 0 & a & 1 & -1 \\ -1 & 1 & a & 0 \\ 1 & -1 & 0 & a\end{array}\right|=\left|\begin{array}{cccc}a & 0 & -1 & 1 \\ 0 & a & 1 & -1 \\ -1 & 1 & a & 0 \\ 0 & 0 & a & a\end{array}\right|$

$=\left|\begin{array}{cccc}0 & a & -1+a^{2} & 1 \\ 0 & a & 1 & -1 \\ -1 & 1 & a & 0 \\ 0 & 0 & a & a\end{array}\right|=-\left|\begin{array}{ccc}a & -1+a^{2} & 1 \\ a & 1 & -1 \\ 0 & a & a\end{array}\right|$

$=-\left|\begin{array}{ccc}a & a^{2}-2 & 1 \\ a & 2 & -1 \\ 0 & 0 & a\end{array}\right|=a^{4}-4 a^{2}$

\vspace{1ex}
{\heiti 44.}已知 $\mathrm{f}(\mathrm{x}), \mathrm{g}(\mathrm{x}) \in \mathrm{P}[\mathrm{x}] .$ 证明: $\mathrm{f}(\mathrm{x}), \mathrm{g}(\mathrm{x}))=1$ 的充分必要条件是对 $\forall \mathrm{r}(\mathrm{x}), \mathrm{s}(\mathrm{x}), \exists \mathrm{p}(\mathrm{x}), \mathrm{q}(\mathrm{x})$

使得 $\mathrm{p}(\mathrm{x}) \mathrm{f}(\mathrm{x})+\mathrm{r}(\mathrm{x})=\mathrm{q}(\mathrm{x}) \mathrm{g}(\mathrm{x})+\mathrm{s}(\mathrm{x})$

(1) 充分性:
$$
    \begin{array}{r}
        (\mathrm{f}, \mathrm{g})=1 \Rightarrow \exists \mathrm{u}, \mathrm{v}, \mathrm{s}, \mathrm{t} \cdot \mathrm{uf}+\mathrm{vg}=1 \\
        \Rightarrow \mathrm{u}(\mathrm{s}-\mathrm{r}) \mathrm{f}+\mathrm{v}(\mathrm{s}-\mathrm{r}) \mathrm{g}=\mathrm{s}-\mathrm{r}
    \end{array}
$$

(2) 必要性 取 $\mathrm{r}=0, \mathrm{~s}=1$, 则
$$
    \mathrm{pf}-\mathrm{qg}=1 \Rightarrow(\mathrm{f}, \mathrm{g})=1
$$

\vspace{1ex}
{\heiti 45.}计算 n+1 阶行列式
$$
    D=\left|\begin{array}{cccc}
        a^{n}   & (a-1)^{n}(a-2)^{n}     & \cdots & (a-n)^{n}   \\
        a^{n-1} & (a-1)^{n-1}(a-2)^{n-1} & \cdots & (a-n)^{n-1} \\
        a^{n-2} & (a-1)^{n-2}(a-2)^{n-2} & \cdots & (a-n)^{n-2} \\
        \vdots  & \vdots                 & \vdots               \\
                & 1                      & 1      & 1
    \end{array}\right|
$$

解:将第 $\mathrm{n}+1$ 行与上面的两两互换,换到第一行,经过 $\mathrm{n}$ 次互换,再将第 $\mathrm{n}$ 行与上面两两 互换,换到第二行

得到
$$
    \mathrm{D}=(-1)^{n+(n-1)+\cdots+1}\left|\begin{array}{cccc}
        1       &                        & 1      & 1           \\
        a       & a-1                    & \ldots & a-n         \\
        a^{n-1} & (a-1)^{n-1}(a-2)^{n-1} & \ldots & (a-n)^{n-1} \\
        a^{n}   & (a-1)^{n}(a-2)^{n}     & \cdots & (a-n)^{n}
    \end{array}\right|
$$

对列也进行类似互换得到
$\mathrm{D}=(-1)^{\dfrac{n(n+1)}{2}}(-1)^{\dfrac{n(n+1)}{2}}\left|\begin{array}{cccc}1 & & 1 & 1 \\ a-n & a-(n-1) & \ldots & a \\ (a-n)^{n} & {[a-(n-1)]^{n}} & \ddots & \vdots \\ & \cdots & a^{n}\end{array}\right|=n !(n-1) ! \ldots .2 ! $(范德蒙
行列式)

\vspace{1ex}
{\heiti 46.}已知A $=\left(\begin{array}{lll}1 & 0 & 0 \\ 1 & 0 & 1 \\ 0 & 1 & 0\end{array}\right)$

(1).证明: $A^{n}=A^{n-2}+A^{2}-E \quad(\mathrm{n} \geq 3)$

(2).求 $A^{2020}$

解:

(1) 证明: 对 $\mathrm{n}$ 做数学归纳法, $\mathrm{n}=3$ 时,由 $\mathrm{A}$ 的特征多项式
$$\mathrm{f}(\lambda)=|\lambda \mathrm{E}-\mathrm{A}|=\lambda^{3}-\lambda^{2}-\lambda+1$$

及 Hamilton-Cayley 定理知 $\mathrm{f}(\mathrm{A})=0 \Rightarrow A^{3}=A+A^{2}-E$

若结论对 $\mathrm{n}$ 成立,则
$$
    \begin{array}{c}
        A^{n+1}=A^{n} A                                    \\
        =\left(A^{n-2}+A^{2}-E\right) A(\text {归纳假设 }) \\
        =A^{n-1}+A^{3}-A                                   \\
        =A^{n-1}+\left(A^{2}+A-E\right)-A                  \\
        =A^{n-1}+A^{2}-E
    \end{array}
$$

(2) 由(1)的结论
$$
    \begin{array}{rl}
                    & A^{2 k}-A^{2(k-1)}=A^{2}-E                                \\
        \Rightarrow & A^{2020}-A^{2}=1009\left(A^{2}-E\right)                   \\
        A^{2020}=1  & 010 A^{2}-1009 E=\left(\begin{array}{ccc}
                1    & 0 & 0 \\
                1010 & 1 & 0 \\
                1010 & 0 & 0
            \end{array}\right)
    \end{array}
$$
\vspace{1ex}
{\heiti 47.}巳知
$\alpha_{1}=(1,2,0) \quad \alpha_{2}=(1, a+2,-3 a) \alpha_{3}=(1,2,0) \quad \beta=(1,3,-3)$

求 a,b的值使得

(1). $\beta$ 不可被 $\alpha_{1}, \alpha_{2}, \quad \alpha_{3}$ 线性表出

(2). $\beta$ 可被 $\alpha_{1}, \alpha_{2}, \alpha_{3}$ 唯一线性表出,并求表达方式

(3). $\beta$ 可被 $\alpha_{1}, \alpha_{2}, \quad \alpha_{3}$ 线性表出且不唯一,并求表达式

(4). $\beta $可由 $\alpha_{1}, \alpha_{2}, \quad \alpha_{3}$ 线性表出
$$
    \begin{aligned}
        \Leftrightarrow & \exists x_{i} . \text { s.t. } \beta=\sum x_{i} a_{i}                                                                                    \\
        \Leftrightarrow & \exists x_{i} . \text { s.t. } \beta^{T}=\sum x_{i} \alpha_{i}^{T}                                                                       \\
        \Leftrightarrow & \mathrm{Ax}=\mathrm{b} \text { 有解,其中 } \mathrm{A}=\left(\alpha_{1}^{T}, \alpha_{2}^{T}, \alpha_{3}^{T}\right), \mathrm{b}=\beta^{T} \\
        \Leftrightarrow & \operatorname{rank}(\mathrm{A})=\operatorname{rank}(\mathrm{A}, \mathrm{b})
    \end{aligned}
$$


对(A,b)实施行初等变换
$$
    (A, b)=\left(\begin{array}{cccc}
            1 & 1    & -1    & 1  \\
            2 & a+2  & -b-2  & 3  \\
            0 & -3 a & a+2 b & -3
        \end{array}\right) \rightarrow\left(\begin{array}{cccc}
            1 & 1 & -1  & 1 \\
            0 & a & -b  & 1 \\
            0 & 0 & a-b & 0
        \end{array}\right)
$$

(1) 若 $\mathrm{a}=\mathrm{b}=0$, 则 $r(A)=1 \neq 2=r(A, b)$

此时,线性线性方程组无解, $\beta$ 不可由 $\alpha_{1}, \alpha_{2}, \alpha_{3}$ 线性表出

(2) 若 $\mathrm{a}=\mathrm{b} \neq 0$, 则$r(A)=2=r(A, b)$

此时线性方程组有无穷多解, $\beta$ 可由 $\alpha_{1}, \alpha_{2}, \quad \alpha_{3}$ 线性表出,由
$$(\mathrm{A}, \mathrm{b}) \rightarrow\left(\begin{array}{cccc}1 & 1 & -1 & 1 \\ 0 & 1 & -1 & 1 / a \\ 0 & 0 & 0 & 0\end{array}\right) \rightarrow\left(\begin{array}{cccc}1 & 0 & 0 & 1-1 / a \\ 0 & 1 & -1 & 1 / a \\ 0 & 0 & 0 & 0\end{array}\right)$$

知
$$\beta=\left(1-\frac{1}{\mathrm{a}}\right) x_{1}+\frac{1}{a} x_{2}+k x_{2}+k x_{3}=\left(1-\frac{1}{a}\right) x_{1}+\left(k+\frac{1}{a}\right) x_{2}+k x_{3} .$$

(3).若 $\mathrm{a} \neq \mathrm{b}, \mathrm{a}=0$, 则 $\mathrm{b} \neq 0$,
$$
    (A, b)=\left(\begin{array}{cccc}
            1 & 1 & -1  & 1 \\
            0 & 0 & 0   & 1 \\
            0 & 0 & a-b & 0
        \end{array}\right)
$$

由第二行知线性方程组无解, $\beta$ 不可由 $\alpha_{1}, \alpha_{2}, \alpha_{3}$ 线性表出.
(4).若 $\mathrm{a} \neq \mathrm{b}, \mathrm{a} \neq 0$, 则 $|\mathrm{A}| \neq 0$, 而 $\beta$ 可由 $\alpha_{1}, \alpha_{2}, \quad \alpha_{3}$ 唯一地线性表出.此时
$$
    \beta=\frac{\mathrm{a}-1}{\mathrm{a}} x_{1}+\frac{1}{a} x_{2}
$$

\vspace{1ex}
{\heiti 48.}已知
$$
    \mathrm{f}\left(x_{1} x_{2} \ldots x_{n}\right)=\sum_{i=1}^{n} \sum_{j=1}^{n} a_{i j} x_{i} x_{j}
$$

正负惯性指数分别为 p, q,且
$$
    \alpha_{1}, \alpha_{2}, \ldots, \alpha_{p}, \beta_{1}, \beta_{2}, \ldots, \beta_{p}
$$

为任意 $\mathrm{p}+\mathrm{q}$ 个正数。证明:存在非退化线性替换 $\mathrm{x}=$ Cy使得 $\mathrm{f}(\mathrm{x})=\alpha_{1} y_{1}^{2}+\cdots+\alpha_{p} y_{p}^{2}-\beta_{1} y_{p+1}^{2}-\cdots-\beta_{q} y_{p+q}^{2}$

解: $\mathrm{f}$ 的正负惯性指数分别为p, q

$\Rightarrow$ 存在可逆线性替换 $\mathrm{x}=\mathrm{Pz}$ 使得
$$\mathrm{f}(\mathrm{x})=Z_{1}^{2}+\cdots+Z_{P}^{2}-Z_{P+1}^{2}+\cdots-Z_{p+q}^{2}$$

$\Rightarrow$ 存在可逆线性替换 $\mathrm{x}=\operatorname{Pdiag}\left(\sqrt{\alpha_{1}}, \ldots, \sqrt{\alpha_{p}}, \sqrt{\beta_{1}}, \ldots, \sqrt{\beta_{q}}\right)$, 使得
$$
    \mathrm{f}(\mathrm{x})=Z_{1}^{2}+\cdots+Z_{P}^{2}-Z_{P+1}^{2}+\cdots-Z_{p+q}^{2}=\alpha_{1} y_{1}^{2}+\cdots+\alpha_{p} y_{p}^{2}-\beta_{1} y_{p+1}^{2}-\cdots-\beta_{q} y_{p+q}^{2}
$$

\vspace{1ex}
{\heiti 49.}设 $\mathrm{V}=\mathbb{P}[x]_{n}$ 是数域$\mathbb{P}$上的 $\mathrm{n}$ 维线性空间,且
$$(\mathrm{f}(\mathrm{x}))=x f^{\prime}(x)-f(x), \forall f(x) \in V$$

(1) 证明: $\mathcal{A}$ 为线性变换

(2) 求 $\mathcal{A}^{-1}(0)$与 $(\mathrm{~V})$

(3) 证明: $\mathrm{V}=\mathcal{A}^{-1}(0) \oplus \mathcal{A}(\mathrm{V})$

证:

(1) $\mathcal{A}(\mathrm{kf}(\mathrm{x})+\lg (\mathrm{x}))=\mathrm{x}[\mathrm{kf}(\mathrm{x})+\lg (\mathrm{x})]^{\prime}-[\mathrm{kf}(\mathrm{x})+\lg (\mathrm{x})]$
$\mathrm{k}\left[\mathrm{xf}(\mathrm{x})^{\prime}-\mathrm{f}(\mathrm{x})\right]+\mathrm{l}\left[\mathrm{xg}(\mathrm{x})^{\prime}-\mathrm{g}(\mathrm{x})\right]$
$=\mathrm{k} \operatorname{Af}(\mathrm{x})+\mathrm{l}_{\mathcal{C}} \mathrm{Ag}(\mathrm{x})$

(2) 由
$\mathrm{f} \in \mathcal{A}^{-1}(0) \Leftrightarrow \mathrm{x} f^{\prime}(x)=f(x) \Leftrightarrow f \equiv$ 或 $\frac{d f}{f}=\frac{1}{x} \Leftrightarrow f(x) \equiv c x$ 知$\mathcal{A}^{-1}(0)=L(x)$

又由
$$
    \begin{aligned}
        \mathcal{A} \mathrm{x}^{k}= & x \cdot k \mathrm{x}^{k-1}-\mathrm{x}^{k}   \\
        =                           & (k-1) \mathrm{x}^{k} \quad(0 \leq k \leq 1) \\
        =                           & \left\{\begin{array}{l}
            (k-1) \mathrm{x}^{k}, k \neq 1 \\
            0, \quad k=1
        \end{array}\right.
    \end{aligned}
$$

知
$$
    \begin{aligned}
        \mathcal{A}(\mathrm{V})= & \mathrm{L}\left(\mathcal{A}(1), \mathcal{A}(\mathrm{x}), \ldots, \mathcal{A}\left(\mathrm{x}^{n}\right)\right) \\
        =                        & \mathrm{L}\left(-1,0, \mathrm{x}^{2}, \ldots,(\mathrm{n}-1) \mathrm{x}^{n}\right)                              \\
        =                        & \mathrm{L}\left(1, \mathrm{x}^{2}, \ldots \mathrm{x}^{n}\right)
    \end{aligned}
$$                                                  

(3)
$$
    \begin{aligned}
        \mathrm{V}= & \mathbb{P}[\mathrm{x}]_{n}                                         \\
        =           & L\left(1, x, \mathrm{x}^{2}, \ldots, \mathrm{x}^{n}\right)         \\
        =           & L(x) \oplus L\left(1, \mathrm{x}^{2}, \ldots \mathrm{x}^{n}\right) \\
        =           & \mathcal{A}^{-1}(0) \oplus \mathcal{A}(\mathrm{V})
    \end{aligned}
$$

\vspace{1ex}
{\heiti 50.}己知 $\mathrm{A}, \mathrm{B}$ 为数域止上的 $\mathrm{n}$ 阶方阵,A有 $\mathrm{n}$ 个互异的特征值.证明: $\mathrm{A}$ 的特征向量是 B 的特征 向量充分必要条件是 AB=BA.

解:
设 A 的 $\mathrm{n}$ 个互异特征值为 $\lambda_{1}, \ldots \ldots \ldots, \lambda_{n}$,对应特征向量为 $\alpha_{1}, \ldots \ldots \ldots, \alpha_{n}$ 

则属于不同特征值得特征向量线性无关

知 $\alpha_{1}, \ldots \ldots \ldots, \alpha_{n}$ 线性无关,是 $\mathbb{P}^{n}$ 的一组基,令$\mathrm{P}=\left(\alpha_{1}, \ldots, \alpha_{n}\right)$则 
$$\mathrm{AP}=\operatorname{Pdiag}\left(\lambda_{1}, \ldots, \lambda_{n}\right) \Rightarrow p^{-1} A P=\operatorname{diag}\left(\lambda_{1}, \ldots, \lambda_{n}\right)$$

再令 $\mathrm{V}_{i}=\left\{\alpha \in \mathbb{P}^{n} ; A \alpha=\alpha\right\}$, 则
$$\operatorname{dim} V_{i}=1 \Rightarrow=L\left(\alpha_{i}\right)$$

可以用反证法证得: $\exists i_{0}$, s.t. $\operatorname{dim} V_{i_{0}} \geq 2$

$\Rightarrow V_{i_{0}}$ 有与 $\alpha_{i}$ 线性无关的向量 $\beta$

$\Rightarrow \alpha_{1}, \ldots \ldots, \alpha_{n}, \beta$ 线性无关(理由: 属于不同特征值的特征向量无关)

$\Rightarrow n=\operatorname{dim} \mathbb{P}^{n} \geq \operatorname{dim} \|_{1}L\left(\alpha_{1}, \ldots, \alpha_{n}, \beta\right)=n+1 \Rightarrow$ 矛盾,故有结(1)

$\Rightarrow$ 设 $\mathrm{A}$ 的特征向量是 B 的特征向量则$\exists \mu_{i} \in \mathbb{P}^{n}, s.t .$

$\Rightarrow p^{-1} B P$

$=\operatorname{diag}\left(\mu_{1} \ldots \mu_{n}\right)$

$\Rightarrow p^{-1} A B P=P^{-1} A P \cdot P^{-1} B P$

$=\operatorname{diag}\left(\lambda_{1} \ldots \lambda_{n}\right) \cdot \operatorname{diag}\left(\mu_{1} \ldots \mu_{n}\right)$

$=\cdot P^{-1} B P \cdot P^{-1} A P$

$\Rightarrow A B=B A$

$(2) \Leftarrow$ 设 $\mathrm{AB}=\mathrm{BA}$, 则

$\mathrm{A}=\mathrm{BA} \alpha_{i}=B \alpha_{i}=\lambda_{i} B \alpha_{i}$

$\Rightarrow \mathrm{B} \alpha_{i}$ 是 $A$ 的属于特征值$\lambda_{i}$ 的特征向量

$\Rightarrow \exists \mathrm{B} \alpha_{i} \in V_{i}$

$\Rightarrow \exists \mu_{i} \in \mathbb{P}$, s.t. $\mathrm{B} \alpha_{i}=\mu_{i} \alpha_{i}\left(\operatorname{dimV}_{i}=1 \Rightarrow V_{i}=L\left(\alpha_{i}\right)\right)$

\end{document}